\documentclass[../main.tex]{subfile}

%page 1

\begin{document}
\part{Introduction}

La mécanique consiste à décrire le mouvement et le prédire.
La cinématique consiste à décrire le mouvement tandis que la dynamique consiste à le prédire.
chap 1: rappels
chap 2: cinématique
chap 3: dynamique


\begin{defi}
	
un solide est un ensemble de points.
On s'interesse plus particulièrement au modèle du solide indéformable.

\end{defi}

\subsection{différence entre le mouvement d'un point et le mouvement d'un solide}
on ajoute des degrès de liberté dû au fait que le point d'application de la force influe sur le mouvement.


\section{Rappels de mécanique du point}

\subsection{Objectif}
On veut décrire et prédire le mouvement d'un point.

\subsection{referentiel}
Il nous faut alors un réferentiel (repère d'espace et du temps)
\subsubsection{repère d'espace}
repère d'espace: une origine et une base
\begin{itemize}
	\item coordonnées cartésiennes 
	$\vec{OM} = x\vec{u_x} + y\vec{u_y} + z\vec{u_z}$
	\item coordonnées cylindriques
	$\vec{OM} = r\vec{u_r} + z\vec{u_z}$
	\item coodrdonnées sphériques
	$\vec{OM} = r\vec{u_r}$
\end{itemize}

\subsubsection{repère de temps}
on a la même horloge dans tous les réferentiels (on est en mécanique classique)

\subsection{degrès de liberté}
nombre de données indépendantes nécessaires pour définir de manière univoque la position d'un système.
pour déterminer le nombre de degrès de liberté d'un système, on somme les ddl de chaque point auxquels on soustrait le nombre de contraintes.

\subsection{cinématique du point}
on souhaite décrire le mouvement $\vec{OM}(t)$
la vitesse $v = \frac{d\vec{OM}}{dt}_\mathcal{R}$
quantité de mouvement $\vec{p} = m \cdot \vec{v}$
l'expression de la vitesse dépend du réferentiel.
en cartésien : $\vec{v} = $%###
en cylindrique : $\vec{v} = \dot{r}\vec{u_r} + r\dot{theta}\vec{u_\theta} +\dot{z}\vec{u_z}$
en sphérique : $\vec{v} = \dot{r}\vec{u_r} + r\dot{theta}\vec{u_\theta} + r\dot{\varphi}\vec{u_\varphi}$

expression de l'accélération:

%###

On introduit alors la ...:
$\vec{d} = m \cdot \vec{a}$

le moment cinétique:
moment cinétique du point $M$ par rapport au point $A$.
$\vec{L_A} = \vec{AM} \wedge \vec{p}$


\subsection{dynamique du point}
principe fonda de la dynamique:
$\sum\limits_i F_i = \frac{d\vec{p}}{dt}$
où les forces $F_i$ sont les forces appliquées au point.
uniquement quand $\mathcal{R}$ est galiléen

th du moment cinétique:
$\frac{d\vec{L_A}}{dt} = \sum\limits_i \vec{\mathcal{M}_A}(\vec{F_i})$

attention, ces équations sont des équations vectorielles
quand on peut ne pas projeter, on ne projète pas


remarque: projection
d'abord on écrit les équation vectorielles 
puis dans un second temps, si nécessaire, on projète

remarque:
le PFD produit trois équations scalaires
un point materiel possède trois degrès de libertés
on peut alors décrire le mouvement d'un point avec le PFD

idem pour le th du moment cinétique (qui produit aussi trois équations scalaires)

ramarque:
dans le théorème du moment cinétique, on fait aparaitre le moment des force, on peut le décrire formellemnt comme suit:
$\vec{\mathcal{M}_A} (\vec{F_i}) = \vec{AM} \wedge \vec{F_i}$

remarque: l'intéret du th du moment cinétique:
le moment en $A$ est nul
permet de montrer qu'une orbite est contenue dans un plan contenant lui même le centre de gravité.

\subsection{l'aspect énergétique de la mécanique du point}
$$
\begin{aligned}
	\frac{d\vec{p}}{dt} &=\sum\limits_i F_i\\
	\vec{v} \cdot \frac{d\vec{p}}{dt} &= \vec{v} \cdot \sum\limits_i F_i\\
	\frac{d}{dt}(\frac{1}{2} m \vec{v}^2) &= \cdot \sum\limits_i F_i \cdot \frac{d}{dt}\vec{l}\\
	\frac{d}{dt}(\frac{1}{2} m \vec{v}^2) &= \cdot \sum\limits_i F_i \cdot \frac{d}{dt}\vec{l}\\
\end{aligned}
dF_C = \delta W\\
F_C = 1 \over 2 m v^2\\
\delta W = \sum\limits_i \vec{F_i} d\vec{l}
$$

%###

jusqu'à la fin de l'exemple du pendule

\section{en mécanique du solide}
pour résoudre des problèmes:
d'abord, la cinématique
puis soit de la dynamique avec le pfd
soit un étude énergétique



\end{document}
