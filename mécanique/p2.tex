\documentclass[../main.tex]{subfile}

%page 2 (postulats de la meca q)

\begin{document}
\part{Chapitre 2 : Cinématique du solide}

\section{Système de points materiels - solide indéformable}
\begin{defi}
	système de points materiels\\
	avec un système de $n$ points notés $M_i = (x_i, y_i, z_i)$ avec $i \in \{1, n\}$
	on veut connaitre le mouvement de chaque points
\end{defi}

\begin{ex}
	exemple du modèle terre lune, utilisation du module fictif
\end{ex}

	nombre de degrès de liberté = 3n (n points de 3 coordonnées)
	une contrainte indépendante des précédentes retire un degrès de liberté
	%####
	exemples des points liés

\begin{defi}
	solide:
	ensemble continu de points materiels
\end{defi}

\begin{defi}
	solide indéformable:
	modèle 
	un solide est dit indéformable si la distance entre deux quelconques de ses points est indépendante du temps.
\end{defi}


\end{document}
