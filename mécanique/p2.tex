\documentclass[../main.tex]{subfile}

%page 2 (postulats de la meca q)

\begin{document}
\part{Chapitre 2 : Cinématique du solide}

\section{Système de points materiels - solide indéformable}
\begin{defi}
	système de points materiels\\
	avec un système de $n$ points notés $M_i = (x_i, y_i, z_i)$ avec $i \in \{1, n\}$
	on veut connaitre le mouvement de chaque points
\end{defi}

\begin{ex}
	exemple du modèle terre lune, utilisation du module fictif
\end{ex}

	nombre de degrès de liberté = 3n (n points de 3 coordonnées)
	une contrainte indépendante des précédentes retire un degrès de liberté
	%####
	exemples des points liés

\begin{defi}
	solide:
	ensemble continu de points materiels
\end{defi}

\begin{defi}
	solide indéformable:
	modèle 
	un solide est dit indéformable si la distance entre deux quelconques de ses points est indépendante du temps.
\end{defi}

\begin{defi}
	Centre de masse:
	Soit un système de points matériels :
	$$
	G :
\begin{aligned}
	\vec{OG} &= \frac{\sum\limits_i m_i \vec{OM_i}}{\sum\limits_i m_i}\\
	\sum\limits_i m_i \vec{GM_i} &= \vec{0}
\end{aligned}
	$$
	Où $G$ est le centre de masse, barycentre, ou centre d'inertie
\end{defi}

\begin{rema}
	le centre de masse est différent du centre de gravité, qui est le point d'application du poids
\end{rema}

\begin{ex}
	sur un nuage : %###inclure fig 21
\end{ex}

\begin{defi}
	expression du centre de masse pour un solide:
	$$\vec{OG} = \frac{\iiint \rho(M) \vec{OM} dV}{\iiint \rho(M) dV}$$
	où $\rho(M)$ est la masse volumique au point $M$ \\
	la masse totale $M_t$ s'exprimant alors:
	$$M_t = \iiint \rho(M) dV$$
\end{defi}

\begin{rema}
	lien entre intégrale et somme: (en 1D)\\
	$$\int\limits_a^b f(x)dx = \lim\limits_{n \to +\infty} \sum\limits_{i = 1}^n f(a+(\frac{b-a}{n})i)(\frac{b-a}{n})$$
\end{rema}

\begin{rema}
	formule vraie pour tout $O$:
	$$\iiint \rho(M) \vec{GM} dV = \vec{0}$$
\end{rema}

\begin{rema}
	$G$ peut ne pas appartenir au solide 
	%### inclure l'exemple du fer à cheval
\end{rema}

\subsection{Résultante cinétique}
\begin{defi}
	$$\vec{P(t)} = \iiint \rho(M) \vec{v}(M, t)_{/R} dV$$
	pour un solide\\
	%un truc
\end{defi}

\begin{defi}
	Champs de vitesses:\\
	$\vec{v(M, t)}$ est la vitesse d'un point situé en $M$ à l'instant $t$.
\end{defi}

\begin{propri}
	propriétés de $\vec{P}$:\\
	Pour un système de points materiels:
	$$
\begin{aligned}
	\vec{P} &= \sum_i m_i \vec{v(M_i)}\\
	&= \sum_i m_i \frac{d}{dt} \vec{OM_i}\\
	&= \frac{d}{dt}
\end{aligned}
	$$
	%a fiiiiiiiinir###
\end{propri}

\subsection{Moment cinétique, torseur cinétique}
\begin{defi}
	Soit un système de points materiels:\\
	$$\vec{L_A} = \sum\limits_i \vec{AM_i} \wedge m_i \vec{v(M_i)}$$
	$$\vec{L_A} = \iiint \vec{AM} \wedge \rho(M) \vec{v(M)} dV$$
	$\vec{L_A}$ est alors le moment cinétique en un point $A$.\\
	$A$ appartient ou pas au solide\\

	$\vec{L_A}$ est un champ de vecteurs, à chaque point $A$ de l'espace, je peux calculer $\vec{L_A}$
\end{defi}

\begin{propri}
	quelles relations y-a-t'il entre $\vec{L_A}$ et $\vec{L_B}$?\\
	$$
\begin{aligned}
	\vec{L_B} &= \iiint \vec{BM} \wedge \rho (M) \vec{v(M)} dV\\
	&= \iiint (\vec{BA} + \vec{AM}) \wedge \rho(M) \vec{v}(M) dV\\
	&= \vec{BA} \wedge \iiint \rho(M) \vec{v(M)} dV + \iiint \vec{AM} \wedge \rho(M)\vec{v(M)} dV\\
	&= \vec{BA} \wedge \vec{P} + \vec{L_A}
\end{aligned}
	$$

	on a donc les deux formules suivantes (équivalentes, en connaitre une par coeur)
	$$\vec{L_B} = \vec{L_A} + \vec{P} \wedge \vec{AB}$$
	$$\vec{L_B} = \vec{L_A} + \vec{BA} \wedge \vec{P}$$

	On peut calculer $\vec{L_B}$ sur tout les $B$ de l'espace en sachant $\vec{L_A}$ et $\vec{P}$
\end{propri}

\begin{defi}
	$(\vec{L_A}, \vec{P})$ est un torseur cinétique\\
	un torseur est un champ de vecteurs avec la propriété d'équiprojectivité:\\
	$$\vec{L_A} \cdot \vec{AB} = \vec{L_B} \cdot \vec{AB}$$
	$$\exists \vec{P} \ / \ \vec{L_B} = \vec{L_A} + \vec{P} \wedge \vec{AB}$$
\end{defi}

\subsubsection{Moment cinétique par rapport a un axe}
$$L_\Delta = \vec{L_A} \cdot \vec{u_\Delta}$$
avec $\Delta$, un axe orienté, \\
et $\vec{u_{\Delta}}$, le vecteur unitaire de cet axe.
%###fig 23
%### reprendre jusqu'à

\subsubsection{Energie cinétique}
$$E_c = \iiint \frac{\rho(M)}{2} \vec{v(M)}^2 dV$$


\begin{rema}
	memo:
\begin{itemize}	
	\item barycentre:
	$$\vec{OG} = \frac{\iiint \rho(M) \vec{OM} dV}{\iiint \rho(M) dV}$$
	$$\vec{OG} = \iiint \rho(M) \vec{GM} dV$$
	
	\item torseur cinétique:
	$$\vec{P} = \iiint \rho(M)\vec{V(\rho)}$$
	$$\vec{L_A} = \iiint \vec{AM} \wedge \rho(M) \vec{v(M)} dV$$
	$$\vec{L_B} = \vec{L_A} + \vec{P} \wedge \vec{AB}$$

	\item torseur dynamique:
	$$\vec{D} = \iiint \rho{M} \vec{a(M)} dV$$
	$$\vec{\delta_A} \iiint \vec{AM} \wedge \rho(M) \vec{a(M)} dV$$
	$$\vec{\delta_B} = \vec{\delta_A} + \vec{D} \wedge \vec{AB}$$

	\item passage du torseur cinétique au torseur dynamique:
	$$\vec{D} = \frac{d}{dt} \vec{P}$$
	$$\vec{\delta_A} = \frac{d}{dt} \vec{L_A} + \vec{V_A} \wedge \vec{P}$$
\end{itemize}
\end{rema}

\section{Réferenciel barycentrique, théorème de Koenig}
notion spécifique aux systèmes de points materiels et aux solides .\\
n'a pas d'intérêt pour un système d'un point materiel.

\subsection{Réferentiel Barycentrique}
\subsubsection{Définitions}
\begin{defi}
	Le réferenciel barycentrique $R^*$ a pour origine, même horloge et même repère d'espace que $R$
	%### inclure figure 24
	$(R^*)$ possédant les mêmes axes que $(R)$, il est en translation par rapport à $(R)$
\end{defi}
	%###reprendre les exemples

\begin{nota}	
	La vitesse de $M$ dans le réferenciel $(R^*)$ se note $\vec{v^*(M)} = \frac{d}{dt}(\vec{GM})$
\end{nota}

\subsection{Resultante cinétique et dynamique dans $(R^*)$}

\begin{defi}
	$$
\begin{aligned}
	\vec{P^*} &= \iiint \rho(M) \vec{v^*(M)} dV\\
	&= \iiint \rho(M) \frac{d}{dt} \vec{GM} dV\\
	&= \frac{d}{dt} (\iiint \rho(M) \vec{GM} dV)\\
	&= \vec{0}\\
\end{aligned}
	$$
\end{defi}

\begin{propri}
	$\vec{P^*} = \vec{0}$\\
	$$\vec{D^*} = \iiint \rho(M) \vec{a^*(M)} dV$$
	$$\vec{D^*} = \vec{0}$$
\end{propri}

\subsection{Moments cinétique et dynamique dans $R^*$}

\begin{defi}
	$$\vec{L_A} = \iiint \vec{AM} \wedge \rho(M) \vec{v^*(M)} dV$$
	$$\vec{\delta^*_A} = \iiint \vec{AM} \wedge \rho(M) \vec{a^*(M)} dV$$
\end{defi}

\begin{propri}
	$$\vec{L^*_B} = \vec{L^*_A} + \vec{P^*} \wedge \vec{AB}$$
	$$\vec{L^*_B} = \vec{L^*_A} = \vec{L^*}$$
	le moment cinétique est indépendant du point dans $R^*$

	$$\vec{\delta^*_B} = \vec{\delta^*_A} + \vec{D^*} \wedge \vec{AB}$$
	$$\vec{\delta^*_B} = \vec{\delta^*_A} = \vec{\delta^*}$$
	le moment dynamique est indépendant du point dans $R^*$
\end{propri}

\subsection{Théorèmes de Koenig}
\begin{theo}
	Relations entre :\\
\begin{itemize}	
	\item $\vec{L_A |_{(R)}}$ et $\vec{L^*}$
	\item $\vec{L_\delta |_{(R)}}$ et $\vec{\delta^*}$
	\item $E_c |_{(R)}$ et $E_c^* = \iiint \frac{1}{2} \rho(M) \vec{v^*}(M) dV$
\end{itemize}
\end{theo}

\subsubsection{Moment cinétique:}
$$
\begin{aligned}
	\vec{L_A}|_{(R)}
\end{aligned}
$$
%### jusqu'au memo


\section{Champ des vitesses dans un solide indéformable}

\begin{rap}
	Pour connaitre le mouvement d'un solide, il faut connaitre sa trajectoire et sa vitesse en chacuns de ses points.\\
	On définit alors le champ des vitesses
\end{rap}

\begin{defi}
	$\vec{v(M)}$, la vitesse du point situé en $M$ au temps $t$
\end{defi}

\subsection{Degrès de liberté}
tout solide (points matériels dont la distance entre eux est fixée) possède 6 degrès de liberté.\\

on utilise la position du barycentre (qui fixe 3 degrès de liberté)
ainsi que trois angles %###insérer fig 25
(qui fixe les 3 autres degrès de liberté)

valable uniquement pour les solides indéformables

\subsection{Torseur des vitesses}

définition du solide indéformable, pour tout points $A$ et $B$ du solide, \\
la distance $AB$ étant constante, $\vec{AB}^2$ est aussi constant, d'où,
$$\vec{AB} \cdot [\vec{v(B)} - \vec{v(A)}] = 0 \Rightarrow \vec{v(B)} \cdot \vec{AB} = \vec{v(A)} \cdot \vec{AB}$$
Le champ des vitesses est équiprojectif, i.e. en trois dimensions, $\exists! \vec{\Omega} \ / \ \vec{v(B)} = \vec{v(A)} + \vec{\Omega} \wedge \vec{AB}$

\begin{defi}
	On définit alors un couple avec le champs des vitesses et un torseur des vitesses (ou torseur cinématique)
	$$(\vec{v(A)}, \vec{\Omega})$$
\end{defi}

\begin{rema}
	$$\vec{v(B)} = \vec{v(A)} + \vec{\Omega} \wedge \vec{AB}$$
	$$\vec{v(B)} \cdot \vec{AB} = \vec{v(A)}  \cdot \vec{AB}+ (\vec{\Omega} \wedge \vec{AB}) \cdot \vec{AB}$$
\end{rema}

\begin{rema}
	Le champ des vitesses peut varier en fonction du temps, \\
	à $t$ donné, $(\vec{v(A, t)}, \vec{\Omega(t)})$
\end{rema}

\begin{rema}
	$\vec{\Omega}$ est homogène à l'inverse d'un temps.
\end{rema}

\subsection{Exemples de mouvements de solide}

On se donne $(\vec{v(A)}, \vec{\Omega})$, on souhaite alors connaitre la nature du mouvement.
\subsubsection{$\vec{v(A)} \neq \vec{0}, \ \vec{\Omega} = \vec{0}$, le mouvement est alors une translation}

Pour tout point $B$ du solide, $\vec{v(B)} = \vec{v(A)}$, il s'agit alors d'une translation

%###insérer fig 26, 27, 28
\subsubsection{$\exists A \ / \ \vec{v(A)} = \vec{0}, \ \vec{\Omega} \neq \vec{0}$, rotation autour d'un axe fixe}
Pour tout point $M$ du solide, $\vec{v(M)} = \vec{v(A)} + \vec{\Omega} \wedge \vec{AM} = \vec{\Omega} \wedge \vec{AM}$
%finir l'exemple


\subsubsection{$\vec{\Omega} \neq \vec{0}, \ \exists A \ / \ \vec{v(A)} \neq \vec{0}$ et $\vec{v(A)}$ et $\vec{\Omega}$ sont colinéaires, il s'agit alors d'un mouvement hélicoïdal}
%###demo

\subsubsection{$\vec{\Omega} \neq \vec{0}, \ \vec{v(A)} \neq \vec{0}$, le mouvement est quelconque}
On montre qu'il existe un point $P$ tel que $\vec{v(P)}$ et $\vec{\Omega}$ sont colinéaires, de manière à se rapporter à un mouvement hélicoïdal autour d'un axe passant par $P$ appelé axe instantané de rotation.\\
l'axe instantané de rotation peut varier dans le temps

\subsection{Comment déterminer le vecteur rotation d'un solide pour un mouvement donné}
On prend deux points quelconques $M_1$ et $M_2$.\\
On s'intéresse alors à $\vec{M_1M_2}(t)$ et $\vec{M_1M_2}(t + dt)$\\
S'ils sont égaux, il s'agit d'un mouvement de translation\\






On se place alors dans $(R^*)$, \\
$\vec{\Omega}$, est le même dans $(R)$ et $(R^*)$
$\vec{v^*(G)} = \vec{0}$
On tante alors de determiner un axe de rotation tel que $\{\Omega \ | \ \vec{v^*(M)} = \vec{0}\}$\\
On prend $M_1$ dans un plan perpendiculaire à l'axe passand par $G$\\
$$\vec{GM_1}(t + dt) - \vec{GM_1(t)} = \vec{v(M_1)}dt = \vec{\Omega} \wedge \vec{GM_1} dt$$






\subsection{Exemples de détermination de $\vec{\Omega}$}
\subsubsection{Le pendule}
%### les schémas.............................

\end{document}





































