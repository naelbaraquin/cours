\documentclass[../main.tex]{subfile}

%page 2

\begin{document}
\part{Loi de coulomb et champs et potentiels électrostatiques}
\section{Loi de Coulomb}
dans la charge ponctuelle:
décrit la relation entre le potentiel, la charge et la force
de manière empirique: on s'aperçoit que 
la force qui s'exerce entre les deux charges est inversement proportionnelle au carré de la distance
la force est portée par la droite donnée par les deux charges
additivité vectorielle des forces
si deux charges de même nature, elles s'attirent
de nature différentes, elle se repoussent

vrai truc:
repose sur des constatations expérimentales:
\begin{itemize}
	\item on a additivité vectorielle des actions entre charges ponctuelles:\\
	Si $F_{A \to C}$ est la force exercée par $q_A$ sur $q_C$\\
	Si $F_{B \to C}$ est la force exercée par $q_B$ sur $q_C$\\
	alors $\vec{F_{A \to C} + \vec{F_{B \to C}}}$ est l'action simultanée de $q_A$ et $q_B$ sur $q_C$
	\item on a opposition des actions réciproques
	$$\vec{F_{A \to B}} = -\vec{F_{B \to A}}$$
	\item expérimentalement, on met en évidence que $||\vec{F_{A \to B}}|| = \frac{\varphi(AB)}{AB^2}$ où $\varphi(AB)$ est un scalaire indépendant des positions de $A$ et $B$
	\item expérimentalement, on met en évidence que $\vec{F_{A \to B}}$ est porté par $\vec{AB}$
	$$\vec{F_{A \to B}} = \frac{\varphi(AB)}{AB^2} \frac{\vec{AB}}{||\vec{AB}||}$$
	le dernier rapport est alors $\vec{u}$, vecteur unitaire de $\vec{AB}$
	\item on montre expérimentalement que 
	$$\frac{F_{A \to B}}{q_A} = \frac{F_{A' \to B}}{q'_A}$$
	La comparaison des forces permet la mesure des charges
	\item on a l'expression de la force de Coulomb, force électrostatique qui s'exerce entre deux charges $q_A$ et $q_B$ placées dans le vide aux points $A$ et $B$:
	$$\vec{F_{A \to B}} = \frac{q_A \cdot q_B}{4 \pi \varepsilon_0} \frac{\vec{u}}{AB^2}$$
	où $\varepsilon_0$ est la permittivité électrique du vide\\
	$\frac{1}{4 \pi \varepsilon_0} \approx 9\cdot 10^9$ dans le SI
\end{itemize}

La force électrique est très grande devant la force gravitationnelle à très petite échelle

\section{Champs électrostatique}
\subsection{Champs créé par une charge ponctuelle}
Soit $q_A$ au point $A$, si on approche $q_{B_1}, q_{B_2}, ...$ du point $B$:
$$\frac{\vec{F_{A \to B_1}}}{q_{B_1}} = \frac{\vec{F_{A \to B_2}}}{q_{B_2}} = ...$$
Ce rapport dépend de $q_A$ et du point $B$.\\
Ce rapport est appelé champs électrostatique (ou champs électrique) créé par la charge $q_A$ au point $B$.

$$\vec{E_A}(\vec{B}) = \frac{1}{4 \pi \varepsilon_0} \frac{q_A}{AB^2} \vec{u}$$
est l'expression du champs électrostatique, son unité est le $V \cdot m^{-1}$
$$[E] = V \cdot m^{-1}$$
en général, on a la notation suivante:\\
avec $P$, le point où se trouve la charge\\
et $M$, le point dans lequel on observe le champ
$$\vec{E_P}(M) = \frac{q}{4\pi \varepsilon_0} \frac{\vec{PM}}{PM^3}$$
avec $\frac{\vec{PM}}{PM^3} = \frac{1}{PM^2} \frac{\vec{PM}}{PM}$\\

\begin{rema}
	Si $q_M$ se trouve en $M$\\
	et $\vec{F_{P \to M}} = q_M \vec{E_P}(M)$, l'expression de la loi de Coulomb\\
	La mesure de la force correspond à la mesure du champs électrostatique
\end{rema}

\begin{rema}
	$E_P(M)$ n'est pas définit en sa source car $\frac{1}{PM^2} \to \infty$
	Correspond à la limite de validité du concept de charge ponctuelle
\end{rema}


\section{Champ créé par un ensemble de charges ponctuelles}

$q$ en $M$ est soumis à l'action de $N$ charges $q_i$ aux points $P_i$:
$$\vec{F(M)} = \frac{q}{4\pi\varepsilon_0} \sum\limits_{i=1}^N q_i \frac{\vec{P_iM}}{P_iM^3}$$
$$\Rightarrow \vec{E(M)} = \frac{\vec{F(M)}}{q} = \sum\limits_{i=1}^N \frac{q_i}{4\pi\varepsilon_0} \frac{\vec{P_iM}}{P_iM^3}$$

\section{Champ créé par une distribution quelconque de charges}
Soit $\Sigma$, une distribution de charges volumique de charges de densité volumique $\rho_c$ située dans un volume $V$.\\
Soit $M$, un point de l'espace situé assez loin de la distribution $\Sigma$ de charges.\\
%### fig 21
$dq$, la charge élémentaire dans $dV$ est égale à $dq = \rho_cdV$.\\
Le champ élémentaire $d\vec{E(M)}$ créé par $dq$ au point $P$ s'écrit:
$$
\begin{aligned}
	d\vec{E(M)} &= \frac{dq}{4\pi\varepsilon_0} \frac{\vec{PM}}{PM^3}\\
	&= \frac{\rho_c(P) dV(P)}{4\pi\varepsilon_0}\frac{\vec{PM}}{PM^3}
\end{aligned}
$$

Le champ $\vec{E(M)}$ créé par la distribution $\Sigma$ s'écrit:
$$\vec{E(M)} = \int d\vec{E(M)} = \iiint_V \frac{\rho_c(P)dV}{4\pi\varepsilon_0} \frac{\vec{PM}}{PM^3}$$

L'intégration porte sur les variables qui caractérisent le point $P$ (appelé point potentiant; $M$, fixe, étant le point potentié).\\
Si $\vec{OM} = \vec{r}$ et $\vec{OP} = \vec{r'}$, alors:
$$\vec{E(M)} = \iiint_V \frac{\vec{r} - \vec{r'}}{4\pi\varepsilon_0 ||\vec{r} - \vec{r'}||^3} \rho_c(\vec{r'}) dV(\vec{r'})$$

De même, si la distribution de charges est surfacique : 
$$\vec{E(M)} = \int d\vec{E(M)} = \iint_S \frac{\sigma(P) dS}{4\pi\varepsilon_0} \frac{\vec{PM}}{PM^3}$$

De même, si la distribution de charges est linéique : 
$$\vec{E(M)} = \int d\vec{E(M)} = \int_l \frac{\lambda(P) dl}{4\pi\varepsilon_0} \frac{\vec{PM}}{PM^3}$$

\section{Lignes de champ}
Le champ est un vecteur que l'on peut visualiser à l'aide de lignes de champ.\\
Ces lignes de champ correspondent à l'ensemble des courbes orientées telles que leur tangente en chaque point ait la même direction et le même sens que le champ.


%### finir partie 2


\end{document}
















