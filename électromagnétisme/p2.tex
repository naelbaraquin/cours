\documentclass[../main.tex]{subfile}

%page 2

\begin{document}
\part{Loi de coulomb et champs et potentiels électrostatiques}
\section{Loi de Coulomb}
dans la charge ponctuelle:
décrit la relation entre le potentiel, la charge et la force
de manière empirique: on s'appercoit que 
la force qui s'exerce entre les deux charges est inversement proportionnelle au carré de la distance
la force est portée par la droite donnée par les deux charges
additivité vectorielle des forces
si deux charges de meme nature, elles s'attirent
de nature différentes, elle se repoussent

vrai truc:
repose sur des constatations experimentales:
\begin{itemize}
	\item on a additivité vectorielle des actions entre charges ponctuelles:\\
	Si $F_{A \to C}$ est la force exercée par $q_A$ sur $q_C$\\
	Si $F_{B \to C}$ est la force exercée par $q_B$ sur $q_C$\\
	alors $\vec{F_{A \to C} + \vec{F_{B \to C}}}$ est l'action simulténée de $q_A$ et $q_B$ sur $q_C$
	\item on a opposition des actions réciproques
	$$\vec{F_{A \to B}} = -\vec{F_{B \to A}}$$
	\item experimentalement, on met en évidence que $||\vec{F_{A \to B}}|| = \frac{\varphi(AB)}{AB^2}$ où $\varphi(AB)$ est un scalaire indépendant des positions de $A$ et $B$
	\item experimentalement, on met en évidence que $\vec{F_{A \to B}}$ est porté par $\vec{AB}$
	$$\vec{F_{A \to B}} = \frac{\varphi(AB)}{AB^2} \frac{\vec{AB}}{||\vec{AB}||}$$
	le dernier rapport est alors $\vec{u}$, vecteur unitaire de $\vec{AB}$
	\item on montre experimentalement que 
	$$\frac{F_{A \to B}}{q_A} = \frac{F_{A' \to B}}{q'_A}$$
	La comparaison des forces permet la mesure des charges
	\item on a l'expression de la force de Coulomb, force électrostatique qui s'exerce entre deux charges $q_A$ et $q_B$ placées dans le vide aux points $A$ et $B$:
	$$\vec{F_{A \to B}} = \frac{q_A \cdot q_B}{4 \pi \varepsilon_0} \frac{\vec{u}}{AB^2}$$
	où $\varepsilon_0$ est la permittivité électrique du vide\\
	$\frac{1}{4 \pi \varepsilon_0} \approx 9\cdot 10^9$ dans le SI
\end{itemize}

La force electrique est très grande devant la force gravitationnelle à très petite échelle

\section{Champs électrostatique}
\subsection{Champs créé par une charge ponctuelle}
Soit $q_A$ au point $A$, si on approche $q_{B_1}, q_{B_2}, ...$ du point $B$:
$$\frac{\vec{F_{A \to B_1}}}{q_{B_1}} = \frac{\vec{F_{A \to B_2}}}{q_{B_2}} = ...$$
Ce rapport dépend de $q_A$ et du point $B$.\\
Ce rapport est appelé charmps électrostatique (ou champs électrique) créé par la charge $q_A$ au point $B$.

$$\vec{E_A}(\vec{B}) = \frac{1}{4 \pi \varepsilon_0} \frac{q_A}{AB^2} \vec{u}$$
est l'expression du champs électrostatique, son unité est le $V \cdot m^{-1}$
$$[E] = V \cdot m^{-1}$$
en général, on a la notation suivante:\\
avec $P$, le point où se trouve la charge\\
et $M$, le point dans lequel on observe le champ
$$\vec{E_P}(M) = \frac{q}{4\pi \varepsilon_0} \frac{\vec{PM}}{PM^3}$$
avec $\frac{\vec{PM}}{PM^3} = \frac{1}{PM^2} \frac{\vec{PM}}{PM}$\\

\begin{rema}
	Si $q_M$ se trouve en $M$\\
	et $\vec{F_{P \to M}} = q_M \vec{E_P}(M)$, l'expression de la loi de Coulomb\\
	La mesure de la force correspond à la mesure du champs électrostatique
\end{rema}

\begin{rema}
	$E_P(M)$ n'est pas définit en sa source car $\frac{1}{PM^2} \to \infty$
	Correspond à la limite de validité du concept de charge ponctuelle
\end{rema}

\end{document}
