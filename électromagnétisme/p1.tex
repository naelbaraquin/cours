\documentclass[../main.tex]{subfile}

%page 1

\begin{document}
\section{informations utiles}
\part{Charge élémentaire, densité de charges}

\section{Introductions}
\begin{prop}
$\phantom{a}$
	%electron > ambre
	%experiences qui donnent lieu à l'electrostatique
\begin{itemize}
	\item les corps électrisés exercent des forces
	\item l'électrisation peut être transféré d'un corps à l'autre
	\item par convention, l'électron possède un charge négative
	\item deux corps de même électrisation se repoussent 
	\item deux corps d'électrisation différente s'attirent
	\item la charge électrique est une grandeur extensive
	\item invariance de la charge électrique
	\item la charge est conservative
\end{itemize}
\end{prop}

\section{Répartition de charges, densité de charges}
Dans la matière, des quantités astronomiques de charges impliquent qu'on introduise la notion de densité de charges
densités de charges volumiques, surfaciques, linéiques, ponctuelles

\subsection{Densité volumique de charges}
%### insérer fig1
Le volume $dV$ contient $dQ$ charges\\
Soit $\rho_c$, le nombre de charges par unité de volume\\
on a donc $[\rho_c] = C \cdot m^{-3}$\\
$dQ = \rho_cdV$
On appelle $\rho_c$ la densité volumique de charges\\
$$Q \int dQ = \iiint\limits_V \rho_cdV$$
Si $\rho_c$ est constant, alors $Q = \rho_c \cdot V$ avec $V$, le volume de système total.\\
En général, $\rho_c$ varie dans l'espace :\\
$$\rho_c = \rho_c(x, y, z)$$
$$\rho_c = \rho_c(r, \vartheta, \varphi)$$
$$\rho_c = \rho_c(\rho, \varphi, z)$$

\subsection{Densité surfacique de charges}
Les charges sont ici distribuées sur une surface.\\
On introduit la densité surfacique de charges, notée $\sigma$
%###insérer fig2
Soit $dQ$, les charges sur $dS$\\
$dQ = \sigma dS$\\
$$Q = \iint \sigma dS$$
vaut $\sigma S$ si $\sigma$ est constant\\
sinon il faudra connaitre $\sigma(x, y, z)$ et l'intégrer\\
$$[\sigma] = C \cdot m^{-2}$$

\subsection{Densité linéique de charges}
Les charges sont ici distribuées sur une ligne (système dont deux des dimensions sont négligeables devant la troisième)\\
On introduit la densité linéique de charges, notée $\lambda$\\
%### insérer fig3
Soit $dQ$, les charges sur $dL$\\
$$dQ = \lambda dL$$
$$Q = \int \lambda dL$$
vaut $\lambda L$ si $\lambda$ est constant\\
sinon il faudra intégrer $\lambda(x, y, z)$\\
$$[\lambda] = C \cdot m^{-1}$$

\subsection{Charges ponctuelles}
Pour les charges ponctuelles, on ne définit pas de densité de charges.
\end{document}
