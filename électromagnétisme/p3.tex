\documentclass[../main.tex]{subfile}

%page x

\begin{document}

\part{Énergie potentielle et Potentiel}

\section{Énergie potentielle d'interaction d'une particule chargée avec un système de charges}

Soit $q_i$, des charges aux points $p_i$, elles exercent sur la charge $q$ placée en $M$, une force $\vec{F}$ telle que :\\
$$\vec{F} = q \sum\limits_i \frac{q_i\vec{P_iM}}{4\pi\varepsilon_0 P_iM^3}$$

\subsection{Travail élémentaire de la force électrostatique}
Le travail $\delta \tau$ élémentaire de la force $\vec{F}$ au cours d'un déplacement $d\vec{r}$ de $q$ qui se situe en $M$ s'écrit:
$$\delta\tau = \vec{F} \cdot d\vec{r} = \frac{q}{4\pi\varepsilon_0} \sum\limits_i \frac{q_i \vec{P_iM}}{P_iM^3} \cdot d\vec{r}$$

%###fig 22

\subsection{Énergie potentielle associée}
Le travail élémentaire de la force électrostatique apparaît comme l'opposé de la différentielle d'une fonction notée $\varepsilon_{p_e}$ et appelée énergie potentielle d'interaction de la charge $q$ avec le système $\{q_i\}$\\
On écrit $\delta\tau = -de \varepsilon_{p_e}$ avec $\varepsilon_{p_e}(\vec{r}) = \frac{q}{4\pi\varepsilon_0} \sum\limits_i \frac{q_i}{||\vec{r} - \vec{r_i}||} + C$ où $C$ est une constante.\\

Par conséquent, le travail apparaît comme la variation d'une fonction de la position.\\
Le travail $\tau$ ne dépend pas des différents chemins suivis par la particule entre les positions initiales et finales.\\

\begin{rema}
	On dit que le force est à "circulation" conservative.\\
	On dit aussi que cette force dérive d'une énergie potentielle.
\end{rema}

Il en résulte que $\oint_\mathcal{C} \vec{F} d\vec{r} = 0$
Pour tout chemin $\mathcal{C}$ emprunté.
%### fig 23

$$\varepsilon_{p_e} = \frac{q}{4\pi\varepsilon_0} \sum\limits_i \frac{q_i}{||\vec{r} - \vec{r_i}} + C$$

\section{Potentiel créé par un ensemble de charges}
\subsection{Potentiel créé par un ensemble de charges ponctuelles}
Par définition, le potentiel créé par le système de charges $\{q_i\}$ aux points $P_i$ en un point $M$ est l'énergie potentielle par unité de charge en $M(\vec{r})$.\\
On le note $V$ (plus précisément $V(M)$ ou $V(\vec{r})$).

$$
\begin{aligned}
	V(\vec{r}) &= \frac{\varepsilon_{p_e}(\vec{r})}{q}\\
	&= \frac{1}{4\pi\varepsilon_0} \sum\limits_i \frac{q_i}{||\vec{r} - \vec{r_i}||}\\
	&= \frac{1}{4\pi\varepsilon_0} \sum\limits_i \frac{q_i}{P_iM}
\end{aligned}
$$

$$V(M) = \frac{q}{4\pi\varepsilon_0} \frac{1}{PM}$$

\begin{rema}
	Contrairement à la mécanique du point, $\varepsilon_{p_e}$ et $V$ ne sont pas toujours du même signe (la charge peut être négative).\\
	L'unité de potentiel est le Volt.
\end{rema}

\subsection{Cas de distribution continue de charges}
$$V(M) = \iiint \frac{\rho_c d \vvv}{4\pi\varepsilon_0} \frac{1}{PM}$$
Si on a une distribution volumique de charges de densité $\rho_c$

$$V(M) = \iint \frac{\sigma dS}{4\pi\varepsilon_0} \frac{1}{PM}$$
Si on a une distribution surfacique de charges de densité $\sigma$

$$V(M) = \int \frac{\lambda dl}{4\pi\varepsilon_0} \frac{1}{PM}$$
Si on a une distribution linéique de charges de densité $\lambda$

\section{Surfaces équipotentielles}
Il s'agit de l'ensemble des points de l'espace tels que $V(\vec{r})$ est constant.

Par exemple, pour une charge ponctuelle, les surfaces équipotentielles sont des sphères.\\

On peut noter que les lignes de champs sont perpendiculaires aux surfaces équipotentielles
le champ $E$ est perpendiculaire aux surfaces équipotentielles 
et que le champ est dirigé selon les potentiels décroissants

\section{Relations entre $\vec{E}$ et $V$}
\subsection{Forme intégrale}
On a vu que $\delta \tau = \vec{F} d\vec{r} = -d \varepsilon_{p_e}$\\
Le long d'un chemin entre deux points $A$ et $B$ : $\varepsilon_{p_e}(A) - \varepsilon_{p_e}(B) = \int_A^B \vec{F} \cdot d\vec{r}$.
En introduisant le fait que $V = \frac{\varepsilon_{p_e}}{q}$ et $\vec{E} = \frac{\vec{F}}{q}$, on a :
$$V(A) - V(B) \neq \int_B^A dV = \int_B^A -\vec{E}d\vec{r} = \int_A^B \vec{E}d\vec{r}$$
$$\Rightarrow V(A) - V(B) = \int_B^A -\vec{E} d\vec{r}$$

\begin{rema}
	Si $A = B$, $\oint_\CCC \vec{E}d\vec{l} = 0$ où $\CCC$ est une courbe fermée.\\
	$\vec{E}$ est alors dit à circulation conservative.
\end{rema}

\begin{theo}{de Stoke}
	$$\iint \vec{r \circ t} \vec{E} dS = \oint_\CCC \vec{E} d\vec{l}$$
	ici:  $\oint_\CCC \vec{E} d\vec{l} = 0 \Rightarrow \vec{R \circ t} \vec{E} = 0$
\end{theo}

\subsection{Forme locale}
On a vu $U = V(A) - V(B) = \int_B^A \vec{E} d\vec{r}$\\
D'où $dV = -\vec{E} d\vec{r}$
$$\vec{E} 
\left|
\begin{array}{l}
	E_x\\
	E_y\\
	E_z
\end{array}
\right.$$

$$\vec{l} 
\left|
\begin{array}{l}
	dx\\
	dy\\
	dz
\end{array}
\right.$$

$$\Rightarrow dV = - E_xdx - E_ydy - E_zdz = \frac{\partial V}{\partial x}dx + \frac{\partial V}{\partial y}dy + \frac{\partial V}{\partial z}dz$$
$$\Rightarrow 
\left|
\begin{array}{l}
	E_x\\
	E_y\\
	E_z
\end{array}
\right.
= 
\left|
\begin{array}{l}
	-\frac{\partial V}{\partial x}\\
	-\frac{\partial V}{\partial y}
	-\frac{\partial V}{\partial z}
\end{array}
\right.
= 
-\vec{grad(V)}
$$

$$\Rightarrow \vec{E} = -\vec{grad(V)}$$

\begin{rema}
	Soit deux points $A$ et $B$ très proches\\
	$$\int_B^A dV = \int_B^A -\vec{E} d\vec{r}$$
	appartient au même équipotentiel que $V(A) = V(B)$
	$$
\begin{aligned}
	dV &= -\vec{E} d\vec{r}\\
	&= -\vec{E} \cdot \vec{AB}\\
	&= 0
\end{aligned}
	$$
	On en déduit que $\vec{E}$ et $\vec{AB}$ sont orthogonaux.
\end{rema}













\end{document}






















