\documentclass[../main.tex]{subfile}

%page x

\begin{document}

\part{Énergie potentielle et Potentiel}

\section{Énergie potentielle d'interaction d'une particule chargée avec un système de charges}

Soit $q_i$, des charges aux points $p_i$, elles exercent sur la charge $q$ placée en $M$, une force $\vec{F}$ telle que :\\
$$\vec{F} = q \sum\limits_i \frac{q_i\vec{P_iM}}{4\pi\varepsilon_0 P_iM^3}$$

\subsection{Travail élémentaire de la force électrostatique}
Le travail $\delta \tau$ élémentaire de la force $\vec{F}$ au cours d'un déplacement $d\vec{r}$ de $q$ qui se situe en $M$ s'écrit:
$$\delta\tau = \vec{F} \cdot d\vec{r} = \frac{q}{4\pi\varepsilon_0} \sum\limits_i \frac{q_i \vec{P_iM}}{P_iM^3} \cdot d\vec{r}$$

%###fig 22

\subsection{Énergie potentielle associée}
Le travail élémentaire de la force électrostatique apparaît comme l'opposé de la différentielle d'une fonction notée $\varepsilon_{p_e}$ et appelée énergie potentielle d'interaction de la charge $q$ avec le système $\{q_i\}$\\
On écrit $\delta\tau = -de \varepsilon_{p_e}$ avec $\varepsilon_{p_e}(\vec{r}) = \frac{q}{4\pi\varepsilon_0} \sum\limits_i \frac{q_i}{||\vec{r} - \vec{r_i}||} + C$ où $C$ est une constante.\\

Par conséquent, le travail apparaît comme la variation d'une fonction de la position.\\
Le travail $\tau$ ne dépend pas des différents chemins suivis par la particule entre les positions initiales et finales.\\

\begin{rema}
	On dit que le force est à "circulation" conservative.\\
	On dit aussi que cette force dérive d'une énergie potentielle.
\end{rema}

Il en résulte que $\oint_\mathcal{C} \vec{F} d\vec{r} = 0$
Pour tout chemin $\mathcal{C}$ emprunté.
%### fig 23

$$\varepsilon_{p_e} = \frac{q}{4\pi\varepsilon_0} \sum\limits_i \frac{q_i}{||\vec{r} - \vec{r_i}} + C$$

\section{qsd}
\end{document}
