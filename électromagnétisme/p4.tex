\documentclass[../main.tex]{subfile}

%page x

\begin{document}
\part{Propriétés de symétrie de $\vec{E}$}
\section{Invariance des lois de l'électrostatique au cours d'un déplacement}
	L'espace considéré sera homogène et isotrope.\\
	Soit $\Sigma$, un système physique (ici, on considère un ensemble de charges)\\
	$\Sigma$ crée $\vec{E(M)}$ en tout point $M$ de l'espace.\\
	L'espace étant homogène homogène et isotrope, si l'on fait subir une transformation
	géométrique $T$ à $\Sigma$ (rotation, translation, symétrie), les effets effets 
	subissent la même transformation.\\
	Si $\Sigma$ crée $\vec{E(M)}$ en $M$, \\
	alors $\Sigma' = T(\Sigma)$ créera $\vec{E'(M')}$ en $M'$ où $M' = T(M)$\\
	On écrit $\vec{E'(M')} = T(\vec{E(M)})$.\\
	En général, $\vec{E'(M')} \neq \vec{E(M)}$

\section{Influence de la symétrie de la source}
\subsection{Principe de Curie:}
\begin{theo}
	"Les effets ont au moins la symétrie des causes."
\end{theo}

\begin{itemize}
	\item
	Dans le cas général précédent, $T$ ne laissait pas le système invariant:\\
	$\Sigma' = T(\Sigma)$ ne coïncide pas avec $\Sigma$.
	
	\item En revanche, si le système $T$ laisse le système invariant, alors:\\
	$\Sigma' = \Sigma$ et les effets de $\Sigma$ et $\Sigma'$ se recouvrent:
	$$
\left\{
\begin{array}{c}
	\forall M, \ \vec{E'(M)} = \vec{E(M)}\\
	\vec{E'(M')} = \vec{E(M')}
\end{array}
\right.
	$$
\end{itemize}

\subsection{invariances des sources par translation}
	Il s'agit d'une distribution idéalisée:\\
	Dans la réalité , une translation ne peut pas être une opération de recouvrement.\\
\begin{ex}	
	Soit un fil rectiligne infini d'axe $Oz$ chargé avec une densité volumique de charge $\rho_c$\\
	On translate cette distribution de charges parallèlement à l'axe $Oz$ d'un vecteur $\vec{T}$ quelconque.\\
	Le fil étant infiniment long suivant $Oz$ et la translation étant parallèle à l'axe $Oz$, les deux distributions
	se recouvrent.\\
	Ainsi, $\rho'_c(I) = \rho_c(I)$\\
	$M \mapsto M'$ et $\vec{MM'} = \vec{T}$\\
	$\vec{E(M')} = \vec{E(M)}$\\
	Par conséquent, on a que pour tout $z_0$, 
	$\vec{E(x, y, z + z_0)} = \vec{E(x, y, z)}$
	$\vec{E}$ ne dépend donc pas de la variable $z$
\end{ex}

\subsection{invariance des sources par rotation autour d'un axe}
	Soit une distribution $\Sigma$ de charges uniformément réparties telle que toute section perpendiculaire à une direction est circulaire.\\
	Effectuons une rotation d'angle $\varphi$ autour d'un axe.\\
	%$\Sigma' = \RRR(\Sigma) \conj \Sigma$
	$\rho'(I) = \rho(I)$\\
	d'où $\vec{E'(M')} = \vec{E(M)}$\\
	Par conséquent, $\forall \varphi_0, \ \vec{E(r, \theta,\varphi)} = \vec{E(r, \theta, \varphi + \varphi_0)}$\\
	$\vec{E}$ est donc indépendant de la variable $\varphi$ par le principe de Curie.\\

\subsection{Lois physiques et symétries par rapport à un plan}
	Soit une opération de symétrie par rapport à un plan.\\
	Soit une charge $q$, placée au point $P$ et une charge $M$ où je regarde le champ.\\
	%###schéma
	Faisons un symétrie par rapport au plan $(xOy)$\\
	%$$P \to\limits_{q \conj q'}^{S/p} P'$$
	%$$M \to\limits^{S/p} M'$$
	$$\vec{E(M')} ) \frac{q'}{4\pi\varepsilon_0} \frac{\vec{P'M'}}{P'M'^3}$$
	$$\vec{P'M'}_{x \text{ ou } y} = \vec{PM}_{x \text{ ou } y}$$
	$$\vec{P'M'}_{z} = \vec{PM}_{z}$$
	Ainsi:
%	$$E'_x(M') = E_x(M)$$
	$$E'_y(M') = E_y(M)$$
	$$E'_z(M') = -E_z(M)$$
	%###schéma

\subsection{Distribution ayant un plan de symétrie de charges}
	Soit une distribution de charges ayant un plan de symétrie.\\
	%$$\Sigma \to\limits^{S/p} \Sigma' = \Sigma$$
	alors,
	$$\rho'(P) = \rho(P)$$


	Si %$M \conj M'$ (les deux appartenant au plan)\\
	alors $Em(M) = 0$, 
	le champ en un point $M$ d'un plan de symétrie de $\Sigma$ est contenu dans le plan.

\subsection{Distribution de charges dans un plan d'antisymétrie}
	Par un raisonnement identique au précédent, on montre que %$\vec{E(M \in Q)} perpendiculaire Q$
	$\vec{E(M \in P)} \subset P$

En un point d'un plan d'antisymétrie, le champ est perpendiculaire au plan.




\end{document}
