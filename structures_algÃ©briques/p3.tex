\documentclass[../main.tex]{subfile}

%suite cours 26/09 page x

\begin{document}

\begin{rap}
	Pour $\varphi : G \to \Gamma$, un isomorphisme 
	($\forall g, h \in G, \ \varphi(gh) = \varphi(g) \varphi(h)$)\\
	$\varphi$ est un isomorphisme si et seulement s'il est un morphisme bijectif.
\end{rap}

\begin{prop}
	Soit $\varphi : G \to Gamma$, un isomorphisme:\\
\begin{enumerate}	
	\item Si $G$ est abélien, alors $\Gamma$ est abélien.
	\item Si $g \in G$ est d'ordre $n$, alors $\varphi(g) \in \Gamma$ est d'ordre $n$.
\end{enumerate}
\end{prop}

\begin{proof}
\begin{enumerate}	
	\item Soient $\gamma_1 = \varphi(g_1), \gamma_2 = \varphi(g_2) \in \Gamma(g_1, g_2 \in G)$
	$$
\begin{aligned}
	\gamma_1\gamma_2 &= \varphi(g_1)\varphi(g_2)\\
	&= \varphi(g_1g_2)\\
	&= \varphi(g_2g_1)\\
	&= \varphi(g_2)\varphi(g_1)\\
	&= \gamma_2\gamma_1
\end{aligned}
	$$
	Donc $\Gamma$ est abélien.

	\item Soit $g \in G$ d'ordre $n$, c'est-à-dire que $n$ est le plus petit entier naturel
	non nul tel que $g^n = e^g$.\\
	On veut montrer que $\varphi(g)$ est d'ordre $n$. 
\begin{itemize}	
	\item 
	$$
\begin{aligned}
	\varphi(g)^n &= \varphi(g^n)\\
	&= \varphi(e_G)\\
	&= e_\Gamma
\end{aligned}
	$$
	\item Il reste à montrer que $\forall k \in \ZZ, \ 0<k<n, \ \varphi(g)^k \neq e_\Gamma$\\
	$$\varphi(g)^k = \varphi(g^k) = e_\Gamma \Leftrightarrow g^k \in ker(\varphi) = \{e_G\}$$
	Donc $\varphi(g)^k = e_\Gamma$ si et seulement si $g^k = e_G$\\
	Ainsi, $\varphi(g)$ et $g$ ont le même ordre.
\end{itemize}
\end{enumerate}
\end{proof}

\begin{ex}
\begin{enumerate}	
	\item $\ZZ/6\ZZ$ n'est pas isomorphe à $Isom(T)$\\
\begin{itemize}	
	\item $\ZZ/6\ZZ$ contient un élément d'ordre $G$
	et les ordres possibles pour les éléments de $Isom(T)$ sont $1$, $2$ et $3$.
	\item $\ZZ/6\ZZ$ est abélien mais $Isom(T)$ ne l'est pas.
\end{itemize}
	\item $\ZZ/4\ZZ$ (d'ordre 4) n'est pas isomorphe à $\ZZ/2\ZZ \times \ZZ/2\ZZ$ (dont l'ordre maximal est 2).\\
	Si $G$ est cyclique d'ordre $n$, alors il existe un $g \in G$ d'ordre $n$.\\
	Donc $\varphi(g)$ est d'ordre $n$. \\
	Ainsi, si $\varphi$ est bijective, $\Gamma$ est d'ordre $n$.
\end{enumerate}
\end{ex}

\begin{rema}
	De façon générale, si $\varphi : G \to \Gamma$, est un isomorphisme et $G$, un groupe cyclique, 
	alors $\Gamma$ est aussi un groupe cyclique.
\end{rema}


\end{document}
