\documentclass[../main.tex]{subfile}

%page 1

\begin{document}
\section{informations utiles}
Slavyana GENINSKA
Jean RAIMBAUT

cours sur:
http://www.math.univ-toulouse.fr/~jraimbau/Enseignement/theorie\_des\_groupes.html
\part{Théories des groupes}

\begin{ex}
	Isométries préservant un triangle équilateral

\begin{rap}
	Isométrie du plan:
	$$
\begin{aligned}
	f: \RR^2 &\to \RR^2\\
	\forall x, y \in \RR^2, &\ d(x, y) = d(f(x), f(y))
\end{aligned}
	$$
\begin{ex}{Isométries}
\begin{itemize}
	\item symétrie
	\item rotation
	\item translation
	\item symétrie glissée
\end{itemize}
\end{ex}
\end{rap}

\begin{rema}
	L'identité, notée $Id$, peut être vue comme une rotation (d'angle 0) ou comme une translation (par le vecteur nul). 
\end{rema}

Soit $T$, un triangle équilatéral.\\
$$Isom(T) \ = \ \{f:\RR^2 \to \RR^2, \ \text{isométrie} \ \| f(T) = T\}$$
est l'ensemble des isométries du plan sui préservent T.\\
%insérer la figure 1
Une telle application $f$ a forcement au moins un point fixe:
$$Isom(T) \ = \ \{Id, r_{\frac{2\pi}{3}}, r_{-\frac{2\pi}{3}}, S_A, S_B, S_C\}$$
On peut alors faire les deux remarques suivantes:
\begin{rema}
\begin{itemize}	
	\item $Isom(T)$ est stable par composition:\\
	$S_A \circ S_B = r_{\frac{2\pi}{3}}$\\
	$S_B \circ S_A = r_{-\frac{2\pi}{3}}$
	\item Toute application $f \in Isom(T)$ admet une transformation inverse $f^{-1} \in Isom(T)$
\end{itemize}
\end{rema}
\end{ex}

\begin{ex}
	Le groupe symétrique:\\
	Soit $E$, un ensemble de $n$ objets, $S_n$ est l'ensemble des bijection de $E$, appelé groupe symétrique.\\
	Par exemple, le groupe symétrique $S_3$ avec $E = \{1, 2, 3\}$
	%### compléter avec comp1

\begin{rema}	
\begin{itemize}	
	\item $S_3$ est stable par composition
	\item Toute bijection admet un inverse qui est encore dans $S_3$
\end{itemize}
\end{rema}

\begin{rema}
	Les deux exemples sont les mêmes d'un certain point de vue, il s'agit de la même structure algébrique (nous verrons plus tard qu'il s'agit d'un isomorphisme)
\end{rema}
\end{ex}

\begin{defi}
	Un groupe est un ensemble $G$ muni d'une application (appelée loi de groupe):
	$$
*: 
\begin{aligned}
	G \times G &\to G\\
	(g, h) &\mapsto g * h
\end{aligned}
	$$
	Cette loi vérifie les propriétés suivantes:
\begin{itemize}	
	\item associativité : 
	$$\forall g, h, k \in G, \ (g*h)*k = g*(h*k)$$
	\item présence d'un élément neutre:
	$$\exists e \in G \ / \ \forall g \in G, \ g * e = e * g = g$$
	\item existance de l'inverse (ou symétrique):
	$$\forall g \in G, \ \exists h \in G \ / \ g * h = h* g = e$$
\end{itemize}
\end{defi}

\begin{ex}
\begin{enumerate}	
	\item $\RR$ avec la loi $+$, l'élément neutre est alors $0$ et le symétrique est l'opposé.
	\item $\RR^*$ avec la loi $\cdot$, l'élément neutre est alors $1$ et le symétrique est l'inverse.
	\item Soit $P \subset \RR^2$, un polygone régulier à $n$ cotés.\\
	On note alors $Isom(P)$, l'ensemble des isométries le concervant:
	$$Isom(P) = \{f:\RR^2 \to \RR^2, \ \text{isométrie} \ \| \ f(P) = P\}$$
	$Isom(P)$ est alors un groupe si on le muni de la loi de composition $\circ$.\\
	L'élément neutre est alors l'identité : $\forall f \in Isom(P), \ f \circ Id = Id \circ f = f$.\\
	Le symétrique est la transformation réciproque $f^{-1}$
	\\
	Ce groupe est alors appelé groupe diédral, on le note $D_n$ (ou $D_{2n}$ étant donné que ce groupe possède $2n$ éléments).
	\begin{ex}	
	\begin{itemize}	
		\item $D_3 = Isom(T)$ est le groupe présenté dans l'exemple 1, \\
		$D_3$ possède six éléments
		\item $D_4$ est l'ensemble des isométries préservant le carré.\\
		%### insérer fig2
		$D_4 = Isom(C) = \{Id, r_{\frac{\pi}{2}}, r_{\pi}, r_{-\frac{\pi}{2}}, S_{AC}, S_{MP}, S_{BD}, S_{NQ}\}$ \\
		$D_4$ possède donc 8 éléments
	\end{itemize}
	\end{ex}
	\item Si $E$ est un ensemble, l'ensemble des bijections de $E$ dans $E$ est un groupe pour la loi $\cdot$ comme précédemment.\\
	Si $E = \{1, ..., n\}, \ Bij(E)S_n$\\
	Si $E = \RR, \ Bij(\RR)$ est un groupe
	\item $\RR^n$ muni de l'addition vectorielle est un groupe.
	Plus généralement, tout espace vectoriel $E$ est un groupe pour l'addition
	\item $GL_n(\RR) = \{A\in M_{n,n}(\RR) \ \| \ det A \neq 0\}$
	Pour la multiplication matricielle, voir l'exercice 1.
\end{enumerate}

\begin{ctex}
\begin{enumerate}	
	\item $(\NN, +)$ n'est pas un groupe car aucun élément n'admet de symétrique
	\item $(\RR, \cdot)$ n'est pas un groupe car $0$ n'admet pas de symétrique
	\item $(\ZZ^*, \cdot)$ n'est pas un groupe car $1$ et $-1$ sont les seuls éléments admettant un symétrique
	\item $(\{-1, 0, 1\}, +)$ n'est pas un groupe car $1 + 1 = 2 \notin \{-1, 0, 1\}$
\end{enumerate}
\end{ctex}

\begin{rema}
	Le groupe $\ZZ$ est $(\ZZ, +)$.\\
	Le groupe $\RR^*$ est $(\RR^*, \cdot)$.\\
	Le groupe $\RR^n$ est $(\RR^n, +)$.\\
\end{rema}

\begin{defi}
	On dit qu'un groupe $G$ est commutatif (ou abélien) si:
	$$\forall g, h \in G, \ \Rightarrow g*h = h*g$$
\end{defi}

\begin{ex}
	$(\ZZ, +)$, $(\RR^*, \cdot)$, $(\CC^*, \cdot)$, $(\RR^n, +)$ sont des groupes abéliens.
\end{ex}

\begin{ctex}
	$S_n$ pour $n \geq 3$, $GL_n(\RR)$ pour $n \geq 2$ ne sont pas des groupes abléliens
\end{ctex}

\begin{ex}
	Soit $n > 0$, un entier fixé.\\
	$\ZZ / n\ZZ$, l'ensemble des entiers $a \in \ZZ$ considéré modulo n:\\
	$$\bar{a} = \{a + kn \ | \ k \in \RR^n\} \in \ZZ / n\ZZ$$
	Pour $a, b \in \ZZ, \ \bar{a} = \bar{b}$ si et seulement si, pour $k \in \ZZ$, $a-b = kn$
\begin{ex}	
	Dans $\ZZ / 3\ZZ$, \\
	$\bar{1} = \bar{4} = \bar{10} = \bar{-2}$ mais $\bar{1} \neq \bar{2}$\\
	$$\ZZ / 3\ZZ = \{\bar{0}, \bar{1}, \bar{2}\}$$
	$$\bar{0} = \{3k \ | \ k \in \ZZ\}$$
	$$\bar{1} = \{1 + 3k \ | \ k \in \ZZ\}$$
	$$\bar{2} = \{2 + 3k \ | \ k \in \ZZ\}$$
	$$\bar{0} \cup \bar{1} \cup \bar{2} = \ZZ$$
\end{ex}
	On définit l'addition sur $\ZZ / n\ZZ$ telle que : $\bar{a} + \bar{b} = \bar{a+b}$.\\
	On vérifie que cette définition ne dépend pas du choix des représentants.\\
	$$
\begin{aligned}
	\bar{a} + \bar{b} &= \bar{a+k_1n} + \bar{b+k_2n}\\
	&= \bar{a+k_1n + b+k_2n}\\
	&= \bar{a+b+(k_1+k_2)n}\\
	&= \bar{a+b}
\end{aligned}
	$$
\begin{rema}	
	Sur $\ZZ/2\ZZ = \{\bar{0}, \bar{1}\}$\\
	$\bar{0} = \{2k \ |\ k \in \ZZ\}$, l'ensemble des nombres pairs\\
	$\bar{1} = \{1 + 2k \ |\ k \in \ZZ\}$, l'ensemble des nombres impairs\\
	%compléter avec les liens somme / parité (quatre lignes)$$
\end{rema}
	$(\ZZ / n\ZZ, +)$ est un groupe abélien.\\
\begin{rema}	
	Comment définir une multiplication sur $\ZZ / n\ZZ$?
\end{rema}
\end{ex}

\begin{nota}
	Un groupe est noté $G$
\end{nota}

\begin{nota}{notation multiplicative}
	Il s'agit de la notation par défaut, \\
	"produit" du $g$ et $h$ : $gh$\\
	élément neutre : $e$, $1$ ou $1_G$\\
	l'inverse de $g$ : $g^{-1}$ (et jamais $1 \over g$)
%###compléter l'exemple
\end{nota}

\begin{nota}{notation additive}
	Il s'agit de la notation préférée pour les groupes abéliens, \\
	"somme" de $a$ et $b$ : $a+b$\\
	élément neutre : $0$ ou $0_G$\\
	l'inverse de $a$ : $-a$
\end{nota}

\section{Les sous-groupes}
\begin{defi}
	Soit $G$, un groupe.\\
	Un sous-ensemble $H \subset G$ est appelé sous-groupe de $G$ et noté $H < G$ si la loi sur $G$ induit une structure de groupe sur $H$, c'est-à-dire:\\
\begin{itemize}	
	\item $\forall h_1, h_2 \in H, \ h_1h_2 \in H$ (la loi est interne)
	\item l'élément neutre $e$ de $G$ est dans $H$
	\item $\forall h \in H$, $h$ admet un symétrique $h^{-1} \in H$ (on dit que $H$ est stable par passage au symétrique)
\end{itemize}
\end{defi}

\begin{ex}
\begin{itemize}	
	\item $n \ZZ = \{nk \ |\ k \in \ZZ\} < \ZZ$
	\item $\RR_{> 0} = \{x \in \RR \ |\ x > 0\} < \RR^*$
	\item Le cercle unité $U = \{x \ | \ |z| = 1\} < \CC^*$\\
	Le groupe des racines n-ièmes de l'unité $U_n = \{z \in \CC \ | \ z^n = 1\} < \CC^*$
\begin{rema}	
	$\forall n, \ U_n \subset U$ mais $\forall n, \ U_n \neq U$
\end{rema}
	\item Soit $P$ un polygone.\\
	$Isom(P)$, le groupe d'isométries préservant $P$ (rotations et symétries).\\
	$Isom^+(P)$, les isométries de $Isom(P)$ qui préservent l'orientation du plan (ici, seulement les rotation).\\
	On a $Isom^+(P) < Isom(P)$
	\item $Diff(\RR) < Bij(\RR)$, le sous-groupe des bijections de $\RR$ de classe $\CCC^\infty$
	\item $SL_n(\RR) = \{M \in GL_n(\RR) \ |\ det(M) = 1\} < GL_n(\RR)$
\end{itemize}
\end{ex}

\begin{prop}
	Soit $G$, un groupe.\\
	Un sous-ensemble $H$ de $G$ est un sous-groupe si et seulement si les deux consitions suivantes sont satisfaites:
\begin{itemize}	
	\item $H \neq \emptyset$
	\item $\forall h_1, h_2 \in H, h_1h_2^{-1} \in H$
\end{itemize}
\end{prop}

\begin{proof}
	On suppose que $H$ satisfait les deux points de la propriété ci-dessus.\\
	$H \neq \emptyset$ donc $\exists h \in H$\\
	pour vérifier que $e \in H$, \\
	on applique la seconde propriété à h, donc $hh^{-1} = e \in H$
	on vérifie ensuite que tout élément de $H$ possède un inverse dans $H$, \\
	soit $h \in H$, on applique la seconde propriété à $e$ donc $eh^{-1} = h^{-1} \in H$\\
	on vérifie enfin que le produit de tout élément de $H$ appartient à $H$\\
	Soient $h_1, h_2 \in H$. On applique la seconde propriété à $h_1, h_2^{-1} \in H$. Donc $h_1(h_2^{-1})^{-1} = h_1h_2 \in H$
\end{proof}
\end{ex}

\section{Sous-groupe engendré}

\begin{prop}
	Soit $G$, un groupe, soit de plus $S \subset G$.\\
	$$\exists! H < G \ / \ S \subset H \ \text{ et } \ \forall F < G \ / \ S \subset F, \ H \subset F$$
\end{prop}



















































\end{document}
