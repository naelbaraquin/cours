\documentclass[../main.tex]{subfile}

%page 5 action de groupe

\begin{document}
\section{Groupe opérant sur un ensemble (action de groupe)}

\begin{defi}
	Soient $G$, un groupe et $X$, un ensemble.\\
	On appelle \emph{action (ou opération) à gauche de $G$ sur $X$} toute application:
	$$
\begin{aligned}
	\cdot : G \times X &\to X\\
	(g, x) &\mapsto g \cdot x
\end{aligned}
	$$
	qui satisfait les deux conditions suivantes:\\
\begin{itemize}	
	\item $\forall g, h \in G, \ \forall x \in X, \ g \cdot (h \cdot x) = (g \cdot h) \cdot x$
	\item $\forall x \in X, \ e \cdot x = x$
\end{itemize}
\end{defi}

\begin{defi}{Opération à droite}
	$$
\begin{aligned}
	\cdot : G \times X &\to X\\
	(g, x) &\mapsto g \cdot x
\end{aligned}
	$$
	avec:\\
\begin{itemize}	
	\item $\forall g, h \in G, \ \forall x \in X, \ g \cdot (h \cdot x) = (g \cdot h) \cdot x$
	\item $\forall x \in X, \ e \cdot x = x$
\end{itemize}
\end{defi}

\begin{ex}
\phantom{a}
\begin{itemize}	
	\item $\ZZ/3\ZZ$ agit par rotation sur $\RR^2 = \CC$.\\
	$\forall z = re^{i\phi} \in \CC, \ \forall \bar{k} \in \ZZ/3\ZZ, \ \bar{k} \cdot z = re^{i\phi + \frac{2\pi k}{3}}$\\

\begin{center}	
	\def\svgwidth{0.3\textwidth}
	\input{fig51stralg.pdf_tex}
\end{center}

	\item $\RR$ agit par translation sur $\RR^2$\\
	$\forall r \in \RR, \ \forall (x, y) \in \RR^2, \ r \cdot (x, y) = (x+r, y)$

	\item $\RR^2$ agit par translation sur $\RR^2$\\
	$\forall (\alpha, \beta) \in \RR^2, \ \forall (x, y) \in \RR^2, \ (\alpha, \beta) \cdot (x, y) = (\alpha + x, \beta + y)$

	\item Si $G$ et un groupe et $X = G$, alors on a l'opération de $G$ sur lui-même:\\
	\begin{enumerate}	
		\item par translation à gauche:
		$$
	\begin{aligned}
		\cdot : G \times G &\to G\\
		(g, x) &\mapsto g \cdot x
	\end{aligned}
		$$

		\item par conjugaison:
		$$
	\begin{aligned}
		\cdot : G \times G &\to G\\
		(g, x) &\mapsto g \cdot x \cdot g^{-1}
	\end{aligned}
		$$
	\end{enumerate}

	\item L'action triviale de $G$ sur $X$ est donnée par :\\
	$g \cdot x = x, \ \forall g \in G, \forall x \in X$
\end{itemize}
\end{ex}

\begin{prop}
	Soit $G$, un groupe et $X$, un ensemble.\\
	Il y a une correspondance bijective et naturelle entre les actions (à gauche) de $G$ sur $X$ et les morphisme de $G$ vers $Bij (X)$.
\end{prop}

\begin{proof}
\begin{itemize}
	\item Soit $\diamond$, une action de $G$ sur $X$.\\
	$\forall g \in G$, on considère l'application 
		$$
	\begin{aligned}
		\sigma_g : X &\to X\\
		x &\mapsto g \diamond x
	\end{aligned}
		$$
	\'Etant donné que :
	$$
\begin{aligned}
	g^{-1} \diamond (g \diamond x) &= (g^{-1}g) \diamond x \\
	&= x\\
	&= (gg^{-1}) \diamond x\\
	&= g \diamond (g^{-1} \diamond x)
\end{aligned}
	$$
	On a :
	$$\sigma_{g^{-1}} \circ \sigma_g = Id_X = \sigma_g \circ \sigma_{g^{-1}}$$

	Donc $\sigma_g \in Bij(X)$\\
	De plus, étant donné que $g_1 \diamond (g_2 \diamond x) = (g_1g_2) \diamond x$, \\
	on a $\sigma_{g_1} \circ \sigma_{g_2} = \sigma_{g_1g_2}$\\

	Donc l'application :
		$$
	\begin{aligned}
		\sigma_\diamond : G &\to Bij(X)\\
		g &\mapsto \sigma_g
	\end{aligned}
		$$
	est un morphisme.


	\item Soit $\Psi : G \to Bij(X)$, un morphisme.\\
	On obtient l'action $\diamond_\Psi$ de $G$ sur $X$ pour $g \diamond_\Psi x = \Psi(g)(x)$
	On vérifie que c'est une action de groupe.\\
	La correspondance bijective vient de $\Psi_{\diamond_\Psi} = \Psi$ et $\diamond_{\Psi_\diamond} = \diamond$
\end{itemize}
\end{proof}

\begin{defi}
	Soit $G$ opérant sur $X$ et $x \in X$.\\
	L'orbite de $x$ sous $G$ est :
	$$Orb(x) = \{g \cdot x \ | \ g \in G\} \subset X$$
\end{defi}

\begin{prop}
	Soit $G$ opérant sur $X$.\\
	Soit $\sim$, la relation sur $X$ définie par 
	$$x \sim y \Leftrightarrow x \in Orb(y)$$
	Alors $\sim$ est une relation d'équivalence sur $X$ dont les classes d'équivalences sont les orbites.
\end{prop}

\begin{corrol}
	Les orbites des éléments de $X$ sous l'action de $G$ forment une partition de $X$.\\
	On note $G \backslash X$ %(###  à reprendre notation d'un quotient à l'envers)
	l'ensemble quotient $X / \sim$
\end{corrol}

\begin{proof}
	On montre que $\sim$ est une relation d'équivalence:\\
\begin{itemize}	
	\item Réflexivité: $\forall x \in X, \ e \cdot x = x$, par conséquent, $x \in Orb(x)$ et $x \sim x$.
	\item Symétrie: Supposons $x \sim y$, \textit{i.e.} $\exists g \in G \ / \ x = g \cdot y \Leftrightarrow x \in Orb(y)$\\
	Donc $g^{-1} \cdot x = g^{-1} \cdot (g \cdot y)^{-1} = (g^{-1}g) \cdot y = e \cdot y = y$\\
	par conséquent, $y = Orb(x)$\\
	Ainsi $y \sim x$
	\item transitivité: Supposons $x \sim y$ et $y \sim z$, \textit{i.e.} $\exists g, h \in G \ / \ x = g \cdot y \ \text{et} \ y = h \cdot z$\\
	Donc $x = g \cdot y = g \cdot (h \cdot z) = (gh) \cdot z$\\
	par conséquent, $x \in Orb(z)$,\\
	ainsi, $x \sim z$
\end{itemize}
	La classé d'équivalence d'un $z \in X$ est $Orb(z)$
\end{proof}

\begin{ex}
\begin{itemize}	
	\item Les orbites pour $\ZZ/3\ZZ$ opérant sur $\CC$ par rotation :
	$$Orb(z) = \{z, e^{\frac{2\pi}{3}}z, e^{\frac{4\pi}{3}}z\}$$
	Comporte $3$ éléments si $z \neq 0$\\
	Sinon (pour $z = 0$), $Orb(0) = \{0\}$

	\item Pour $\RR$ opérant sur $\RR^2$,\\
	$$r \cdot (x,y) \to (x + r , y)$$
	$$Orb((x,y)) = \{r \cdot (x,y) \ | \ r \in \RR\} = \{(x + r, y) \ | \ r \in \RR\} = \{(\lambda, y) \ | \ \lambda \in \RR\}$$
	Ce qui correspond aux droites horizontales passant par $(0, y)$
\begin{center}	
	\def\svgwidth{0.3\textwidth}
	\input{fig52stralg.pdf_tex}
\end{center}

	\item $\RR^2$ opérant par translation sur $\RR^2$, \textit{i.e.} $\exists (\alpha, \beta), (x, y) \in \RR^2$
	$$Orb(x,y) = \{(\alpha + x , \beta + y) \ | \ (\alpha, \beta) \in \RR^2\} = \{(s,t) \ | \ (s,t) \in \RR^2\} = \RR^2$$
	On a donc qu'une seule orbite

	\item $G$ opérant par translation sur $G$ tel que $g \cdot x$
	$$Orb(e) = \{g \cdot e \ | \ g \in G\} = \{ g | g \in G\} = G$$
	Donc, $\forall h \in G, \ Orb(h) = Orb(e) = G$.\\
	Il n'y a donc qu'une seule orbite.
\end{itemize}
\end{ex}

\begin{defi}
	L'action $G$ opérant sur $X$ est transitive s'il existe exactement une orbite dans $X$.
\end{defi}

\begin{defi}	
	Soit $G$ opérant sur $X$, une action de groupe.\\
\begin{itemize}	
	\item \emph{Le stabilisateur de $x \in X$ dans $G$} est le sous groupe
	$$Stab(x) = \{g \in G \ | \ g \cdot x = x\} \subset G$$

	\item \emph{Les points fixés par $g \in G$} sont définis par :
	$$Fig(g) = \{x \in X \ | \ g \cdot x = x\}$$

	\item $x \in X$ est un \emph{point fixe} sous l'action de $G$ si $\forall g \in G, \ g \cdot x = x (\Leftrightarrow Stab(x) = G \Leftrightarrow Orb(x) = \{x\})$
\end{itemize}
\end{defi}

\begin{prop}
	Soit $G$ opérant sur $X$, une action de groupe.\\
	Si $x$ et $y$ sont dans la même orbite, \textit{i.e.} $Orb(x) = Orb(y)$, 
	alors $Stab(x)$ et $Stab(y)$ sont conjugués.
\end{prop}

\begin{proof}
	Supposons $y=g \cdot x$.
	Alors 
	$$g^{-1} \cdot y = g^{-1} \cdot (g \cdot x) = (g^{-1}g) \cdot x = e \cdot x = x$$
	On va montrer que $g Stab(x) g^{-1} = Stab(y)$.\\
\begin{itemize}	
	\item Soit $h \in Stab(x)$, \textit{i.e.} $ghg^{-1} \in g Stab(x) g^{-1}$\\
	On souhaite montrer que $ghg^{-1} \in Stab(y)$
	$$(ghg^{-1}) \cdot y = g \cdot (h \cdot (g^{-1} \cdot y)) = g \cdot (h \cdot x) = g \cdot x = y$$
	Donc $ghg^{-1} \in Stab(y)$ et donc $g Stab(y) g^{-1} \subseteq Stab(y)$

	\item L'argument précédent, en échangeant $x$ par $y$ et $g$ par $g^{-1}$, nous donne:\\
	$$g^{-1} Stab(y) g \subseteq Stab(x)$$
	$$Stab(y) \subseteq g Stab(x) g^{-1}$$
\end{itemize}
\end{proof}

\begin{theo}
	Soit $G$ opérant sur $X$ une action de groupe.\\
	Soit, de plus, $x \in X$.\\
	Il y a une bijection entre $Orb(x)$ et $G/Stab(x)$ (les classes d'équivalence à gauche de $Stab(x)$).
	En particulier, si $Orb(x)$ est fini, alors $Stab(x)$ est d'indice fini dans $G$ et $|Orb(x)| = [G \cdot Stab(x)]$
\end{theo}

\begin{proof}
	On veut définir la fonction:
	$$
\begin{aligned}
	f : \phantom{...} Orb(x) &\to G/Stab(x) = \{g Stab(x) \ | \ g \in G\}
	g \cdot x &\mapsto g Stab(x)
\end{aligned}
	$$
\begin{itemize}	
	\item
	$f$ est en effet une fonction.\\
	Soient $g,h \in G$\\
	$$
\begin{aligned}
	g \cdot x = h \cdot x &\Leftrightarrow h^{-1} \cdot (g \cdot x) = h^{-1} \cdot (h \cdot x)\\
	&\Leftrightarrow (h^{-1} \cdot g) \cdot x = (h^{-1} \cdot h) \cdot x\\
	&\Leftrightarrow (h^{-1} \cdot g) \cdot x = x\\
	&\Leftrightarrow h^{-1} \cdot g \in Stab(x)\\
	&\Leftrightarrow \exists s \in Stab(x), \ h^{-1}g = s \ \text{autrement dit} \ g = hs\\
	&\Leftrightarrow g Stab(x) = h Stab(x)
\end{aligned}
	$$

	\item $f$ est injective car $g Stab(x) = h Stab(x) \Rightarrow gx = hx$

	\item $f$ est surjective car $\forall g Stab(x) \in G / Stab(x), \ f(g \cdot x) = g Stab(x)$
\end{itemize}
\end{proof}

\begin{corrol}{Formule de classes}
	Si $G$ est un groupe fini, alors 
	$$\forall x \in X, \ |G| = |Stab(x)| \cdot |Orb(x)|$$
\end{corrol}

\begin{proof}
	Si $G$ est fini, alors $Orb(x)$ est fini et 
	$$|orb(x)| = [G : Stab(x)] = \frac{|G|}{|Stab(x)|}$$
\end{proof}

\end{document}
