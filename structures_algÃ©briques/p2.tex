\documentclass[../main.tex]{subfile}

%page 2

\begin{document}
\begin{rema}
	Un groupe monogène est nécessairement commutatif.
\end{rema}

\begin{proof}
	Si $G = <g>$, alors $G = \{g^n, \ n \in \ZZ\}$\\
	De plus, si $k, l \in \ZZ$, on a $g^kg^l = g^{k+l} = g^{l+k} = g^lg^k$
\end{proof}

\begin{rema}
	En particulier, un groupe non commutatif ne peut pas être monogène (contraposée de la remarque précédente)
\begin{ex}	
	$S_3$, par exemple, n'étant pas commutatif, n'est pas non plus monogène.
\end{ex}
\end{rema}

\subsection{Produit de groupes}

Soient $G_1$ et $G_2$, deux groupes.\\
Le groupe produit $G = G_1 \times G_2$ est défini par l'ensemble $\{(g_1, g_2) \ | \ g_1 \in G_1,\ g_2 \in G_2\}$\\
avec l'opération $* : G \times G \to G$ définie par 
$$(g_1, g_2) \times (h_1, h_2) = (g_1h_1, g_2h_2)$$
On définit de manière similaire le produit d'une famille de groupes.\\

\begin{rema}
	$$G_1 \times (G_2 \times G_3) = (G_1 \times G_2) \times G_3 = G_1 \times G_2 \times G_3$$
\end{rema}

\begin{ex}
\begin{itemize}	
	\item $$\ZZ^2 = \ZZ \times \ZZ = \{(m, n) \ | \ m, n \in \ZZ\}$$
	$$\RR^2 = \RR \times \RR$$

	\item
	$$
\begin{aligned}	
	\frac{\ZZ}{2\ZZ} \times \frac{\ZZ}{2\ZZ} &= \{(a, b) \ | \ a \in \frac{\ZZ}{2\ZZ} , \ b \in \frac{\ZZ}{3\ZZ}\}\\
	&= \{(\bar{0}, \bar{0}), (\bar{0}, \bar{1}), (\bar{0}, \bar{2}), (\bar{1}, \bar{0}), (\bar{1}, \bar{1}), (\bar{1}, \bar{2})\}\\
	&= <(\bar{1}, \bar{1})> \ \text{qui est un groupe cyclique d'ordre 6}
\end{aligned}
	$$

	\item $\frac{\ZZ}{2\ZZ} \times \frac{\ZZ}{2\ZZ}$ n'est pas cyclique:\\
	$\{(\bar{0}, \bar{0}), (\bar{0}, \bar{1}), (\bar{1}, \bar{0}), (\bar{1}, \bar{1})\}$ contient un élément d'ordre $1$ et trois éléments d'ordre $2$.\\
	$\frac{\ZZ}{2\ZZ} \times \frac{\ZZ}{2\ZZ}$ ne contient pas d'élément d'ordre $4$ et n'est donc pas cyclique.
\end{itemize}
\end{ex}

\subsection{Morphismes}

\begin{defi}
	Soient $(G, \star)$ et $(\Gamma, \diamond)$, deux groupes.\\
	On appelle morphisme (ou homomorphisme) de groupes de $G$ vers $\Gamma$ toute application 
	$$\varphi : G \to \Gamma$$
	telle que $\forall g, h \in G, \ \varphi(g \star h) = \varphi(g) \diamond \varphi(h)$
\end{defi}

\begin{rema}
	S'il n'y a pas d'ambiguïté, on utilisera la notation multiplicative pour $G$ et $\Gamma$ :
	$$\varphi(gh) = \varphi(g) \varphi(h)$$
\end{rema}

\begin{propri}
\begin{itemize}	
	\item $\varphi(e_G) = e_\Gamma$
\begin{proof}	
	$$\varphi(e_G) = \varphi(e_Ge_G) = \varphi(e_G) \varphi(e_G)$$
	On multiplie par $\varphi(e_G)^{-1}$ l'égalité précédente et on obtient 
	$$\varphi(e_G)\varphi(e_G)^{-1} = \varphi(e_G) \varphi(e_G) \varphi(e_G)^{-1}$$
	$$\Rightarrow e_\Gamma = \varphi(e_G) e_\Gamma$$
	Donc, $e_\Gamma = \varphi(e_G)$
\end{proof}

	\item $\forall g \in G, \ \varphi(g^{-1}) = \varphi(g)^{-1}$
	%###démo

	\item $\forall n \in \ZZ, \ \forall g \in G, \ \varphi(g^n) = \varphi(g)^n$
\begin{proof}	
	Par récurrence pour $n > 0$, \\
	On utilise $\varphi (g^1) = \varphi (g)$\\
	Pour $n \geq 2$, \\
	$$
\begin{aligned}
	\varphi(g^n) &= \varphi(g^{n-1} g)\\
	&= \varphi(g^{n-1}) \varphi(g)\\
	&= \varphi(g)^{n-1} \varphi(g)\\
	&= \varphi(g)^n\\
\end{aligned}
	$$
	Pour $n = 0$, \\
	$g^0 = e_G$ et $\varphi(g)^0 = e_\Gamma$\\
	Pour $n = 1$, \\
%###finir démo
\end{proof}
\end{itemize}
\end{propri}

%### reprendre les exemples

\begin{defi}
	Soit $\varphi : G \to \Gamma$, un morphisme de groupes.\\
	On appelle : \\
\begin{itemize}	
	\item Image de $\varphi$, l'ensemble
	$$Im (\varphi) = \{\varphi(g) \ | \ g \in G\} = \{\gamma \in \Gamma \ | \ \exists g \in G \ / \ \varphi(g) = \Gamma\} \subset \gamma$$

	\item noyau de $\varphi$, 
%### finir les définitions
\end{itemize}
\end{defi}

\begin{rema}
\begin{itemize}	
	\item Les exemples 1, 3 et 4 sont des isomorphismes
	\item Le déterminant n'est pas bijectif pour $n > 1$
\end{itemize}
\end{rema}

\begin{defi}
	Soit $G$ un groupe.\\
	Un sous-groupe $H$ de $G$ est distingué (ou normal) dans $G$ si 
	$$\forall g \in G, \ \forall h \in H, \ ghg^{-1} \in H$$
\begin{nota}	
	On note alors $H \triangleleft G$
\end{nota}
	Un élément de type $ghg^{-1}$ est dit conjugué de $h$ par $g$.
\end{defi}

\begin{prop}
	Soit $\varphi : G \to \Gamma$, un morphisme.\\
\begin{itemize}	
	\item $Im(\varphi)$ est un sous-groupe de $\Gamma$
	\item $Ker(\varphi)$ est un sous-groupe distingué de $G$
\end{itemize}
\end{prop}

\begin{proof}
\begin{itemize}	
	\item 
\begin{itemize}	
	\item $\varphi(e_G)= e_\Gamma$ donc $e_\Gamma \in Im(\varphi)$
	%### finir la preuve
\end{itemize}
\end{itemize}
\end{proof}


\begin{prop}
	Soit $\varphi : G \to \Gamma$, un morphisme.\\
	Alors $\varphi$ est injectif si et seulement si $Ker(\varphi) = \{e_G\}$
\end{prop}

\begin{proof}
	Dans le sens direct:\\
	On suppose que $\varphi$, est injectif, i.e.
	$$\forall g_1, g_2 \in G, \ \varphi(g_1) = \varphi(g_2) \Rightarrow g_1 = g_2$$
	Soit $g \in Ker(\varphi)$.
	On sait que $e_G \in Ker(\varphi)$\\
	On a alors, $\varphi(g) = e_\Gamma = \varphi(e_G)$.\\
	Par l'injectivité de $\varphi$, on a $g = e_G$.\\
	Donc $Ker(\varphi) = \{e_G\}$

	Dans le sens indirect:\\
	On suppose que $Ker(\varphi) = \{e_G\}$\\
	On veut montrer que $\varphi$ est injectif.\\
	Soient $g_1, g_2 \in G$ tels que $\varphi(g_1) = \varphi(g_2)$\\
	$$
\begin{aligned}
	\varphi(g_1g_2^{-1}) &= \varphi(g_1)\varphi(g_2^{-1})\\
	&= \varphi(g_2) \varphi(g_2)^{-1}\\
	&= e_\Gamma
\end{aligned}
	$$
	Donc $g_1g_2^{-1} \in Ker(\varphi) = \{e_G\}$\\
	Donc $g_1g_2^{-1} = e_G$\\
	Donc $g_1 = g_2$\\
	Donc $\varphi$ est injectif.
\end{proof}

\begin{prop}
	Si $\varphi : G \to \Gamma$ est un morphisme bijectif, alors l'application $\varphi^{-1} : \Gamma \to G$ est un morphisme (lui aussi bijectif).\\
	Autrement dit, si $\varphi : G \to \Gamma$ est un isomorphisme, alors $\varphi^{-1} : \Gamma \to G$ est aussi un isomorphisme.
\end{prop}

\begin{proof}
	Soient $\gamma_1, \gamma_2 \in \Gamma$\\
	Il existe $g_1, g_2 \in G$ tels que $\gamma_1 = \varphi(g_1)$ et $\gamma_2 = \varphi(g_2)$
	$$
\begin{aligned}
	\varphi^{-1}(\gamma_1\gamma_2) &= \varphi^{-1}(\varphi(g_1)\varphi(g_2))\\
	&= \varphi^{-1}(\varphi(g_1g_2))\\
	&= g_1g_2\\
	&= \varphi^{-1}(\gamma_1)\varphi^{-1}(\gamma_2)
\end{aligned}
	$$
	Donc $\varphi^{-1}$ est un morphisme.
\end{proof}

\begin{defi}
	Deux groupes $G$ et $F$ sont isomorphes s'il existe un isomorphisme $\varphi : G \to \Gamma$.
\begin{nota}	
	On note alors $G \cong \Gamma$
\end{nota}
\end{defi}

\begin{rema}
	Deux groupes sont isomorphes quand ils possèdent la même structure de groupe.
\end{rema}

\begin{ex}
\begin{itemize}	
	\item $exp : (\RR, +) \to (\RR_{>0}, \cdot)$ est un isomorphisme.\\
	Donc $(\RR, +) \cong (\RR_{>0}, \cdot)$
	\item $
\begin{aligned}
	f: Isom(T) &\to S_3\\
	g &\mapsto f(g) \ / \ g(x_i) = x_{f(g)(i)}
\end{aligned}$ est un isomorphisme\\
	Donc $Isom(T) \cong S_3$

	\item Les trois groupes suivants sont deux à deux isomorphes:\\
	$\ZZ / n\ZZ$, les racines n-ièmes de l'unité $U_n$ et $Isom(n-\text{gônes réguliers})$

	\item $\ZZ / 6\ZZ \cong \ZZ/2\ZZ \times \ZZ/3\ZZ$

	\item 
	$$
\begin{aligned}
	f_1: \ZZ/n\ZZ &\to U_n\\
	\bar{k} &\mapsto e^{i\frac{2k\pi}{n}}
\end{aligned}
	$$
	est un isomorphisme

	\item
	$$
\begin{aligned}
	f_2: U_n &\to Isom(n-\text{gônes réguliers})\\
	e^{i\frac{2k\pi}{n}} &\mapsto r_{\frac{2k\pi}{n}}
\end{aligned}
	$$
\end{itemize}
\end{ex}

\begin{prop}
	Soit $G$ un groupe.\\
\begin{enumerate}	
	\item Si $\varphi : (\ZZ, +) \to G$ est un morphisme, alors il existe un unique élément de $G$ tel que $\forall n \in \ZZ, \ \varphi(n) = g^n$
	l'élément neutre est donné par $\varphi(1) = g$.
\begin{proof}	
	$g=\varphi(1)$ est unique.\\
	Puis on se sert de la troisième propriété (qui nous donne $\varphi(n) = \varphi(1)^n = g^n$)
\end{proof}

	\item Si $g$ est un élément quelconque de $G$, alors il existe un unique morphisme $\varphi_g : (\ZZ, +) \to G$, tel que $\varphi(1) = g^n$
\begin{proof}	
	$\varphi$ est un morphisme de groupes car
	$$\varphi(m+n) = g^{m+n} = g^mg^n = \varphi(m)\varphi(n)$$
\end{proof}
\end{enumerate}
\end{prop}









\end{document}













