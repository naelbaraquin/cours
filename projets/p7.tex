\documentclass[../main.tex]{subfile}

%page déterminants

\begin{document}
\part{Déterminants}
\section{Volume orienté du parallélogramme dans le plan}

%###figure du parallélogramme d'aire \ssim(u,v)
%%%%%%%%$$\ssim(u,v) = 
%%%%%%%%\left\{
%%%%%%%%\begin{aligned}	
%%%%%%%%%	A(u,v) \text{si} \theta .................%###
%%%%%%%%\end{aligned}
%%%%%%%%\right.
%	$$

\begin{prop}
%	$\ssim$ est bilinéaire et antisymétrique si :
\begin{itemize}	
	\item 
\end{itemize}
\end{prop}

%### il y a une démo entre temps

\section{Définitions et propriétés}
\subsection{Définition par récurrence}
\begin{defi}
	définition du déterminent d'une matrice par récurrence (la définition est facile mais la preuve est pénible)
\end{defi}

\subsection{définition par forme multilinéaire alternée}
\begin{rap}
	Une permutation de $\{1, ..., n\}$ est une bijection de $\{1, ..., n\}$ dans lui-même.
\end{rap}

\begin{ex}
	$$
\begin{aligned}
	1 &\to 2\\
	2 &\to 1\\
	3 &\to 3\\
\end{aligned}
	$$
	est une permutation.
\end{ex}

\begin{nota}
	$S_n$ est l'ensemble des permutations de $\{1, ..., n\}$
\end{nota}

\begin{rema}
	$S_n$ contient $n!$ éléments
\end{rema}

\begin{defi}
	une transposition $(ij)$ est la permutation qui échange $i$ et $j$.
\end{defi}

\begin{rema}
	toute permutation est une composée de transpositions.
\end{rema}

\begin{defi}
	La signature $\varepsilon(\sigma)$ pour $\sigma \subset S_n$ est le nombre de transpositions pour avoir $\sigma$. 
\end{defi}

\begin{defi}
	Soit $A = \left[a_{ij}\right]_{i,j \in \{1, ..., n\}}$ une matrice dans $\mathcal{M}_n(\KK)$.\\
	On pose le déterminant 
	$$det(A) = \sum\limits_{\sigma \in S_n} \varepsilon(\sigma)a_{\sigma_{1,1}} $$%###
\end{defi}

\begin{rema}
	Pour $n=2$ ou $n=3$, on obtient les formules usuelles.
\end{rema}

\begin{defi}
	Soit $E$, un $\KK$-espace vectoriel de dimension $n$ et 
	$$
\begin{aligned}
	f : \phantom{....} E^n &\to \KK\\
	(v_1,...,v_n) &\mapsto f(v_1, ..., v_n)
\end{aligned}
	$$

	$f$ est multilinéaire si elle est linéaire par rapport à chaque $v_i$ (quand les autres termes sont constants)
\end{defi}

%### fin du cours du 22/10

\begin{rema}
	$det(Id) = 1$ \'Etant donné que la seule manière d'avoir des termes non nuls dans le produit (de la formule du déterminant) est de prendre $\sigma = Id$
\end{rema}

\begin{prop}
	Soit $f:E^n \to \KK$, une application multilinéaire.\\
	Alors, si $\KK$ est tel que $1 + 1 = 2 \neq 0$, 
	$f$ est alternée si et seulement si $f$ est antisymétrique.
\end{prop}

\begin{proof}
	On prend $n = 2$ pour simplifier les notations.\\
	Si $f$ est antisymétrique, \\
	$f(v, v) = -f(v, v)$ (on permute les $v$)\\
	Donc $2f(v,v) = 0$\\
	donc $f(v,v) = 0$ car $2 \neq 0$.\\

	Si $f$ est alternée, alors,
	$$
\begin{aligned}
	0 &= f(v_1 + v_2, v_1 + v_2)\\
	&= f(v_1, v_1 + v_2) + f(v_2, v_1 + v_2)\\
	&= f(v_1, v_1) + f(v_1, v_2) + f(v_2, v_1) + f(v_2, v_2)\\
	&= 0 + f(v_1, v_2) + f(v_2, v_1) + 0\\
\end{aligned}
	$$
	D'où 
	$$f(v_1, v_2) = -f(v_2, v_1)$$
\end{proof}

\begin{theo}
	$A \mapsto det(A)$ est multilinéaire et alternée par rapport aux volumes de la matrice.
\end{theo}

\begin{proof}
%###finir la démo
\begin{itemize}	
	\item $A \mapsto a_{\sigma_{(1)1}} ... a$
\end{itemize}
\end{proof}
%###finir aussi la demo

\begin{corrol}{Conséquences pratiques}
\begin{itemize}	
	\item Le déterminant ne change pas quand on ajoute à une colonne un multiple d'une autre.
	\item le déterminant est multiplié par $-1$ si on permute deux colonnes.
\end{itemize}
\end{corrol}

\begin{theo}
	Soient $A = \left[a_{ij}\right]{(i,j)}$ et sa transposée $\phantom{A}^tA = \left[a_{ji}\right]_{(i,j)}$\\
	$det(A) = det(\phantom{A}^tA)$\\
	$det(\phantom{A}^tA) = \sum\limits_\sigma \varepsilon(\sigma) \prod\limits_{i=1}^n a_{i\sigma(i)}$
\end{theo}
%###écrire les théorèmes impossibles

\begin{lemme}
	$det \left(
\begin{bmatrix} a & b \\ c & d \end{bmatrix}
	\right)
	$
\end{lemme}


\end{document}
