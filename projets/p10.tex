\documentclass[../main.tex]{subfile}

%page x

\begin{document}
\part{Diagonalisation}

%###

\begin{theo}
	Soit $f \in \LLL(E)$, \\
	$f$ est diagonalisable si :
\begin{itemize}	
	\item $\khi_f$ est scindé
	\item pour toute valeur propre $\lambda$, la dimension de $E_\lambda(f)$ est la multiplicité de $\lambda$ dans $khi_f$.
\end{itemize}
\end{theo}

\begin{rap}
	Soit $P \in \KK[X]$, \\
	$P$ est scindé si
	$$\exists a_1, ..., a_n \in \KK \ \text{et } \ \exists \alpha_1, ..., \alpha_n \in \NN^*, \ / \ P = k \prod\limits_{i=1}^n (X-a_i)^{\alpha_i}$$
	avec $k$, une constante
\end{rap}


%###un exemple 
\begin{proof}
	Si $f$ est diagonalisable, on a 
	$$\exists \BBB \ / \ A = Mat_\BBB(f) = 
\begin{bmatrix}
	\lambda & 0 & ... & 0\\
	0 & \lambda & ... & 0\\
	... \\
	0 & 0 & ... & \lambda
\end{bmatrix}
	$$

	Alors $\Khi_f = \Prod(X - \lambda_i)^{\alpha_i}$, avec $\alpha_i$, la multiplicité
	%###

\begin{lemme}
	Les sous espaces vectoriels propres sont en somme directe.
\end{lemme}

\begin{proof}
	Pour tout $i$, soit $x_i \in E_{\lambda_i}$\\
	on suppose $x_1 + x_2 + ... + x_n = 0$\\
	on applique $f$, ainsi:
	$$\lambda_1x_1 + \lambda_2x_2 + ... + \lambla_nx_n = 0$$
	$$(\lambda_1 - \lambda_n)x_1 + (\lambda_2 - \lambda_n)x_2 + ... + (\lambla_n - \lambda_n)x_n = 0$$
\end{proof}
%###fin de la preuve du théorème 
\end{proof}

%si le polynome carac possède des racines...

\begin{rema}
	En particulier, pour diagonaliser, on trouve une base pour chaque sous espace propre avant de réunir les bases.
\end{rema}

\begin{prop}
	Pour tout $i$, $1 \leq dim E_{\lambda_i}(f) \leq \alpha_i$\\
	($\apha_i$ étant la multiplicité de $\lambda_i$ dans $\Khi_f$)
\end{prop}

\begin{proof}
	%### à refaire
\end{proof}
\end{document}
