\documentclass{article}
%\usepackage{fullpage}
\usepackage{comment}
\usepackage{color}
\usepackage{lipsum}
\usepackage{natbib} 
\usepackage[utf8]{inputenc} 
\usepackage[T1]{fontenc}
\usepackage[french]{babel}
\usepackage{amsthm} %principalement pour les démonstrations avec l'environnement {proof}
\usepackage{amssymb}
\usepackage{mathtools}
\usepackage{ulem}
\usepackage{stmaryrd} %pour faire des bornes d'intervalles d'entiers
\usepackage{import, fancyhdr, lastpage, sectsty, url, array, multirow, tabu, subfiles, graphicx, wrapfig, subcaption, setspace, booktabs}

\date{}
\author{}
\title{Algèbre linéaire}

\newtheorem{defi}{Définition}
\newtheorem*{ex}{Exemple}
\newtheorem*{exerc}{Exercice}
\newtheorem{rema}{Remarque}
\newtheorem*{nota}{Notation}
\newtheorem{rap}{Rappel}
\newtheorem{prop}{Proposition}
\newtheorem{propri}{Propriété}
\newtheorem*{theo}{Théorème}
\newtheorem{corrol}{Corrolaire}
\newtheorem*{mthd}{Méthode}
\newtheorem*{vocab}{Vocabulaire}
\newtheorem{lemme}{Lemme}

\newcommand\NN{\mathbb{N}}
\newcommand\QQ{\mathbb{Q}}
\newcommand\RR{\mathbb{R}}
\newcommand\CC{\mathbb{C}}
\newcommand\KK{\mathbb{K}}

\newcommand\LLL{\mathcal{L}}
\newcommand\BBB{\mathcal{B}}

%\newcommand\ssim{\accentset{\sim}{A}}

\newcommand\ssi{\Leftrightarrow}

\begin{document}

\begin{theo}{Théorème des polynômes annulateurs}
$$\phantom{a}$$
Un endomorphisme $f$ est diagonalisable si et seulement s'il existe un polynôme scindé à racines simples annulant $f$.\\
\end{theo}

$$\phantom{a}$$

\begin{proof}
$\phantom{a}$
	\section{sens direct}
	Soit $f$, un endomorphisme de polynôme caractéristique 
	$$\chi_f(X) = (-1)^n \prod\limits_{i=1}^p (X - \lambda_i)^{\alpha_i}$$

$\phantom{a}$\\[5mm]

	En supposant $f$ diagonalisable, il existe une base $\BBB = \{v_1, ..., v_n\}$ formée de vecteurs propres de $f$.\\
	{\tiny par définition d'une application diagonalisable}

$\phantom{a}$\\[1cm]


	Si $\{\lambda_i\}_{i \in \llbracket 1, p \rrbracket}$ sont \textbf{les} valeurs propres de $f$ alors, pour tout vecteur $v \in \BBB$, il existe $\lambda_j \in \{\lambda_i\}$ telle que $(f - \lambda_j Id) (v) = 0$\\

$\phantom{a}$\\[1cm]

\newpage
	Donc, $\forall v \in \BBB$, \\
	$(f-\lambda_1 Id) \circ ... \circ (f-\lambda_p Id)(v) = 0$ par commutativité des $(f - \lambda_k Id)$\\ 
	{\tiny
	$(f-\lambda_1 Id) \circ ... \circ (f-\lambda_{j-1} Id) \circ (f-\lambda_{j+1} Id) \circ ... \circ (f-\lambda_p Id) \circ (f-\lambda_j Id)(v) = 0$
\begin{rema}	
	commutativité des polynômes d'endomorphismes:\\
	Soient $P(X) = \sum\limits_{i=1}^n a_iX^i$ et $Q(X) = \sum\limits_{j=1}^m b_jX^j$
	$$
\begin{aligned}
	P(f) \circ Q(f) &= \sum_{i=1}^n a_if^i \circ \sum_{j=1}^m b_jf^j\\
	&= \sum_{i,j} a_ib_jf^i \circ f^j\\
	&= \sum_{i,j} a_ib_jf^{i+j}\\
	&= \sum_{i,j} b_ja_if^j \circ f^i\\
	&= \sum_{j=1}^m b_jf^j \circ \sum_{i=1}^n a_if^i\\
	&= Q(f) \circ P(f)
\end{aligned}
	$$
\end{rema}}

$\phantom{a}$\\[1cm]

	Comme on a $(f-\lambda_1 Id) \circ ... \circ (f-\lambda_p Id)(v) = 0$ % faire un renvoie
	sur la base $\BBB$, cette égalité se vérifie aussi sur $E$
	{\tiny tout vecteur de $E$ étant une combinaison linéaire de vecteurs de $\BBB$}\\

	$(f-\lambda_1 Id) \circ ... \circ (f-\lambda_p Id)(v) = 0$

$\phantom{a}$\\[5mm]

	Donc $Q(X) = \prod\limits_{i=1}^p (X-\lambda_i)$, qui est scindé et à racines simples, annule $f$.

$\phantom{a}$\\[1cm]
$\phantom{a}$\\[1cm]
\newpage
	\section{sens indirect}
	On peut écrire un polynôme scindé à racines simples sous la forme:
	$$Q(X) = \prod\limits_{i=1}^p (X - \lambda_i)$$
	où les $\lambda_i$ sont différents deux à deux.
	{\tiny les termes de ce produit sont donc premiers entre eux (les racines étant différentes)}
	
$\phantom{a}$\\[5mm]

	En appliquant $Q$ à un endomorphisme $f$, on a:
	$$Q(f) = \prod\limits_{i=1}^p (f - \lambda_i Id)$$

$\phantom{a}$\\[1cm]

	En supposant maintenant que $Q$ annule $f$, on a, d'après le lemme des noyaux %(que nina vient de démontrer)
	$$E = \bigoplus\limits_{i=1}^p ker(f - \lambda_i Id)$$
	{\tiny le noyau de cet endomorphisme correspond aux vecteurs pour lesquels $f(v) = \lambda_i v$, \textit{i.e.} les vecteurs de $E$ ayant pour valeur propre $\lambda_i$}

$\phantom{a}$\\[1cm]

	D'où $E = \bigoplus_{i=1}^p E_{\lambda_i}$

$\phantom{a}$\\[1cm]

	$f$ est donc diagonalisable.
	{\tiny par le théorème 1 vu dans le cours}
\end{proof}




\end{document}
