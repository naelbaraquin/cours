\documentclass[../main.tex]{subfile}

%page x

\begin{document}

\begin{prop}
Soit $E$ de dimension finie, et $F$ sous espace vectoriel de $E$.\\
\begin{enumerate}	
	\item $F$ est de dimension finie et $\dim F \leq \dim E$
	\item $\dim F = \dim E$ si et seulement si $F = E$
\end{enumerate}
\end{prop}
\begin{proof}
\begin{enumerate}	
	\item Semble évident mais ne l'est pas.
	\item Si $\dim F = \dim E$, \\
	on pose $\{f_1, ..., f_n\}$, base de $F$, est aussi une famille libre de $E$ avec $n = \dim E$ éléments dans une base de $E$.\\
	Pour la réciproque, si $\dim F \neq \dim F$, $E \neq F$
\end{enumerate}
\end{proof}

\begin{rema}
	$f_1, ..., f_k$ de $F$ sous espace vectoriel de $E$.\\
	$f_1, ..., f_k$ est une famille libre de $F$ si et seulement si c'est une famille libre de $E$
\end{rema}

\begin{rema}
	Le second point est utile en pratique pour montrer que deux espaces vectoriels $F$ et $G$ sont égaux, on montre que $F \subset G$ et que les deux sont de même dimension.
\end{rema}

\subsection{Exemples de dimensions (et bases)}
\begin{itemize}
	\item $\KK^n$\\
	base canonique : $e_i = (0, ..., 0, 1, 0, ..., 0)$ ou $1$ se trouve à la i-ème place.\\
	$(e_1, ..., e_n)$ est une base avec $\dim \KK^n = n$

	\item Soient $E, F$, deux espaces vecteurs de dimensions finies \\
	$E \times F$ est de dimension finie $\dim E + \dim F$\\
	Soit $(e_1, ..., e_n)$ base de $E$\\
	Soit $(f_1, ..., f_n)$ base de $F$\\
	$$e_i^\sim = (e_i, 0) \in E \times F$$
	$$f_j^\sim = (0, f_j) \in E \times F$$

\begin{exerc}	
	$(e_1^\sim, ..., e_n^\sim, f_1^\sim, ..., f_m^\sim)$
	%### finir les exemples
\end{exerc}
\end{itemize}

\begin{theo}
	On considère un système d'équations linéaires homogène à $n$ inconnues et $k$ équations.\\
	On suppose que ce système est échelonné, alors l'ensemble des solutions est de dimension $n-k$.
\end{theo}
\begin{proof}
\begin{rema}	
	Quitte à renommer les inconnues, on peut supposer que les inconnues principales sont $x_1, ..., x_k$, les autres sont des variables libres.\\
	$$
\left\{
\begin{array}{r}
	a_{11}x_1 + ... = 0\\
	a_{22}x_2 + ... = 0\\
	...\\
	a_{kk}x_k + ... = 0
\end{array}
\right.
	$$
	Alors, les solutions sont de la forme $(x_1, ..., x_n)$ avec,
	pour $j > k$, on pose arbitrairement $x_j = \lambda_j \in \KK$\\
	$$x_k = \alpha_{k k+1} \lambda_{k+1} + ... + \alpha_{kn}\lambda_n$$
	$$...$$
	$$x_1 = \alpha_{1, k+1} \lambda_{k+1} + ... + \alpha_{1n}\lambda_n$$
	Donc, 
	$$
(x_1, ..., x_n) = \lambda_{k+1}()
%###finir la démo
	$$
\end{rema}

\end{proof}

\begin{theo}
	Soient $F, G$, deux sous espace vectoriel de $E$ (de dimensions finies), \\
	alors, $\dim(F + G) = \dim F + \dim G - \dim(F \cap G)$
\end{theo}

\begin{proof}
	"Idée : construire une base sympathique de $F + G$"
\begin{itemize}	
	
	\item $F\cap G$ est de dimension finie ($F \cap G$ étant sous espace vectoriel de $F$) \\
	on pose alors $(e_1, ..., e_n)$ base de $F \cap G$.

	\item $(e_1, ..., e_n)$ est une famille libre de $F$\\
	se complète en une base $(e_1, ..., e_n, f_1, ..., f_s)$ de $F$
%### finir la démo
\end{itemize}
\end{proof}

\subsection{rang d'une famille de vecteurs}
\begin{defi}
	$$rg(v_1, ..., v_p) = \dim vect(v_1, ..., v_p)$$
\end{defi}

\begin{propri}
	On a 
\begin{itemize}
	\item $rg (v_1, ..., v_p) \leq dim E$
	\item $rg (v_1, ..., v_p) \leq p$
	\item $rg (v_1, ..., v_p) = dim E$ si et seulement si la famille est génératrice.
	\item $rg (v_1, ..., v_p) = 1$ si et seulement si la famille est libre.
\end{itemize}
\begin{proof}
\begin{itemize}	
	\item $vect{v_1, ..., v_p}$ est sous espace vectoriel de $E$.
	\item par définition, $(v_1, ..., v_p)$ engendre $vect(v_1, ..., v_p)$\\
	\item $vect (v_1, ..., v_p) = E$ si et seulement s'ils sont monodimensionels
	\item $rg(v_1, ..., v_p) = 1$ si et seulement si $v_1, ..., v_p$ est génératrice d'un sous espace vectoriel de dimension $p$\\
	si et seulement si $v_1, ..., v_p$ est une base de $vect(v_1, ..., v_p)$\\
	si et seulement si $v_1, ..., v_p$ est libre
\end{itemize}
\end{proof}
\end{propri}

\begin{rema}
	$vect(v_1, ..., v_p)$ ne change pas (le rang non plus) si:
\begin{itemize}	
	\item on permute deux vecteurs
	\item on multiplie un vecteur par un scalaire $\lambda \in \KK^*$
	\item on ajoute à un vecteur un multiple d'un autre
\end{itemize}
\end{rema}

On peut alors appliquer l'algorithme de Gauss  au calcul du rang pour des vecteurs de $\KK^n$.\\
Tout d'abord, on échelonne une famille de vecteur.\\
Une fois que la famille est échelonnée, le rang correspond au nombre de vecteurs non nuls.\\

\begin{rema}
	Cette famille de vecteurs est une base du sous espace vectoriel initial.
\end{rema}

\subsection{Somme directe}
	Soient $F$ et $G$ des sous espaces vectoriels de $E$\\
	$$F+G = \{f+g \ | \ f \in F, \ g \in G\}$$

	\textit{a priori} pour $u \in F+G$
	la paire $(f,g) \in F \times G \ / \ u=f+g$ n'est pas nécessairement unique.\\

\begin{defi}
	$F$ et $G$ sont en somme directe si et seulement si 
	$$\forall u \in F + G, \ \exists! (f,g) \in F \times G, \ u = f+g$$
\end{defi}

\begin{nota}
	On note alors $F \oplus G$
\end{nota}

\begin{rema}
	$F \oplus G$ désigne à la fois le sous espace vectoriel $F+G$ et la propriété selon laquelle ils sont en somme directe.
\end{rema}

%### reprendre l'exemple et les schéma

\begin{prop}
	Les énoncés suivants sont équivalents:
\begin{itemize}	
	\item $F$ et $G$ sont en somme directe
	\item $0 = f+g$ avec $f \in F$ et $g \in G$ implique $f=g=0$
	\item $F \cap G = \{0\}$
	\item En réunissant une base de $F$ et une base de $G$, on obtient une base de $F+G$
	\item $\dim(F+G) = \dim F + \dim G$
\end{itemize}
	(où les deux dernières propositions ne sont vraies qu'en dimension finie)
\end{prop}

\begin{proof}
%### reprendre la démo en faisant proprement les références
\end{proof}

\begin{defi}
	$F$ et $G$ sont supplémentaire si $F \oplus G = E$
\end{defi}

\begin{rema}
	Il peut être intéressant de "découper" l'espace en sous espaces plus simples.
\end{rema}

\begin{rema}
	Tout sous espace vectoriel $F$ de $E$ admet un supplémentaire.
\end{rema}

\begin{rema}
	Les supplémentaires ne sont pas uniques, on ne parle donc pas "du" supplémentaire.
\end{rema}

Cas général:\\
Soient $F_1, ..., F_k$, sous espaces vectoriels de $E$.\\
$F_1, ..., F_k$ sont en somme directe si :
$$\forall u \in F_1 + ... + F_k, \ \exists! (f_1, ..., f_k) \in F_1 \times ... \times F_k \ / \ u = f_1 + ... + f_k$$

\begin{prop}
	Les propositions suivantes sont équivalentes :
\begin{itemize}	
	\item $\bigoplus\limits_{i = 1}^k F_i$ 
	\item $\forall i \in [1, k], \ f_i \in F_i, \ \sum\limits_{i = 1}^k f_i = 0 \Rightarrow \forall i, \ f_i = 0$
	\item en réunissant un base de chaque $F_i$, on obtient une base de la somme
	\item $\dim \sum F_i = \sum \dim F_i$
\end{itemize}
\end{prop}

%### reprendre jusqu'à ...



\end{document}
