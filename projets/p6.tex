\documentclass[../main.tex]{subfile}

%page cours du 15 octobre

\begin{document}

\section{calcul de l'inverse d'une matrice}
\subsection{Systèmes linéaires}
\begin{prop}
	Si $A$ est inversible, alors $AX = B \ssi X = A^{-1}B$.
\end{prop}
	Donc on trouve $A^{-1}$ en résolvant un système d'équations linéaires d'inconnues $x_1, ..., x_n$ et de seconds membres génériques $b_1, ..., b_n$.\\
	$$X = 
	$$

%###expressions de X et B

\begin{ex}
	%$$A=
	%\left[
%\begin{array}{c}
	%sqdlfkj\\
	%qlmskdjf
%\end{array}
	%\right]$$
	%finir l'exemple
\end{ex}

\subsection{Méthode "magique"}


\end{document}
