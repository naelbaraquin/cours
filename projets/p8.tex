\documentclass[../main.tex]{subfile}

%page x

\begin{document}

\begin{rap}
	Soit $A$, une matrice, \\
	on note $A_{ij}$, la matrice obtenue en ne considérant ni la $i$-ème ligne ni la $j$-ème colonne.\\
	On a alors le cofacteur:\\
	$$C_{ij}(A) = (-1)^{i+j} det(A_{ij})$$
	La comatrice $Com(A)$ est alors la matrice des cofacteurs
	$$Com(A) = \left[(-1)^{i+j} det(A_{ij})\right]$$
\end{rap}

\begin{prop}
	$$A \cdot ^tcom A = ^tcom A \cdot A = det A \cdot I_n$$
	ainsi, si le déterminant, $A^{-1}=\frac{1}{det A} (^tcom A)$
\end{prop}
\begin{rema}
	cette formule est utile si $n = 2$, bof pour $n=3$ et à fuir si $n>3$\\
	Mais elle possède un intérêt théorique.
\end{rema}

\begin{proof}
	ne pas oublier la démo
\end{proof}

\section{Applications}
\subsection{Systèmes linéaires}
Dans un système de $n$ équations et $n$ inconnues pouvant être rapporté à une équation matricielle $AX=B$\\
\begin{prop}
	Il existe une unique solution si et seulement si $det A$ est non nul.
\end{prop}

Dans le cas où $det A \neq 0$, \\
on a les formules de Cramer:
$$x_i = \frac{1}{det A} det A_i$$
où $A_i$ est la matrice obtenue en remplaçant la i-ème colonne de $A$ par $B$.\\

\begin{proof}
	Si $det A \neq 0$, $A$ est inversible donc $X = AB^{-1}$\\
	Sinon :
	$f: \KK^n \to \KK^n$...%###à finir
\end{proof}

\subsection{Endomorphisme}
\begin{defi}
	Soit $E$, un $\KK$ espace vectoriel de dimension $n$.\\
	Soient de plus $f \in \LLL(E)$ et $\BBB$ une base de $E$.
	Alors, $det(Mat(f)_\BBB)$ ne dépend pas du choix de $\BBB$, on l'appelle déterminant de $f$.
\end{defi}

\begin{proof}
	$M = Mat (f)_\BBB$\\
	$\BBB'$, une autre base\\
	$M' = Mat(f)_\BBB'$\\
	alors $M'=P^{-1}MP$ avec $P$ la matrice de passage de $\BBB$ à $\BBB'$.\\
	or $detM' = det P^{-1} det M det P = det M$ ($det P^{-1} = 1 \over det P$)
\end{proof}

\begin{rema}
	Un autre invariant de $f$ est la trace de $f$:
	$$Tr(f) = Tr(Mat_\BBB f)$$
	La trace ne dépend pas du choix de $\BBB$
\end{rema}
%###


\subsection{Rang}
\begin{defi}
	Soit $A$, une matrice $A \subset M_{np}(\KK)$\\
	Une mineur d'ordre $n$ de $A$ est le déterminant d'une sous matrice carrée de $A$ de taille $r$.\\
	Un bordant d'un mineur $\delta$ d'ordre $r$ est un mineur d'ordre $r+1$ qui contient $\delta$. 
\end{defi}

\begin{ex}
	%###
\end{ex}

\begin{theo}
	Soit $A$, une matrice\\
	alors 
	$$rg A = r $$ si et seulement s'il existe $r \neq 0$ et tous les mineurs d'ordre $n+1$ sont nuls.\\
	si et seulement s'il existe un mineur $\delta$ d'ordre $r$ non nul et tous les bordants de $\delta$ sont nuls
\end{theo}

exemples d'application
%les reprendre correctement
\end{document}
