\documentclass[../main.tex]{subfile}

%page 1

\begin{document}
\part{Systèmes d'équations linéaires}

Soit $\KK$, un corps.
\begin{defi}
	Un système d'équtions linéaires à $n$ inconnues et $p$ équations est un système d'équations de la forme :
	$$
	(S)
	\left\{
	\begin{array}{l}
		a_{1,1}x_1 + ... + a_{1,n}x_n = b_1\\
		...\\
		a_{n,1}x_{n,1} + ... + a_{n, n}x_{n,n} = b_n
	\end{array}
	\right.
	$$
	avec avec $a_{i,j}$ et $b_i$ des éléments de $\KK$\\
	et $x_j$ sont les inconnues.
\end{defi}

\begin{defi}
	Une solution est le $n$-uplet $(x_1, ..., x_n)$ tel que x... sont solutions de toutes les équations.
\end{defi}

\begin{defi}
	Les $b_1, ..., b_p$ sont appelés seconds membres.
\end{defi}

\begin{rema}
	à priori, $n \neq p$ 
\end{rema}


\section{Résolution}
\subsection{\'Equivalence de systèmes}
Pour résoudre, on se ramène à un système équivalent plus simple :
$$
(S) \Leftrightarrow (S') 
%mettre un système
$$

$(S) \Leftrightarrow (S')$ signifie que les deux systèmes ont les mêmes solutions.

\subsection{Méthode du pivot de Gauss}
On ne change pas les solutions en faisant une des trois opérations suivantes:
\begin{itemize}
	\item changer l'ordre des équations
	\item multiplier une équation par un élément $\lambda \in \KK \backslash \{0\}$
	\item Ajouter à une équation un multiple d'une autre
\end{itemize}
ou toute opération qui peut se décomposer en une série de telles opérations

%reprendre l'exemple

%\begin{ex}
%	$$
%	\left\{
%	\begin{array}{l}
%		
%	\end{array}
%	\right.
%	$$	
%\end{ex}

%reprendre "erreur possible"

Méthode du pivot de Gauss:\\
\begin{itemize}
	\item Si $a_{1,1} \neq 0$\\
	\begin{nota}
		$a_{1,1}$ est alors appelé le pivot
	\end{nota}
	pour tout $i$ strictement supérieur à 1, on remplace la ligne $L_i$ par $L_i - \frac{a_{i,1}}{a_{1,1}}$
\end{itemize}

\`A la fin, on obtient un système dit échelonné, c'est-à-dire de la forme:\\
	$$
	\left\{
	\begin{array}{r}
		a'_{1, j_1}x_{j_1} + ... + a'_{1, n}x_n = b'1\\

	\end{array}
	\right.
	$$

\end{document}
