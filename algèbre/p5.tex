\documentclass[../main.tex]{subfile}

%page cours du 8 octobre

\begin{document}

\section{Matrice d'application linéaire}

Soit $f : E \to E'$, une application linéaire.\\
Soit $\BBB = (e_1, ..., e_n)$, une base de $E$\\
Soit $\BBB' = (e_1', ..., e_p')$, une base de $E'$\\

On écrit $f(e_1), ..., f(e_n)$ dans la base $\BBB'$\\

$$
\begin{aligned}
	f(e_1) &= a_{11}e'_1 + ... + a_{p1}e'_p\\
	f(e_2) &= a_{12}e'_2 + ... + a_{p2}e'_p\\
	...\\
	f(e_n) &= a_{1n}e'_n + ... + a_{pn}e'_p\\
\end{aligned}
$$

On définit la matrice de $f$ relativement aux bases $\BBB$ et $\BBB'$ par:
$$
Mat(f)_{\BBB\BBB'} =
\left(
\begin{array}{cccc}
	a_{11} & a_{12} & ... & a_{1n}\\
	a_{21} & a_{22} & ... & a_{2n}\\
	...\\
	a_{p1} & a_{p2} & ... & a_{pn}\\
\end{array}
\right)
$$

\begin{ex}
	Pour l'application:
	$$
\begin{aligned}
	f :\phantom{......} \RR^2 &\to \RR^2\\
	 (x,y) &\mapsto (-x+6y , -x+4y)
\end{aligned}
	$$

	$\BBB_1 = (e_1, e_2) = ((1, 0), (0, 1))$ est la base canonique de $\RR^2$.\\
	$\BBB_2 = (e'_1, e'_2) = ((3, 1), (2, 1))$\\

	Alors :\\
\begin{itemize}	
	\item $Mat(f)_{\BBB_1\BBB_1} = $
\end{itemize}
\end{ex}

\end{document}
