\documentclass[../main.tex]{subfile}

%page 2 (postulats de la meca q)

\begin{document}
\part{Chapitre 2 : Espaces vectoriels}
Soit $\KK$, un corps ($\RR$, $\CC$, ou autre)

\section{Notion d'espace vectoriel}
\subsection{Définitions}
\begin{defi}{vague}
	Un $\KK$-espace vectoriel est un ensemble d'éléments appelés vecteurs tels qu'on puisse les additionner entre eux et les multiplier par des scalaires, c'est-à-dire des éléments de $\KK$
	avec des relations naturelles de compatibilité
\end{defi}

\begin{defi}
	Un $\KK$-espace vectoriel est un ensemble $E$ muni de deux lois:\\
\begin{itemize}
	\item une loi de composition interne:
	$$
\begin{aligned}
	+ : E \times E &\to E \\
	(u,v) &\mapsto u + v
\end{aligned}
	$$

	\item une loi de composition externe:
	$$
\begin{aligned}
	\cdot : \KK \times E &\to E \\
	(\lambda, u) &\mapsto \lambda \cdot v
\end{aligned}
	$$
\end{itemize}
	Ces lois vérifient:
\begin{itemize}
	\item $\forall u, v, w \in E, \ (u + v) + w = u + (v + w)$ \\
	la loi $+$ est donc associative

	\item $\forall u, v \in E, \ u + v = v + u$\\
	la loi $+$ est donc commutative

	\item $\exists 0_E \in E, \ \forall u \in E, \ u + 0_E = 0_E + u = u$\\
	la loi $+$ admet un élément neutre

	\item $\forall u \in E, \ \exists v \in E, \ u+v = v+u = 0_E$\\
	chaque élément de $E$ admet, par $+$, un inverse ou opposé

	\item $\forall \lambda, \mu \in \KK, \ \forall u \in E, \ \lambda \cdot (\mu \cdot u) = (\lambda \cdot \mu) \cdot u$\\
	la loi $\cdot$ est associative

	\item $\forall \lambda, \mu \in \KK, \ \forall u \in E, \ (\lambda + \mu) \cdot u = \lambda \cdot u + \mu \cdot u$\\
	la loi $\cdot$ est distributive à gauche

	\item $\forall \lambda \in \KK, \ \forall u, v \in E, \ (u + v) \cdot \lambda = \lambda \cdot u + \lambda \cdot v $\\
	la loi $\cdot$ est distributive à droite

	\item $\forall u \in E, \ 1 \cdot u = u$\\
	la loi $\cdot$ admet un élément neutre
\end{itemize}
\end{defi}

\begin{rema}
	Dans le troisième axiome, l'élément neutre est unique.\\
	Dans le quatrième axiome, le vecteur $v$ est en fait unique, on le note $-u$.
\end{rema}

\begin{prop}
	On a également, $\forall u \in E, \ \forall \lambda \in \KK$:
\begin{enumerate}
	\item $\lambda \cdot 0_E = 0_E$
	\item $0_{\KK} \cdot u = 0_E$
	\item $\lambda \cdot u = 0_E \Rightarrow \lambda = 0_\KK \ \text{ou} \ u = 0_E$
	\item $(-\lambda) \cdot u = \lambda \cdot (-u) = - (\lambda \cdot u)$
\end{enumerate}
\end{prop}

\begin{proof}
\begin{enumerate}
	\item 
$$
\begin{aligned}
	\lambda \cdot 0_E &= \lambda \cdot (0_E + 0_E)\\
	&= \lambda \cdot 0_E + \lambda \cdot 0_E\\
	&= \lambda \cdot 0_E + 0_E\\
	\lambda \cdot 0_E = O_E
\end{aligned}
$$
\item 
$$
\begin{aligned}
	0_\KK \cdot u &=   (0_\KK + 0_\KK) \cdot u\\
	&= 0_\KK \cdot u + 0_\KK \cdot u\\
	&= 0_\KK \cdot u + 0_\KK\\
	0_\KK \cdot u = O_\KK
\end{aligned}
$$
\item Si $\lambda = 0_\KK$, cf. 2\\
Si $\lambda \neq 0$, alors $\lambda^{-1} \in \KK$,\\
$$0 = \lambda^{-1} \cdot 0 = \lambda^{-1} (\lambda \cdot u) = (\lambda^{-1} \cdot \lambda) \cdot u = 1 \cdot u = u$$
\end{enumerate}
\end{proof}

\begin{nota}
	On note souvent :
\begin{itemize}
	\item $0_E = 0$ et $0_\KK = 0$
	\item $u - v = u + (-v)$
\end{itemize}
\end{nota}

\begin{lemme}
	$\forall u, v, w \in E, u + w = v + w \Rightarrow u = v$
\end{lemme}

\begin{proof}
	$$
\begin{aligned}
	v &= (u + w) - w\\
	&= u + (w - w)\\
	&= u + 0_E\\
	&= u
\end{aligned}
	$$
	donc $v = u$
\end{proof}

\begin{rema}
\begin{itemize}
	\item Pour $\lambda \in \KK$ et $u \in E$
	$u \cdot \lambda$ ne veut rien dire.

	\item Pour $u, v \in E$
	$u \cdot v$ ne veut rien dire
\end{itemize}
\end{rema}

\begin{ex}
	1/ Pour les lois de compositions internes et externes usuelles, 
\begin{itemize}
	\item $\KK$ est un $\KK$-espace vectoriel
	\item $\KK^n$ est un $\KK$-espace vectoriel 
	\item plus généralement, si $E_1$ et $E_2$ sont des $E_1 \times E_2$ est un $\KK$-espace vectoriel
\end{itemize}
	2/ Soit $E$, un $\KK$-espace vectoriel et $A$, un ensemble qualconque, 
\begin{itemize}
	\item $\mathcal{F}(A, E)$, l'ensemble des applications de $A$ dans $E$, est un $\KK$-espace vectoriel
		$$
		\forall f_1, f_2 \in \mathcal{F}(A, E), \ \forall \lambda \in \KK,
		$$
		$$
		\begin{aligned}
			f_1 + f_2 : A &\to E\\
			a &\mapsto f_1(a) + f_2(a)
		\end{aligned}
		$$
		$$
		\begin{aligned}
			\lambda \cdot f_1 : A &\to E\\
			a &\mapsto \lambda \cdot f_1(a) 
		\end{aligned}
		$$
	\item Si $\KK = \RR$ et $A = I \subset \RR$, un intervalle, on peut avoir $\mathcal{F}(I, \RR)$
	\item Si $\KK = \RR$ et $A = \NN$, on a $\mathcal{F}(\NN, \RR)$, l'ensemble des suites numériques
\end{itemize}
	3/ $\KK[X]$, l'ensemble des polynômes\\
	4/ $M_{n,p}(\KK)$, l'ensemble des matrices à coefficient dans $\KK$, à $n$ lignes et $p$ colonnes.
\end{ex}

\begin{rema}
	$\RR^2$, munit de la loi $+$ usuelle et $\lambda \cdot (x_1, x_2) = (\lambda \cdot x_1, 0)$\\
	n'est pas un $\KK$-espace vectoriel, pourquoi?
\end{rema}

\subsection{Sous-espace vectorel}

\begin{defi}
	Soit $E$, un $\KK$-espace vectoriel, et $F \subset E$.\\
	$F$ est un sous espace vectoriel de $E$ s'il s'agit d'un $\KK$-espace vectoriel pour les lois $+$ et $\cdot$ de $E$.
\begin{itemize}
	\item $\forall u, v \in F, \ u + v \in F$
	\item $\forall \lambda \in \KK, \ \forall u \in F, \ \lambda \cdot u \in F$
	\item $+$ et $\cdot$ vérifient les propriétés des lois de composition interne et externe des espaces vectoriels
\end{itemize}
\end{defi}

\end{document}
