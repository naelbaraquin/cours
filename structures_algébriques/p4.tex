\documentclass[../main.tex]{subfile}

%cours du 3 octobre

\begin{document}
\begin{theo}{de Lagrange}
	Soit $G$, un groupe d'ordre fini et $H < G$, \\
	alors $|H|$ divise $|G|$.\\
	En particulier, $\forall g \in G$, $|g| = |<g>|$ divise $|G|$
\end{theo}

\begin{corrol}
	Tout groupe d'ordre premier $p$ est isomorphe à $\ZZ/p\ZZ$.
\end{corrol}

\begin{proof}
	Soit $G$ tel que $|G| = \emptyset$\\
	Soit $g \in G$, $g \neq e$.\\
	On a l'ordre de $|G| = p$ (donné par le théorème de Lagrange)\\
	Donc $|g|$ vaut $1$ ou $p$.\\
	\'Etant donné que $g \neq e$, on a $|g| = p$\\
	Donc $|<g>| = p$. On a $<g> < G$ et $|G| = p$\\
	Donc $<g> = G$\\
	Donc $G$ est cyclique d'ordre $p$\\
	Donc $G \cong \ZZ/p\ZZ$
\end{proof}

\begin{theo}{Identité de Bézout}
	Soient $m, n \in \ZZ$, tels que $PGCD(m,n) = k$\\
	Alors, $\exists u, v \in \ZZ$, tels que $mu + nv = k$
\end{theo}

\begin{proof}
	$|x| = m$, $|y| = n$, $PGCD(m,n) = 1$\\
	$mu + nv = 1$\\
	$$
\begin{aligned}
	(xy)^{nv} &= x^{nv}y^{nv}\\
	&= x^{nv}(y^n)^v\\
	&= x^{nv}e\\
	&= x^{nv}\\
	&= x^{1 - mu}\\
	&= x(x^m)^{-u}\\
	&= x
\end{aligned}
	$$
\end{proof}


\end{document}
