\documentclass[../main.tex]{subfile}

%page 2

\begin{document}
\begin{rema}
	Un groupe monogène est nécessairement commutatif.
\end{rema}

\begin{proof}
	Si $G = <g>$, alors $G = \{g^n, \ n \in \ZZ\}$\\
	De plus, si $k, l \in \ZZ$, on a $g^kg^l = g^{k+l} = g^{l+k} = g^lg^k$
\end{proof}

\begin{rema}
	En particulier, un groupe non commutatif ne peut pas être monogène (contraposée de la remarque précédente)
\begin{ex}	
	$S_3$, par exemple, n'étant pas commutatif, n'est pas non plus monogène.
\end{ex}
\end{rema}

\subsection{Produit de groupes}

Soient $G_1$ et $G_2$, deux groupes.\\
Le groupe produit $G = G_1 \times G_2$ est défini par l'ensemble $\{(g_1, g_2) \ | \ g_1 \in G_1,\ g_2 \in G_2\}$\\
avec l'opération $* : G \times G \to G$ définie par 
$$(g_1, g_2) \times (h_1, h_2) = (g_1h_1, g_2h_2)$$
On définit de manière similaire le produit d'une famille de groupes.\\

\begin{rema}
	$$G_1 \times (G_2 \times G_3) = (G_1 \times G_2) \times G_3 = G_1 \times G_2 \times G_3$$
\end{rema}

\begin{ex}
\begin{itemize}	
	\item $$\ZZ^2 = \ZZ \times \ZZ = \{(m, n) \ | \ m, n \in \ZZ\}$$
	$$\RR^2 = \RR \times \RR$$

	\item
	$$
\begin{aligned}	
	\frac{\ZZ}{2\ZZ} \times \frac{\ZZ}{2\ZZ} &= \{(a, b) \ | \ a \in \frac{\ZZ}{2\ZZ} , \ b \in \frac{\ZZ}{3\ZZ}\}\\
	&= \{(\bar{0}, \bar{0}), (\bar{0}, \bar{1}), (\bar{0}, \bar{2}), (\bar{1}, \bar{0}), (\bar{1}, \bar{1}), (\bar{1}, \bar{2})\}\\
	&= <(\bar{1}, \bar{1})> \ \text{qui est un groupe cyclique d'ordre 6}
\end{aligned}
	$$

	\item $\frac{\ZZ}{2\ZZ} \times \frac{\ZZ}{2\ZZ}$ n'est pas cyclique:\\
	$\{(\bar{0}, \bar{0}), (\bar{0}, \bar{1}), (\bar{1}, \bar{0}), (\bar{1}, \bar{1})\}$ contient un élément d'ordre $1$ et trois éléments d'ordre $2$.\\
	$\frac{\ZZ}{2\ZZ} \times \frac{\ZZ}{2\ZZ}$ ne contient pas d'élément d'ordre $4$ et n'est donc pas cyclique.
\end{itemize}
\end{ex}

\end{document}
