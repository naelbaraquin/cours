\documentclass[../main.tex]{subfile}

%partie 3

\begin{document}
\part{Théorie d'intégration au sens de Riemann}

\section{Intégration des fonctions constantes par morceaux}
Soit $I = [a, b]$ un intervalle compact (i.e. fermé borné) de $\RR$.\\

\begin{defi}
	Une subdivision de $I$ est la donnée d'un N+1-uplet $(t_0, t_1, ..., t_N)$ tel que:\\
	$a = t_0 < t_1 < ... < t_{N-1} < t_N = b$
\end{defi}

\begin{defi}
	Une fonction $g:I \to \RR$ est dite constante par morceaux lorsqu'il existe une subdivision de $I$ en $(t_0, t_1, ..., t_{N})$ adaptée à $g$ telle que:
	$$\forall i \in [0, N-1], \ \exists k \in \RR, \ g_{|[t_i, t_{i+1}[} (x) = g_i(x) = k$$
\end{defi}

\begin{rema}
	Dès qu'il y a une subdivision adaptée à $g$, on a la certitude que $g$ admet plusieurs subdivisions adaptées.
\end{rema}

\begin{defi}
	Si $g$ est une fonction constante par morceaux dans $I$,\\
	la quantité $\sum g_i \cdot (t_{i+1} - t_i)$ est notée:
	$$\int_a^bg(x)dx$$
\end{defi}

\begin{proof}[Preuve de la consistence de la définition]
	On doit s'assurer que $\sum g_i \cdot (t_{i+1} - t_i)$ ne dépende pas de la subdivision choisie.\\

	On raisonne alors par récurrence sur $\NN$.\\
	Soit $n \in \NN$.\\
	On souhaite montrer $\mathcal{P}(n)$ vraie:\\
	$\mathcal{P}(n)$ : S'il existe une subdivision de $I$ en $n$ intervalles, alors, $\sum\limits_{i=0}^{n-1} g_i \cdot (t_{i+1} - t_i)$ ne dépend pas de l'intervalle choisi de la subdivision.
	$$\phantom{a}$$
	Pour $n=1$, \\
	On a $t_0 = a$ et $t_1 = b$, \\
	Par conséquent la fonction $g$ est constante sur $[a,b[$, \\
	Soit $k \in \RR$ tel $g(x) = k$, \\
	$$\sum\limits_{i=0}^0 g_i(a-b) = k(a-b)$$
	Considérons une autre subdivision adaptée à $g$:\\
	$a = \tau_0 < \tau_1 < ... < \tau_p = b$\\
	$$
	\begin{aligned}
		\sum\limits_{i=0}^{p-1} g_{|[\tau_i, \tau_{i+1}[} \cdot (\tau_{i+1} - \tau_i) &= \sum\limits_{i=0}^{p-1} k \cdot (\tau_{i+1} - \tau_i)\\
		&= k \cdot \sum\limits_{i=0}^{p-1} (\tau_{i+1} - \tau_i)\\
		&= k \cdot (\tau_{p} - \tau_0) \text{ la somme étant télescopique}\\
		&= k \cdot (b - a)\\
	\end{aligned}
	$$
	$\mathcal{P}(1)$ est donc vraie.\\
	$$\phantom{a}$$

	Supposons maintenant, pour un $N$ fixé, la propriété $\mathcal{P}(N-1)$ vraie.\\
	On considère une subdivision adaptée à $g$:
	$$(a= t_0 < t_1 < ... < t_{N-1} < t_N = b)$$
	Soit $(a = \tau_0 < \tau_1 < ... < \tau_{N-1} < \tau_N = b)$, une autre subdivision adaptée à $g$.\\
	Il existe un indice $0 \leq i_0 \leq p-1$ tel que : $\tau_{i_0} \leq t_{N-1} < \tau_{i_0 + 1}$\\

	La fonction $g_{|[a, t_{N-1}]}$ étant constante par morceaux, \\
	les subdivision $(a = t_0 < t_1 < ... < t_{N-1})$\\
	et $(a = \tau_0 < \tau_1 < ... < \tau_{i_0} < t_{N-1})$\\
	ou $(a = \tau_0 < \tau_1 < ... < \tau_{i_0} = t_{N-1})$\\
	sont adaptées.\\

	La fonction $g_{|[t_{N-1}, t_N]}$ étant constante par morceaux, \\
	les subdivision $(t_{N-1} < t_N)$\\
	et $(t_{N-1} < \tau_{i_0} < ... < \tau_p = b)$\\
	ou $(t_{N-1} < \tau_{i_0} = \tau_p = b)$\\
	sont adaptées.\\

	On a ainsi, \\
	$$
	\begin{aligned}
		&\sum\limits_{i=0}^{n-1} g_{|[t_i, t_{i+1}[} \cdot (t_{i+1} - t_i)\\
		&= \left( \sum\limits_{i=0}^{n-1} g_{|[t_i, t_{i+1}[} \cdot (t_{i+1} - t_i) \right) + \left( g_{|[t_{n-1}, t_{n}[} \cdot (t_{n} - t_{n-1}) \right)\\
		&\text{Où le premier terme découle de l'hypothèse de récurrence tandis que le second découle de l'initialisation} \\
		&= \left( \sum\limits_{i=0}^{i_0-1} g_{|[\tau_i, \tau_{i+1}[} \cdot (\tau_{i+1} - \tau_i) \right) + \left( g_{|[\tau_{i_0}, t_{n-1}[} \cdot (t_{n-1} - \tau_{i_0}) \right) + \left( g_{|[t_{n-1}, \tau_{i_0 + 1}[} \cdot (\tau_{i_0 + 1} - t_{n-1}) \right) + \left( \sum\limits_{i=i_0 + 1}^{p-1} g_{|[\tau_{i_0}, \tau_{i_0+1}[} \cdot (\tau_{i_0+1} - \tau_{i_0}) \right)\\
		&= \left( \sum\limits_{i=0}^{i_0-1} g_{|[\tau_i, \tau_{i+1}[} \cdot (\tau_{i+1} - \tau_i) \right) + \left( g_{|[\tau_{i_0}, \tau_{i_0+1}[} \cdot (\tau_{i_0+1} - \tau_{i_0}) \right) + \left( \sum\limits_{i=i_0 + 1}^{p-1} g_{|[\tau_{i_0}, \tau_{i_0+1}[} \cdot (\tau_{i_0+1} - \tau_{i_0}) \right)\\
		&\text{Le second terme découlant du fait que $g$ est constante sur } [\tau_{i_0}, \tau_{i_0+1}] \\
		&= \sum\limits_{i=0}^{p-1} g_{|[\tau_i, \tau_{i+1}[} \cdot (\tau_{i+1} - \tau_i)\\
	\end{aligned}
	$$
	$\mathcal{P}(N)$ est donc vraie.\\
	Par récurrence, $\mathcal{P}(n)$ est vraie pour tout entier $n$ non nul, la quantité $\int_a^b g(x)dx$ est donc bien définie.
\end{proof}

\begin{theo}
	Soient $f$ et $g$, deux fonctions constantes par morceaux et soit $\alpha \in \RR$, \\
	la fonction $\alpha f + g$ est alors constante par morceaux, et:\\
	$$\int_a^b \alpha f + g = \alpha \int_a^b f + \int_a^b g$$
	L'opérateur intégral est donc un opérateur linéaire.
\end{theo}

\begin{proof}
	Soient $\{a = t_0 < t_1 < ... < t_n = b\}$ et $\{autre subdivision\}$, 
	deux subdivisions adaptées respectivement à $f$ et $g$.\\
	On considère la subdivision suivante:
	$$\{a = T_0 < T_1 < ... < T-n\}$$
	obtenue en superposant les deux subdivisions précédentes.\\
	Cette subdivision est alors adaptée à la fois à $f$ et à $g$.\\
	$$\forall i \in [0, n[, \ f_{|[T_i, T_{i+1}]} \ \text{et} \ g_{|[T_i, T_{i+1}]} \ \ \ \text{sont constantes}$$
	$\alpha f + g$ est donc constant par morceaux sur $[a, b]$.\\
	$$
	\begin{aligned}
		\int\limits_a^b \alpha f + g &= \sum\limits_i (\alpha f + g)_{|[T_i, T_{i+1}[} \cdot (T_{i+1} - T_i)\\
		&=\alpha \sum\limits_i (f_{|[T_i, T_{i+1}[} + g_{|[T_i, T_{i+1}[}) \cdot (T_{i+1} - T_i)\\
		&=\alpha \sum\limits_i f_{|[T_i, T_{i+1}[} \cdot (T_{i+1} - T_i) + g_{|[T_i, T_{i+1}[} \cdot (T_{i+1} - T_i)\\
		&=\alpha \int\limits_a^b f + \int\limits_a^b g
	\end{aligned}
	$$
\end{proof}

\section{Fonctions Riemann-intégrables}

\begin{nota}
	Soit $f$, une fonction sur $[a, b]$ à valeurs dans $\RR$.\\
	Soient, de plus, $\mathcal{C}^+(f)=\{\Phi : [a, b] \to \RR, \ \text{constante par morceaux} \ | \ \forall x \in [a, b], \ f(x) \leq \Phi(x)\}$\\
	et, $\mathcal{C}^-(f)=\{\Phi : [a, b] \to \RR, \ \text{constante par morceaux} \ | \ \forall x \in [a, b], \ f(x) \geq \Phi(x)\}$
\end{nota}

\begin{defi}
	$f$ est Riemann-intégrable si et seulement si :
	\begin{enumerate}
		\item $\mathcal{C}^+(f)$ et $\mathcal{C}^-(f)$ sont non vides.
		\item $\sup \int\limits_a^b \Phi^- = \inf \int\limits_a^b \Phi^+$\\
		Avec $\Phi^- \in \mathcal{C}^-(f)$ et $\Phi^+ \in \mathcal{C}^+(f)$
	\end{enumerate}

	Quand elle est définie, on note cette grandeur $\int\limits_a^b f$
\end{defi}

\begin{ex}[de fonctions non Riemann-intégrables]
	$$
	\begin{aligned}
	\mathbb{1}_{\mathbb{Q}} &: [0, 1] \to \{0, 1\}\\
	&: x \mapsto 
	\left \{
	\begin{array}{l}
		1 \text{ si } x \in \mathbb{Q}\\
		0 \text{ si } x \notin \mathbb{Q}
	\end{array}
	\right .
	\end{aligned}
	$$

	$\Phi \in \mathcal{C}^-(\mathbb{1}_\mathbb{Q})$
\end{ex}

\begin{proof}
	%%%%%%%%%%%%%%%%%%%%%%dernier cours



	$$\phantom{a}$$
	On montre maintenant l'autre implication.\\
	Soit $\varepsilon > 0$, \\
	Soient $\Phi^+ \in \mathcal{C}^+$ et $\Phi^- \in \mathcal{C}^-$ telles que
	$$\int\limits_a^b \Phi^+ - \Phi^- \leq \varepsilon$$

	$\forall \Phi \in \mathcal{C}^-(f)$, \\
	$\Phi \leq f \leq \Phi^+$\\
	Donc, $\int \Phi \leq \int \Phi^+$
	L'ensemble $\{\int \Phi : \Phi \in \mathcal{C}^-(f)\}$ est une partie majorée de $\RR$.\\
	Cet ensemble admet donc une borne supérieure:
	$$B = \sup \{\int \Phi : \Phi \in \mathcal{C}^-(f)\}$$

	De la même manière, \\
	$\forall \Phi \in \mathcal{C}^+(f)$, \\
	$\int \Phi \geq \int \Phi^-$
	L'ensemble $\{\int \Phi : \Phi \in \mathcal{C}^+(f)\}$ est une partie minorée de $\RR$.\\
	Cet ensemble admet donc une borne inférieure:
	$$b = \inf \{\int \Phi : \Phi \in \mathcal{C}^+(f)\}$$

	Comme $\int \Phi^+ \geq B$ et $\int \Phi^- \leq b$\\
	On a : $\int \Phi^+ \int \Phi^- = B - b$\\
	D'où : $B - b \leq \int \Phi^+ - \Phi^- \leq \varepsilon$\\

	Cette assertion étant valable pour tout $\varepsilon > 0$, on a $b = B$.\\
	$f$ est donc Riemann-intégrable.
\end{proof}

\begin{prop}
	L'intégrale de Riemann:
	\begin{itemize}
		\item est positive : $f \geq 0 \Rightarrow \int f \geq 0$
		\item est linéaire : $\forall \alpha \in K, \ \int \alpha f + g = \alpha \int f + \int g$
		\item vérifie $|\int f| \leq \int |f|$
		\item vérifie la relation de Chasle.
	\end{itemize}
\end{prop}

%%%%%%%%%%%%%%%%%%%%%%%%%%%%%manque les démonstrations du passage de constante par morceaux à continues

\begin{theo}
	Toute fonction continue sur l'intervalle $[a, b]$ est Riemann-intégrable.
\end{theo}

\begin{rappel}
	$f$ est continue en $x \in [a, b]$ si et seulement si\\
	$$\forall \varepsilon > 0, \ \exists \delta > 0, \ |x - x_0| < \delta \Rightarrow |f(x) - f(x_0)| < \varepsilon$$
\end{rappel}

\begin{defi}
	$f$ est continue sur $[a, b]$\\
	$$\forall x \in [a, b], \ \forall \varepsilon > 0, \ \exists \delta > 0, \ |x - y| < \delta \Rightarrow |f(x) - f(y)| < \varepsilon$$
\end{defi}

\begin{lemme}
	Toute fonction continue sur un compact $[a, b]$ est uniformémeent continue sur $[a, b]$.
\end{lemme}

\begin{defi}
	$$\forall \varepsilon > 0, \ \exists \delta > 0, \forall x \in [a, b], \ |x - y| < \delta \Rightarrow |f(x) - f(y)| < \varepsilon$$
\end{defi}

\begin{proof}
	Ce lemme est démontré par l'absurde.\\
	$$\exists > 0, \ \forall \delta > 0, \ \exists x, y \in [a, b], $$
	$$|x-y| < \delta \ \text{et} \ |f(x) - f(y)| > \varepsilon$$

	Prenons un $\varepsilon$ fixé.\\
	Soit $n \in \NN^*$, on pose $\delta = \frac{1}{n}$, \\
	puis, $x_n \in [a, b]$ et $y_n \in [a, b]$ tels que \\
	$|x_n - y_n| < \frac{1}{n}$ et $|f(x_n) - f(y_n)| > \varepsilon$\\

	$\forall n \in \NN, \ x_n \in [a, b]$, \\
	La suite $(x_n)$ est bornée, \\
	donc, d'après le théorème de Bolzano-Weirstrass, il existe une suite extraite $(x_{\varphi(n)})$ qui converge vers $x_\infty = [a, b]$.\\
	$$
	\begin{aligned}
	|y_{\varphi(n) - x_\infty}| &= |y_{\varphi(n)} - x_{\varphi(n)} + x_{\varphi}(n) - x_\infty|\\
	&= |y_{\varphi(n)} - x_{\varphi(n)}| + |x_{\varphi}(n) - x_\infty|\\
	&= \frac{1}{\varphi(n)} + |x_{\varphi}(n) - x_\infty|
	\end{aligned}
	$$

	Comme $\lim\limits_{n \to + \infty} \frac{1}{\varphi(n)} = 0$\\
	et $\lim\limits_{n \to + \infty} |x_{\varphi(n) - x_\infty}| = 0$\\
	on a $\lim\limits_{n \to + \infty} y_{\varphi(n)} = x_\infty$\\

	Mais $f$ est continue en $x_\infty$, \\
	Donc $\lim\limits_{n \to + \infty} f(y_{\varphi(n)} = f(x_\infty))$\\
	donc $\lim\limits_{n \to + \infty} |f(y_{\varphi(n)}) - f(x_{\infty})| = 0$\\
	donc $\lim\limits_{n \to + \infty} |f(y_{\varphi(n)}) - f(x_{\varphi(n)})| = 0$\\
	$$\forall \varepsilon > 0, \ \exists N \in \NN, \ n \geq N \Rightarrow |f(y_{\varphi(n)}) - f(x_{\varphi(n)})| < \varepsilon$$
	
\end{proof}












































\begin{theo}
	Toute fonction continue sur $[a, b]$ est Riemann-intégrable.
\end{theo}

\begin{proof}
	Soit $\varepsilon > 0,$\\
	Comme $f$ est uniformément continue sur $[a, b]$, \\
	$$\exists \delta > 0, \ |x-y| \leq \delta \Rightarrow |f(x) - f(y)| \leq \frac{\varepsilon}{b-a}$$
	Soit $N \in \NN$ tel que $\frac{b-a}{N} = \delta$\\
	Considérons la subdivision suivante:\\

	%$$
	%\right \{
	%\begin{array}{l}
		%a=t_0\\
		%t_k = a + k \frac{b-a}{N}, \ \ \ k \in [1, N]\\
	%\end{array}
	%\left .	
	%$$

	Représentation d'une portion de la subdivision:

	\begin{center}

	\def\svgwidth{0.5\textwidth}
	\input{dessin.pdf_tex}
	
	\end{center}

	On pose les deux fonctions suivantes:\\
	$$\Phi_{|[t_k, t_{k+1}[}^+ = \sup(f(x))$$
	$$\Phi_{|[t_k, t_{k+1}[}^- = \inf(f(x))$$

	Ainsi, \\
	$$\int_{t_k}^{t_{k+1}} |\Phi^+ - \Phi^-| \leq \frac{\varepsilon}{b-a}$$

	D'où, \\
	$$\int_{a}^{b} |\Phi^+ - \Phi^-| \leq \varepsilon$$

	$f$ est donc Riemann-intégrable.
\end{proof}


\section{Propriétés des fonctions Riemann-intégrables}
	
\begin{theo}
	Soit $f$, une fonction Riemann-intégrable sur $[a, b]$, \\
	Alors, pour tout réel $\alpha$, la fonction qui à $t$ associe $\int_\alpha^t f$ est continue.
\end{theo}

\begin{proof}
	Soit $f$, une fonction Riemann-intégrable sur $[a, b]$, \\
	$\forall \alpha \in \RR, $
	$$
	\begin{aligned}
		\varphi : [a, b] & \to \RR\\
		t & \mapsto \int_\alpha^t f
	\end{aligned}
	$$

	Soit $t_0 \in [a, b]$, \\
	On souhaite montrer que $\varphi$ est continue. i.e.:\\
	$\forall \varepsilon > 0, \ \exists \delta > 0,$\\
	$$|t - t_0| < \delta \Rightarrow |\varphi(t) - \varphi(t_0)| < \varepsilon$$
	Autrement dit:
	$$|t - t_0| < \delta \Rightarrow \left| \int_\alpha^t f - \int_\alpha^{t_0} f \right| < \varepsilon$$

	Soit $\varepsilon > 0$, \\
	$f$ étant Riemann-intégrable, il existe deux fonctions $\Phi^-$ et $\Phi^+$ constantes par morceaux telles que $\Phi^- \leq f \leq \Phi^+$.\\
	$f$ est donc bornée et :\\
	$$\exists M > 0 \ / \ \forall x \in [a, b], \ f(x) < M$$

	On pose alors $\delta = \frac{\varepsilon}{M}$, ainsi, \\
	$\forall t \in [a, b], \ |t - t_0| < \delta \Rightarrow$
	$$
	\begin{aligned}
		\left|\int_\alpha^t f - \int_\alpha^{t_0} f \right| &= \left|\int_{t_0}^t f \right|\\
		& \leq \left|\int_{t_0}^t M \right|\\
		& \leq M|t - t_0|\\
		& \leq \varepsilon
	\end{aligned}
	$$
	
\end{proof}

\begin{theo}
	Si $f$ est une fonction continue sur $[a, b]$, \\
	Alors la fonction qui à $t$ associe $\int_\alpha^t f$ est dérivable et sa dérivée est $f$. (i.e. $\int_\alpha^t f$ est une primitive de $f$).
\end{theo}

\begin{theo}[théorème fondammental du calcul intégral]
	Soit $f$, une fonction Riemann-intégrable sur $[a, b]$.\\
	Si $f$ admet une primitive $F$ sur $[a, b]$, \\
	alors, \\
	$$\int_a^b f = F(b) - F(a)$$
\end{theo}































%cours du 11 avril.

\begin{defi}[Somme de Riemann]
	Soit $f$, une fonction définie sur $[a, b]$, \\
	Soit $\{a=t_0 < t_1 < ... < t_n = b\}$, une subdivision de $[a, b]$.\\
	Soit $\{ c_i \in [t_i, t_{i+1}] \ | \ 0 \leq i <n\}$.\\

	On appelle somme de Riemann associée, le nombre $R(f) = \sum\limits_{i=0}^{n-1} f(c_i)(t_{i+1} - t_i)$
\end{defi}

\begin{theo}
	Si $f$ est Riemann-intégrable sur $[a, b]$, \\
	alors, $\lim\limits_{n \to + \infty} R_n(f) = \int_a^b f(t)dt$
\end{theo}

\section{Intégration sur un intervalle quelconque}

\begin{defi}
	$f$, est localement intégrable sir $ I \subset \RR$ lorsque pour tout segment $[c, d] \subset I$, $f$ est Riemann intégrable sur $[c, d]$
\end{defi}

\begin{defi}
	Soit $f: [a, b[ \to \RR$, une fonction localement intégrable.\\
	L'intégrale impropre $\int_a^b f$ est convergente lorsque $\int_a^X f$ admet une limite finie quand $X$ tend vers $b$. %rajouter les resp.
\end{defi}

\begin{ex}
	On cherche à déterminer $\int_0^{+\infty} \lambda e^{-\lambda t} dt$\\
	$$\int_0^X \lambda e^{-\lambda t} dt = \left[ -e^{-\lambda t} \right]_0^X = -e^{-\lambda X} + 1$$
	$$\lim\limits_{X \to +\infty} -e^{-\lambda X} + 1 = 1$$
\end{ex}















\begin{theo}
	Soit $f : [a, b[ \to \RR^+$, une fonction \emph{positive} localement intégrable, \\
	alors, $\int_a^b f$ est convergente si et seulement si $\exists M \in \RR, \ \forall c \geq a, \ \int_a^c f \leq M$
\end{theo}

\begin{proof}[Heuristique]
	On pose $
	\begin{aligned}
	\Phi :& [a, +\infty[ \to \RR\\
	& c \mapsto \int_a^c f	
	\end{aligned}$
	est convergente et majorée.
\end{proof}

\begin{coroll}
	Soient $f$ et $g$, deux fonctions positives telles que $f \leq g$, \\
	alors la convergence de $\int_a^b g$ implique la convergence de $\int_a^b f$.
\end{coroll}

\begin{theo}
	Si $\int_a^b |f|$ converge, alors $\int_a^b f$ converge.\\
	On dit alors que $f$ est absolument convergente.
\end{theo}

\begin{ctex}
	$$\phantom{a}$$
	$\int_0^{+\infty} \frac{\sin(t)}{t} dt$ converge tandis que\\
	$\int_0^{+\infty} |\frac{\sin(t)}{t}| dt$ diverge.\\

	\begin{itemize}
		\item En 0, il n'y a pas de problème car $\lim\limits_{t \to +\infty} \frac{\sin(t)}{t} = 0$\\
		et la fonction $
		\begin{aligned}
			f:& ]0, + \infty[ \to \RR\\
			&t \mapsto \frac{\sin(t)}{t}
		\end{aligned}
		$ est prolongeable par continuité en $0$ en posant $f(0) = 0$.\\

		\item En $+\infty$, \\
		on intègre par parties $\int_1^X \frac{\sin(t)}{t} dt$ en posant : $u = \frac{1}{t}, \ \ \ u' = - \frac{1}{t^2}$ et $v = - \cos(t), \ \ \ v'= \sin(t)$:\\
		$$\int_1^X \frac{\sin(t)}{t} dt = \left[ - \frac{\cos(t)}{t} \right]_1^X - \int_1^X \frac{\cos(t)}{t^2} dt$$

		\item $\int_1^X |\frac{\cos(t)}{t^2}| dt \leq \int_1^X \frac{1}{t^2} dt$\\
		Le second membre étant convergent, on a : $\int_1^X \frac{\cos(t)}{t^2} dt$ converge.

		\item $\frac{-1}{X} \leq \frac{- \cos(X)}{X} \leq \frac{1}{X}$, par conséquent:\\
		$$\lim\limits_{X \to +\infty} \left[ \frac{- \cos (t)}{t} \right]_1^X = \cos(1)$$

		\item $\int_0^{+\infty} \frac{\sin(t)}{t} dt$ étant convergente, 
		montrons que $\int_0^{+\infty} |\frac{\sin(t)}{t}| dt$ diverge.\\
		Soit $N \geq 4$, 
		$$
		\begin{aligned}
			\int_1^N |\frac{\sin(t)}{t} dt &\geq
			\sum\limits_{k=1}^{E(\frac{N - \frac{3\pi}{2}}{\pi})} \int_{k \pi + \frac{\pi}{4}}^{k \pi + \frac{3 \pi}{4}} |\frac{\sin(t)}{t}| dt\\
			&\geq \sum\limits_{k=1}^{E(\frac{N - \frac{3\pi}{2}}{\pi})} \frac{\sqrt{2}}{2} \int_{k \pi + \frac{\pi}{4}}^{k \pi + \frac{3 \pi}{4}} \frac{1}{t} dt\\
			&\geq \sum\limits_{k=1}^{E(\frac{N - \frac{3\pi}{2}}{\pi})} \frac{\sqrt{2}}{2} \int_{\pi + \frac{\pi}{4}}^{\pi + \frac{3 \pi}{4}} \frac{1}{t} dt\\
		\end{aligned}
			\lim\limits{N \to +\infty} \sum\limits_{k=1}^{E(\frac{N - \frac{3\pi}{2}}{\pi})} \frac{\sqrt{2}}{2} \int_{\pi + \frac{\pi}{4}}^{\pi + \frac{3 \pi}{4}} \frac{1}{t} dt = + \infty \\
		$$
	\end{itemize}
\end{ctex}

\begin{theo}
	Si $\int_a^b g$ converge et si $|f| \sim_b g$, (ou si $|f| =_b o(g)$)\\ %mettre le b sous les opérateurs
	alors $\int_a^b |f|$ converge.
\end{theo}

\begin{proof}
	$|f| \sim_b g \Rightarrow \lim\limits_{X \to b} \frac{|f|}{g} = 1$\\
	$|f| =_b o(g) \Rightarrow \lim\limits_{X \to b} \frac{|f|}{g} = 0$.
\end{proof}













\end{document}




%contre exemples en mathématiques, Bertrand Hanchecorne, corrigé du test page 209
%Calcul infinitésimal, J. Dieudonné
