\documentclass[../main.tex]{subfile}

\begin{document}
\part{Relation d'équivalence et relation d'ordre}

\section{Relation d'équivalence}
Soit $E$ un ensemble (théorie ZF).
\begin{defi}[Produit cartésien]
	On définit le produit cartésien de deux ensembles $A$ et $B$ tel que:
	$$A \times B = \{ (x, y) \ | \ x \in A, \ y \in B\}$$
\end{defi}

\begin{defi}[Relation d'équivalence]
	Une relation d'équivalence sur $E$ est une partie du produit cartésien $E \times E$, notée $\mathcal{R}$.\\
	Lorsqu'un couple $(x, y) \in E$ est en relation par $\mathcal{R}$, on écrit $x \mathcal{R} y$.\\
	Une relation d'équivalence répond aux conditions suivantes:
	\begin{itemize}
		\item $\mathcal{R}$ est \emph{réflexive} : $\forall x \in E, \ x \mathcal{R} x$
		\item $\mathcal{R}$ est \emph{symétrique} : $\forall (x, y) \in E^2, \ x \mathcal{R} y \Rightarrow y \mathcal{R} x$
		\item $\mathcal{R}$ est \emph{transitive} : $\forall (x, y, z) \in E^3, \ x \mathcal{R} y \text{ et } y \mathcal{R} z \Rightarrow x \mathcal{R} z$
	\end{itemize}
\end{defi}

\begin{ex}
	$\phantom{a}$\\
	\begin{itemize}
		\item Sur n'importe quel ensemble, la relation d'égalité est une relation d'équivalence.
		\item Dans l'ensemble des entiers relatifs, $\forall (x, y) \in \mathbb{Z}^2$, on pose la relation d'équivalence suivante : $x \mathcal{R} y \Leftrightarrow 2 | x - y $ .
		\begin{proof}
			$\phantom{a}$\\

			Soit $x \in \mathbb{Z}$.\\
			$x-x = 0$ est divisible par 2.\\
			$\mathcal{R}$ est donc réflexive.
			$$\phantom{a}$$

			Soient $x, y \in \mathbb{Z}$ tels que $x \mathcal{R} y$,\\
			On a donc $x - y$ divisible par 2.\\
			$\exists k \in \mathbb{Z} \ / \ x -y = 2 k$\\
			Donc, $y -x = -(y-x) = -2k$\\
			Donc, $2|y-x$\\
			Donc, $y \mathcal{R} x$\\
			Donc, $\mathcal{R}$ est donc symétrique\\
			$$\phantom{a}$$

			Soient $x, y, z \in \mathbb{Z}$ tels que $x \mathcal{R} y$ et $y \mathcal{R} z$\\
			$x - z = x-y + y-z$\\
			Or, $\exists k \in \mathbb{Z}, \ x - y = 2k$\\
			et, $\exists k' \in \mathbb{Z}, \ y - z = 2k'$\\
			Donc, $x-y+y-z = 2(k+k')$ est divisible par 2, par conséquent, $x \mathcal{R} z$ et $\mathcal{R}$ est transitive.
		\end{proof}
	\end{itemize}
\end{ex}

\begin{defi}[Classe d'équivalence]
	Soit $\mathcal{R}$ une relation d'équivalence sur $E$.\\
	Soit $x \in E$, on appelle \emph{classe d'équivalence} l'ensemble $\mathcal{C}(x) = \{ y \in E \ | \ x \mathcal{R} y \}$
\end{defi}

\begin{ex}
	Sur $\mathbb{Z}, \ x \mathcal{R} y \Leftrightarrow 2 | x-y$\\
	\begin{itemize}
		\item $\mathcal{C}(0) = \{y \in \mathbb{Z} \ | \ 0 \mathcal{R} y\}$
		\item $\mathcal{C}(0) = \{y \in \mathbb{Z} \ | \ -y \text{ divisible par } 2\}$
		\item $\mathcal{C}(0) = \{\text{nombres pairs}\}$
		\item $\mathcal{C}(2) = \{y \in \mathbb{Z} \ | \ 2 \mathcal{R} y\}$
		\item $\mathcal{C}(2) = \{y \in \mathbb{Z} \ | \ \exists k \in \mathbb{Z} , \ 2-y = 2 k\}$
		\item $\mathcal{C}(2) = \{\text{nombres pairs}\}$
		\item $\mathcal{C}(1) = \{y \in \mathbb{Z} \ | \ 1 \mathcal{R} y\}$
		\item $\mathcal{C}(1) = \{y \in \mathbb{Z} \ | \ \exists k \in \mathbb{Z} , \ 1-y = 2 k\}$
		\item $\mathcal{C}(1) = \{\text{nombres impairs}\}$
	\end{itemize}
\end{ex}

\begin{prop}
	$\forall (x, y) \in E^2$, \\
	\begin{itemize}
		\item $x \mathcal{R} y \Rightarrow \mathcal{C}(x) = \mathcal{C}(y)$
		\begin{proof}
			Supposons $x \mathcal{R} y$, \\
			Montrons que $\mathcal{C}(x) \cap \mathcal{C}(y)$\\
			Soit $z \in \mathcal{C}(x)$, on a donc $x \mathcal{R} z$\\
			Or $y \mathcal{R} x$ (par symétrie de $\mathcal{R}$)\\
			Donc $y \mathcal{R} z$ (par transitivité de $\mathcal{R}$)\\
			Donc $z \in \mathcal{C}(y)$\\
			Donc $\mathcal{C}(x) \subset \mathcal{C}(y)$\\
			On montre de la même manière que $\mathcal{C}(y) \subset \mathcal{C}(x)$
			Par conséquent, $x \mathcal{R} y \Rightarrow \mathcal{C}(x) = \mathcal{C}(y)$
		\end{proof}
		\item $(\mathcal{C}(x) \cup \mathcal{C}(y) = \emptyset) \Rightarrow \mathcal{C}(x) = \mathcal{C}(y)$
		\begin{proof}
			Soit $z \in \mathcal{C}(x) \cup \mathcal{C}(y)$\\
			On a $x \mathcal{R} z$ et $z \mathcal{R} y$\\
			Donc, par transitivité, $x \mathcal{R} y$ \\
			Par conséquent, $\mathcal{C}(x) = \mathcal{C}(y)$
		\end{proof}
		\item Les classes d'équivalence forment une partition de $E$.
	\end{itemize}
\end{prop}

\begin{defi}[Ensemble quotient]
	Soit $\mathcal{R}$, une relation d'équivalence sur $E$.\\
	L'ensemble quotient, noté $\frac{E}{\mathcal{R}}$ est l'ensemble dont les éléments sont les classes d'équivalence.\\
	Il existe une application de passage au quotient:
	$$\Pi : \left\{
		\begin{array}{l}
			E \to \frac{E}{\mathcal{R}}\\
			x \mapsto \mathcal{C}(x)
		\end{array}
		\right .$$
\end{defi}

\begin{ex}
	$(x, y) \in \mathbb{Z}^2, \ x \mathcal{R} y \Leftrightarrow 2 \ | \ x -y$\\
	$\frac{\mathbb{Z}}{\mathcal{R}} = \{\mathcal{C}(0), \mathcal{C}(1)\} = \{\bar{0}, \bar{1}\} = \frac{\mathbb{Z}}{2 \mathbb{Z}}$
\end{ex}

\section{Relations d'ordre}

\begin{defi}[Relation d'ordre]
	Une \emph{relation d'ordre} $\prec$ sur $E$ est une partie de $E \times E$ qui vérifie les propriétés suivantes:
	\begin{itemize}
		\item \emph{Réflexivité :} $\forall x \in E, \ x \prec x$
		\item \emph{Anti-symétrie :} $\forall x, y \in E, \ (x \prec y) \text{ et } (y \prec x) \Rightarrow x = y$
		\item \emph{Transitivité :} $\forall x, y, z \in E, \ (x \prec y) \text{ et } (y \prec z) \Rightarrow (x \prec z)$
	\end{itemize}
\end{defi}

\begin{defi}
	Une relation d'ordre est dite totale lorsque $\forall (x, y) \in E^2, \ x \prec y \text{ ou } y \prec x$
\end{defi}

\begin{defi}
	Soit $A$ une partie de $(E, \prec)$, 
	\begin{itemize}
		\item Un \emph{majorant} de $A$ est un élément $M \in E$ tel que $\forall a \in A, \ a \prec M$
		\item Un \emph{minorant} de $A$ est un élément $m \in E$ tel que $\forall a \in A, \ m \prec a$
		\item $A$ est \emph{bornée} lorsqu'elle admet à la fois un majorant et un minorant.
		\item On dit que $A$ admet un \emph{plus grand élément (ou maximum)} s'il existe un majorant $M$ de $A$ tel que $M \in A$.
		\item On dit que $A$ admet un \emph{plus petit élément (ou minimum)} s'il existe un minorant $m$ de $A$ tel que $m \in A$.
		\item $A$ admet une \emph{borne supérieure} $B \in E$ si $B$ est un majorant de $A$ et si pour tout majorant de $A$, on a $B \prec M$
		\begin{rema}
			Quand il existe, $B$ est le plus petit des majorants.
		\end{rema}
		\item $A$ admet une \emph{borne inférieure} $b \in E$ si $b$ est un minorant de $A$ et si pour tout minorant de $A$, on a $m \prec b$
		\begin{rema}
			Quand il existe, $b$ est le plus grand des minorants.
		\end{rema}
	\end{itemize}
\end{defi}

\begin{ex}
	$\phantom{a}$\\

	$E = \mathbb{R}$, doté de l'ordre standard $\leq$\\
	$A = [0; 1[$\\
	$A$ est bornée par -42 (en tant que minorant) et 1,1 (en tant que majorant)\\
	$A$ admet un plus petit élément 0 \\
	$A$ n'admet pas de plus grand élément mais admet en revanche une borne supérieure 1.\\
	$$\phantom{a}$$

	$E = \mathbb{Q}$, doté de l'ordre standard $\leq$\\
	$A = \{x \in \mathbb{Q} | \ x^2 < 2\}$\\
	$A$ est majorée par 24 mais n'a pas de borne supérieure car $\sqrt{2} \notin \mathbb{Q}$
	$$\phantom{a}$$

	$E = \mathbb{N}^*$, doté de la relation d'ordre $\prec$ telle que $\forall a \prec b \Leftrightarrow a | b$\\
	On montre qu'il s'agit d'une relation d'ordre:\\
	\begin{itemize}
		\item $\forall a \in \NN^{*2}, \ a|a$\\
		Par conséquent, $a \prec a$ et $\prec$ est transitive.

		\item Soient $a \in \NN^*$ et $b \in \NN^*$\\
		$a|b \Rightarrow \exists k \in \NN, \ ka=b$\\
		$b|a \Rightarrow \exists k' \in \NN, \ kb=a$\\
		Donc, $a = kk'a$\\
		Donc, $kk' = 1$\\
		Donc, $k = k' = 1$\\
		Donc, $a = b$\\
		Par conséquent, $\prec$ est anti-symétrique.

		\item Soient $a, b, c \in \NN^{*}$\\
		$a|b \Rightarrow \exists k \in \NN, \ ka=b$\\
		$b|c \Rightarrow \exists k' \in \NN, \ k'b=c$\\
		Donc, $c = k'ka$\\
		Donc $a|c$,\\
		$\prec$ est par conséquent transitive.\\

		$\phantom{a}$\\
		Pour cette même relation d'ordre,\\
		On pose $A = \{2; 3; 5\}$.
		120 est un majorant de $A$.\\
		$A$ n'a pas de plus grand élément.\\
		$A$ admet 30 comme borne supérieure (avec $30 = PPCM(2;3;5)$)
		$A$ admet pour minorant 1.
		$A$ admet aussi 1 comme borne inférieure
		\begin{rema}
			$\prec$ n'est pas un ordre total.
		\end{rema}
	\end{itemize}
\end{ex}

\section{Construction de $\RR$}

\begin{axiome}
	Il existe un ensemble $\NN$ muni d'une relation d'ordre $\leq$ telle que :
	\begin{enumerate}
		\item $\leq$ est totale.
		\item Toute partie non vide admet un plus petit élément.
		\item Toute partie majorée non vide admet un plus grand élément.
		\item L'ensemble n'a pas de plus grand élément.
	\end{enumerate}
\end{axiome}

\begin{theo}
	Soit $(\mathcal{N}, \prec)$, un ensemble munit d'une relation d'ordre vérifiant les propriétés précédentes, alors, il existe une bijection croissante de $\mathcal{N}$ dans $\NN$.
	$$(\forall x, y \in \mathcal{N}, \ x \prec y \Rightarrow f(x) \leq f(y))$$

\end{theo}

\emph{Qui est $\mathbb{Z}$?}
On définit sur $\NN^2$ la relation d'équivalence $\mathcal{R}$ définie par 
$$(m,n) \mathcal{R} (m', n') \Leftrightarrow m + n' = m' + n$$
Ainsi, on définit alors l'ensemble des entiers relatifs tel que :
$$\mathbb{Z} = \frac{\mathbb{N} \times \mathbb{N}}{\mathcal{R}}$$

\emph{Qui est $\mathbb{Q}$?}
On définit sur $\mathbb{Z} \times \mathbb{Z}^*$ la relation d'équivalence $\mathcal{R}'$ définie par:
$$(p, q) \mathcal{R}' (p', q') \Leftrightarrow pq' = p'q$$
Ainsi, on définit alors l'ensemble des rationnels tel que :
$$\mathbb{Q} = \frac{\mathbb{Z} \times \mathbb{Z}^*}{\mathcal{R}'}$$

\emph{Qui est $\mathbb{R}$?}
Soit $\mathcal{P}_M(\mathbb{Q})$, l'ensemble des parties majorées non vides de $\mathbb{Q}$. 
On le munit de la relation d'équivalence $\mathcal{R}''$ définie par:
$$A \mathcal{R}'' B \Leftrightarrow \text{A et B ont le même ensemble de majorants dans} \mathbb{Q}$$
Ainsi, on définit alors l'ensemble des réels tel que :
$$\mathbb{R} = \frac{\mathbb{Q} \times \mathbb{Q}}{\mathcal{R}''}$$

\begin{ex}
	$\phantom{a}$\\
	$A = \{1\}$\\
	$B = \{-2, 0, 1\}$\\
	$A \mathcal{R}'' B$\\
	$$\phantom{a}$$
	$A= \{x \in \mathbb{Q} | x^2 \leq 2\}$\\
	On appelle $\sqrt{2}$ la classe d'équivalence de $A$
\end{ex}

\section{Caractérisation de $\mathbb{R}$}

\begin{theo}
	Tout corps totalement ordonné, complet et archémédien est isomorphe à $\mathbb{R}$.
\end{theo}

\begin{theo}[de la borne supérieure]
	Toutes parties de $\mathbb{R}$ majorée et non vide admet une borne supérieure.
\end{theo}

\end{document}
