\documentclass[../main.tex]{subfile}

\begin{document}
\part{Introduction à l'atomistique}

\section{Rappels}

\emph{Effet photoélectrique :} (a permis de démontrer l'existence des photons) (montre les insuffisances de la mécanique classique.)
\begin{itemize}
         \item l'énergie cinétique des photons et électrons dépend de la fréquence du rayonnement incident.
         \item relation fréquence/énergie 
         \item l'énergie est indépendante de l'intensité lumineuse
         \item permet de rendre compte de la dualité onde-corpuscule (verif)
\end{itemize}


\emph{Rayonnement du corps noir :}
   Soit un corps opaque et non-réfléchissant, \\
   son spectre d'émission ne dépend alors que de sa température.\\

\emph{Spectre d'émission d'un atome :} $\Delta E = \frac{hc}{\lambda}$\\
$h \equiv J \cdot s$


\emph{Longueur d'onde de De Broglie :} $\lambda = \frac{h}{mv} = \frac{h}{p}$ où $mv = p$ est la quantité de mouvement. 
\begin{rema}
      La particule peut être dotée d'une quantité de mouvement même si elle ne possède pas de masse.
\end{rema}


\emph{Inégalité de Heisenberg :} $\Delta q \cdot \Delta p \geq \frac{\hbar}{2}$\\
\begin{itemize}
         \item $\Delta q$, incertitude sur la position
         \item $\Delta p$, incertitude sur la quantité de mouvement
         \item $\hbar = \frac{h}{2\pi}$
\end{itemize}
   On ne peut avoir une connaissance précise de la vitesse et de la position d'une même particule en même temps.\\


   L'objectif de ce cours est de résoudre l'équation de Schrödinger $\hat{H}\Psi = E \Psi$

\newpage
\end{document}
