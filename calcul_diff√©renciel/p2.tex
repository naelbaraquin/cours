\documentclass[../main.tex]{subfile}

%page 2

\begin{document}
\part{Différentiabilité}
\begin{rap}
	Rappels en dimension 1:\\
	$f : \RR \to \RR$ est dérivable en $x_0 \in \RR$ de dérivée $\lambda$\\
	$\lim\limits_{x \to x_0} \frac{f(x) - f(x_0)}{x - x_0} = \lambda$\\

	Une définition équivalente est la suivante:\\
	$f$ est dérivable en $x_0$ de dérivée $\lambda$ s'il existe une fonction $\varepsilon(h)$ telle que :\\
\begin{itemize}	
	\item $f(x) = f(x_0) + \lambda(x-x_0)\varepsilon(x-x_0)$
	\item $\lim\limits_{h \to 0} \varepsilon(h) = 0$
\end{itemize}
	Il suffit alors de poser $\varepsilon(h) = \frac{f(x-h) - f(x_0)}{h}$
	%### fig21
	Où on a alors la pente de $\Delta_{x_1}$ s'exprimant $\frac{f(x_1) - f(x_0)}{x_1 - x_0}$\\
	On peut aussi voir $\Delta_{x_1}$comme le graphe d'une application affine:
	$$u(x) = f(x_0) + \tau(x-x_0)$$
	$$u(x_0+h) = f(x_0) + \tau h$$
	Ces deux points de vue induisent deux points de vue si la dérivée $f'(x_0) \in \RR$ est la pente de la tangente :
	$$f'(x_0) : \RR \to \RR$$
	$$f'(x_0) : h \mapsto f'(x_0) h$$
	Le second point se généralise aux dimensions superieures à 1.
\end{rap}

\section{Applications différentiables}
$$f : \Omega \subset E \to F$$
$$p_0 \in \Omega$$

\begin{defi}
	$f$ est différentiable en $p_0$ s'il existe une application linéaire $l : E \to F$ est une fonction $\varepsilon : \Omega \to F$ telle que:
\begin{itemize}	
	\item $f(p_0 + h) = f(p_0) + l(h) + ||h|| \varepsilon (h)$
	\item $\lim\limits_{h \to 0} \varepsilon (h) = 0$
\end{itemize}
\end{defi}

\begin{prop}
	Si $f$ est différentiable en $p_0$, l'application linéaire de la différentielle est unique. On l'appelle différentielle de $f$ en $p_0$
et on la note $L = D_{p_0} f$
\end{prop}

\begin{proof}
	Si $L_1$ et $L_2$ conviennent:
	$$
\begin{aligned}
	f(p_0+h) &= f(p_0) + L_1h + ||h||\varepsilon_1(h)\\
	f(p_0+h) &= f(p_0) + L_2h + ||h||\varepsilon_2(h)\\
\end{aligned}
	\\
	\Rightarrow 0 = 0 + (L_1- L_2)h + ||h||(\varepsilon_1 - \varepsilon_2)(h)
	$$
	fixons $h$, pour $t \in [0, 1]$:\\
	$$
\begin{aligned}
	&(L_1 - L_2)(th) + ||th||(\varepsilon_1 - \varepsilon_2)(th)\\
	&=t((L_1 - L_2)(h) + ||h||(\varepsilon_1 - \varepsilon_2)(h))\\
	&=t0\\
	&=0
\end{aligned}
$$
	D'où $\lim\limits_{t \to 0} (\varepsilon_1 - \varepsilon_2)(th) = 0$, \\
	Par conséquent $(L_1 - L_2)(h) = 0$\\
	Donc, pour tout $h$ tel que $(p_0 + h) \in \Omega$, $L_1(h) = L_2(h)$\\
	Donc $L_1 =L_2$ sur une petite boule $B(0, 1)$.\\
	On peut alors généraliser à $L_1 = L_2$
\end{proof}

\begin{defi}
	Si $f$ est différentiable en tout point de $\Omega$, on dit que $f$ est différentiable sur $\Omega$.\\
	Si de plus, $
\begin{aligned}
	\Omega &\to \mathcal{L}(E, F)\\
	p &\mapsto D_pf
\end{aligned}
	$ est continue, ont dit que $f$ est de classe $\mathcal{C}^1$.
\end{defi}

\begin{ex}
\begin{itemize}	
	\item $f : \RR^n \to \RR^m$ linéaire:
	$$
\begin{aligned}
	f(p_0 + h) &= f(p_0) + f(h)\\
	&= f(p_0) + f(h) + ||h||\varepsilon(h)\\
\end{aligned}
	$$
	$f$ est différentiable, et $D_pf = f$

	\item $
\begin{aligned}
	f : \RR^2 &\to \RR\\
	(x, y) &\mapsto x^2 + y^2
\end{aligned}$
	$p_0 =(x_0, y_0)$\\
	$h =(u,v)$\\
	%###finir l'exemple

\end{itemize}
\end{ex}

\end{document}
