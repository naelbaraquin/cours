\documentclass[../main.tex]{subfile}

\begin{document}

\part{Rudiments de mécanique quantique et d'atomistique}
\subsection{Les postulats de la mécanique quantique}

La mécanique quantique est basée sur six postulats, en voici quatre :\\
\begin{enumerate}
         \item L'état d'un système à un instant $t$ est complètement définit par la connaissance de sa fonction d'onde, notée $\Psi(\vec{r}, t)$\\
            Dans un cas à $n$ particules, on a $\Psi(\vec{r_1}, \vec{r_2}, ..., \vec{r_n}, t)$
            $\vec{r}$ est alors la variable d'espace.
         \item \`A toute grandeur physique $A$ mesurable, on associe un opérateur linéaire et hermitique noté $\hat{A}$.
         \item L'évolution d'un état d'un système est régie par l'équation de Schrödinger dépendante du temps:
            $$\hat{H}\Psi(\vec{r}, t) = i\hbar \frac{\partial \Psi (\vec{r}, t)}{\partial t}$$
         \item Les valeurs de $A$ mesurables expérimentalement ne peurvent être que les valeurs propres associées à $\hat{A}$
\end{enumerate}

\paragraph{Postulat 1}

\begin{itemize}
         
   \item La fonction d'onde n'a aucun sens physique.\\

   \item On a $d P(\vec{r}, t) = \Psi^*(\vec{r}, t) \Psi(\vec{r}, t) dV = |\Psi(\vec{r}, t)|^2$\\
   (où $\Psi^*$ est le conjugué complexe de $\Psi$)\\
      $dP(\vec{r}, t)$ est alors la probabilité infinitésimale de trouver la particule dans le volume $dV$ centré en $(\vec{r}, t)$\\

   $d P(\vec{r}, t)$ correspond à la probabilité de trouver une particule dans le volume $dV$ autour de $\vec{r}$ à l'instant $t$.\\

   \item On a donc :
   $$\frac{dP(\vec{r}, t)}{dV} = \Psi^*(\vec{r}, t) \cdot \Psi(\vec{r}, t)$$
   (la densité de probabilité $\frac{dP(\vec{r}, t)}{dV}$ est donc égale au carré module de la fonction d'onde $\Psi(\vec{r}, t)$)

   \item La fonction d'onde est normée :
      $$
      \begin{aligned}
         P(t) &= \int_x \int_y \int_z dP(\vec{r}, t)\\
         &=\int_x \int_y \int_z \Psi^*(\vec{r}, t) \cdot \Psi(\vec{r}, t)\\
         &=\int_x \int_y \int_z |\Psi(\vec{r}, t)|^2\\
         &= 1
      \end{aligned}
      $$
      La fonction d'onde est donc normée, la probabilité de trouver la particule dans l'espace est donc de 1.\\
      Attention, on intègre dans toutes les directions de l'espace, autant de fois qu'il y a de particules.\\
      

   \item La fonction d'onde est continue, dérivable et de carré sommable (l'intégrale de son carré converge pour pouvoir valoir 1).\\
      Sa dérivée première est aussi continue et dérivable. (\emph{elle doit l'être}).

   \item \emph{Notation de Dirac :} 
      \begin{itemize} 
         \item à $\Psi(\vec{r}, t)$, on associe un "ket", noté $|\Psi(t)>$
         \item à $\Psi^*(\vec{r}, t)$, on associe un "bra", noté $<\Psi(t)|$
      \end{itemize}

      On a donc $P(t) = <\Psi(t) | \Psi(t) > = 1$


   \item On définit le produit scalaire entre deux fonctions d'onde par :
      $$\int_x \int_y \int_z \Psi^*(\vec{r}, t) \cdot \Phi(\vec{r}, t) dV = <\Psi(t) | \Phi(t)>$$
      \begin{ex} 
         $\phantom{a}$
         \begin{itemize} 
            \item (fig1) $\Psi^*(x) \cdot \Psi(x) = |\Psi(x)|^2$
               Il s'agit de la probabilité de trouver la particule en $x$ (probabilité infinitésimale).\\
               $|\Psi(x)|^2 dx$ est alors la probabilité de trouver la particule sur le segment de longueur $dx$
            
            \item (fig2) $\int_a^b dP(x) = P$\\
               $\int_a^b |\Psi(x)|^2$ est alors la probabilité de trouver la particule sur le segment $[a,b]$
         \end{itemize}
      \end{ex}
\end{itemize}

\paragraph{Postulat 2}

\begin{itemize}

   \item Les opérateurs agissent sur les fonctions d'onde : $\hat{A} \Psi(\vec{r}, t) = \Phi(\vec{r}, t)$.\\
  \begin{ex} 
     Soit $\hat{A} = \frac{\partial}{\partial x}$ et $\Psi(x) = exp(-2x^2)$\\
     On a alors $ \hat{A} \Psi(x) = \frac{\partial exp(-2x^2)}{\partial x} = -4x \Psi(x)$
  \end{ex}

  \item \emph{Cas particulier :} si $\hat{A} \Psi(\vec{r}, t) = \lambda \Psi(\vec{r}, t)$ (où $\lambda$ est un scalaire),\\
     on dit que : 
   
      \begin{itemize} 
               \item $\lambda$ est une valeur propre de $\hat{A}$
               \item $\Psi(\vec{r}, t)$ est vecteur propre de $\hat{A}$ associé à la valeur propre $\lambda$
               \item On parle alors d'une équation aux valeurs propres.\\
                  Dans une équation aux valeurs propres, on a toujours un couple $(\lambda, \Psi(\vec{r}, t))$.\\
                  L'équation de Schrödinger est une équation aux valeurs propres.
      \end{itemize}

   \item \emph{Notion de dégénérescence :} deux fonctions propres différentes donnent, pour un même opérateur donné, la même valeur propre, on dit alors qu'elles sont dégénérée.
      \begin{ex} 
            $sin$ et $cos$ sont des fonctions propres dégénérées de la dérivée seconde:
            $$\frac{\partial^2}{\partial x^2}(cos(kx)) = -k^2 cos(kx)$$
            $$\frac{\partial^2}{\partial x^2}(sin(kx)) = -k^2 sin(kx)$$
            La valeur propre commune est alors $-k^2$

      \end{ex}

   \item \emph{Algèbre des opérateurs :}

      \begin{itemize} 
            \item Somme $\hat{S} = \hat{A} + \hat{B}$\\
               $$\hat{S} \Psi(\vec{r}, t) = (\hat{A} + \hat{B}) \Psi(\vec{r}, t) = \hat{A} \Psi{\vec{r}, t} + \hat{B} \Psi(\vec{r}, t)$$
               \begin{ex} 
                  $\hat{A} = x$, $\hat{B} = -ih\frac{\partial}{\partial x}$, $\Psi(\vec{r}, t) = e^{-2x}$ \\
                  $$
                    \begin{aligned} 
                       \hat{S} \Psi(\vec{r}, t) &= (x - ih\frac{\partial}{\partial x}) \Psi(\vec{r}, t)\\
                       &= x \Psi(\vec{r}, t) - ih\frac{\partial}{\partial x} (\Psi(\vec{r}, t)\\
                       &= xe^{-2x}-ih\frac{\partial}{\partial x} (e^{-2x})
                       &= (x+2i\hbar)e^{-2x}
                    \end{aligned}
                  $$
               \end{ex}

            \item Produit $\hat{P} = \hat{A} \cdot \hat{B}$ (Il s'agit de la composition dans l'espace des fonctions.)\\
               Attention, le produit d'opérateurs n'est pas commutatif.\\
               $$\hat{P} \Psi(x) = (\hat{A} \cdot \hat{B}) \Psi(x) = \hat{A} (\hat{B} \Psi(x))$$

               Le produit n'étant pas commutatif, on en profite pour introduire le commutateur :
               $[\hat{A}, \hat{B}] = \hat{A} \hat{B} - \hat{B} \hat{A}$\\
               Si ce commutateur est nul, alors les deux opérateurs commutent, dans le cas contraire, ils ne commutent pas.\\
              \begin{rema} 
                  Deux opérateurs qui commutent admettent un même ensemble de vecteurs propres.\\
              \end{rema}

            \item Linéaire : un opérateur $\hat{A}$ est linéaire si : 
               $$\hat{A}(\alpha \Psi(\vec{r}, t) + \beta \Phi(\vec{r}, t)) = \alpha \hat{A} \Psi(\vec{r}, t) + \beta \hat{A} \Phi(\vec{r}, t)$$
               Toute combinaison linéaire de fonctions propres dégénérées d'un opérateur est également fonction propre de cet opérateur avec la même valeur propre (qui est alors un scalaire).\\
               On retrouve cette propriété dans la théorie des orbitales atomiques hybrides.\\

      \end{itemize}
      
   \item \emph{Hermiticité :} Si $\hat{A}$ est hermitique, alors ses valeurs propres sont réelles et ses vecteurs propres sont orthogonaux.\\

   \item En mécanique quantique, tout opérateur peut être construit à partir des opérateurs position et quantité de mouvement, c'est le \emph{principe de correspondance} :
      \begin{itemize} 
         \item \emph{Opérateur position :} associé à la coordonnée $q_i$ (où $q_i$ peut correspondre à $x$, $y$ ou $z$) d'une particule, il consiste à multiplier par la variable $q_i$.\\
            $$\hat{q_i} = q_i \cdot $$

            Pour l'opérateur $\hat{x}$, on a :
            $$\hat{x} = x \cdot$$
         \item \emph{Opérateur quantité de mouvement :} associé à la coordonnée $q_i$, il s'exprime $p_i = mv_i$; il consiste à dériver par rapport à cette coordonnée et à multiplier par $-i\hbar$ :\\
            $$\hat{p_i} = -i\hbar \frac{\partial}{\partial q_i}$$
            Dans le cas tridimensionnel, on a :
            $$\hat{p} = -i\hbar \nabla = -i \hbar (\frac{\partial}{\partial x} + \frac{\partial}{\partial y} + \frac{\partial}{\partial z}) $$

            \begin{ex}[pour une onde de matière mono-dimensionnelle]
               $\phantom{a}$\\
               $\Psi(x, t) =\Psi_0 e^{i(kx - \omega t)}$
               \begin{itemize} 
                        \item \emph{opérateur quantité de mouvement associé à $\Psi(x, t)$:} \\
                           $\hat{p_x} \Psi(x, t) = \frac{\partial \Psi(x, t)}{\partial x} (-i \hbar)$\\
                           D'où :
                           $$
                             \begin{aligned} 
                                 \hat{p_x} \Psi(x, t) 
                                 &= -i \hbar i k \Psi_0 e^{i(kx- \omega t)}\\
                                 &= \hbar k \Psi_0 e^{i(kx- \omega t)}\\
                                 &= k \hbar \Psi(x, t)
                             \end{aligned}
                           $$
                           $\hbar k$ est alors la valeur propre associée à la fonction d'onde, elle correspond à la quantité de mouvement.

                     \item \emph{opérateur énergie cinétique :} \\
                        $$E_c = \frac{1}{2} mv^2 = \frac{p^2}{2m} = \frac{(\hbar k)^2}{2m}$$
                        $$\hat{T} \Psi(x, t) = \frac{(\hbar k)^2}{2m} \Psi(x, t)$$
                        (l'opérateur énergie cinétique est noté $\hat{T}$)
                        $$\hat{T} = -\frac{\hbar^2}{2m} \frac{d^2}{dx^2}$$ 
                        Car $-\frac{\hbar^2}{2m} \frac{d^2}{dx^2}\Psi_0 e^{i(kx - \omega t)} = -\frac{\hbar^2}{2m}i^2\hbar^2 \Psi_0 e^{i(kx - \omega t)} = \frac{(\hbar k)^2}{2m}$
               \end{itemize}
            \end{ex}
         \item \emph{opérateur énergie cinétique et laplacien :}\\
            Dans le cas d'une onde : $\Psi(x, t) = \Psi_0 e^i{i(xk-\omega t)}$\\

            Dans le cas tridimensionnel :
            $\hat{T} = \frac{-\hbar^2}{2m} \Delta = \frac{-\hbar^2}{2m} (\frac{\partial^2}{\partial x^2} + \frac{\partial^2}{\partial y^2} + \frac{\partial^2}{\partial z^2})$\\

            Expression de l'opérateur énergie d'une particule de masse $m$ qui se déplace dans les trois directions de l'espace et d'énergie potentielle $V(\vec{r})$ :
            $\hat{H} = \hat{T} + \hat{V} = \frac{-\hbar^2}{2m} \Delta + \hat{V(\vec{r})}$\\

            $\hat{H}$ est appelé opérateur Hamiltonien.\\
      \end{itemize}
\end{itemize}

\paragraph{Postulat 3 :} \emph{L'équation de Schrödinger}\\

$$\hat{H} \Psi(\vec{r}, t) = i \hbar \frac{\partial \Psi(\vec{r}, t)}{\partial t}$$

   Lorsque le potentiel agissant sur le système $V(\vec{r})$ ne dépend pas du temps, la fonction d'onde qui décrit le système ne dépend pas du temps.\\
   Le système est alors dans un \emph{état stationnaire}.\\

   La fonction d'onde est alors obtenue par la résolution de l'équation de Schrödinger indépendante du temps :
   $$\hat{H} \Psi(\vec{r}) = E \Psi(\vec{r})$$

   Les états stationnaires d'un système sont aussi appelés \emph{états propres}, ils sont en effet les vecteurs propres associés à l'opérateur hamiltonien.\\

  \begin{rema} 
   Dès qu'un système quantique est soumis à une perturbation dépendante du temps, il n'est plus dans un état propre. Il se retrouve dans un état superposé qui est une combinaison linéaire des états propres.\\
  \end{rema}

\paragraph{Postulat 4 :} \emph{mesures physiques :}\\

   Soit une grandeur physique $A$ associée à l'opérateur $\hat{A}$, tel que :\\
   $$\hat{A} \Psi_n(\vec{r}) = a_n \Psi_n(\vec{r})$$

  \begin{itemize} 
         \item Alors seules les valeurs propres $a_n$ peuvent être mesurées pour la propriété $A$.\\
  \end{itemize}

   Soit un système décrit par $\Psi(\vec{r})$, une fonction d'onde quelconque :

  \begin{itemize} 
     \item si $\Psi(\vec{r})$ est une fonction propre de $\hat{A}$ associé à la valeur propre $\lambda$, alors toute mesure de la propriété $A$ conduira à $\lambda$.\\

     \item si $\Psi(\vec{r})$ n'est pas une fonction propre de $\hat{A}$, alors la valeur moyenne d'une série de mesures expérimentales sur la grandeur physique sera :
        $$<A> = \frac{\iiint_V \Psi^*(\vec{r}) \hat{A} \Psi(\vec{r}) dV}{\iiint_V \Psi^*(\vec{r})\Psi(\vec{r}) dV} = \frac{<\Psi | \hat{A} | \Psi >}{<\Psi | \Psi >}$$
  \end{itemize}

  Ayant définit la valeur moyenne d'une mesure, nous pouvons également définir la dispersion des résultats (l'écart type):
  $$\Delta A = \sqrt{<A^2> - <A>^2}$$

 \begin{ex} 
      Avec $\hat{A}\Psi(x)$, \\
      $$
    \begin{aligned}
       <A> &= \int_x \Psi^*(x) \hat{A} \Psi(x) dx\\
       &= \lambda \int_x \Psi^*(x) \Psi(x)dx\\
       &= \lambda
    \end{aligned}
      $$

      $$
     \begin{aligned} 
         <A^2> &= \int_x \Psi^*(x) \hat{A}^2 \Psi(x)dx \\
         &= \int_x \Psi^*(x) \lambda^2 \Psi(x)dx\\
         &= \lambda \int_x \Psi^*(x) \lambda \Psi(x)dx\\
         &= \lambda^2
     \end{aligned}
      $$

      $$\Delta A = \sqrt{\lambda^2 - \lambda^2} = 0$$

      La moyenne est donc égale à $\lambda$ et la dispertion est égale à $0$
 \end{ex}
 
 \subsection{Applications des postulats}
 \subsubsection{Application 1: particule dans une boîte 1D}

 %fig3

\begin{rap} 
      $$\sin a \sin b = \frac{\cos (a-b) - \cos (a+b)}{2}$$
\end{rap}

Ensemble des vecteurs et valeurs propres de la fonction d'onde :\\
$$\hat{H} \Psi(x) = E \Psi (x)$$

$\hat{H} = \hat{T} + \hat{V}$\\
où $\hat{T} = \frac{- \hbar^2}{2m} \Delta = \frac{-\hbar^2}{2m} \frac{\partial^2}{\partial^2 x}$,\\
et $\hat{V} = 0$, car dans ce cas, la particule n'a pas de potentiel.\\

Donc $\hat{H} = \frac{-\hbar^2}{2m} \frac{d^2}{d x^2}$\\
D'où, 
$$
\begin{aligned}
      \hat{H} \Psi(x) &= E \Psi(x)\\
      \Rightarrow \frac{-\hbar^2}{2m} \frac{d^2}{d x^2} \Psi(x) &= E \Psi(x)\\
      \Rightarrow \frac{-\hbar^2}{2m} \frac{d^2}{d x^2} \Psi(x) - E \Psi(x) &= 0\\
      \Rightarrow \frac{d^2 \Psi(x)}{d x^2} + \frac{2mE}{\hbar^2} \Psi(x) = 0
\end{aligned}
$$

En posant $\frac{2mE}{\hbar^2} = k^2$, on a une équation différentielle du second ordre à coefficient constant homogène de la forme:
      $$\frac{d^2 \Psi(x)}{d x^2} + k^2 \Psi(x) = 0$$

      Dont la solution homogène est (forme d'une solution homogène d'oscillateur harmonique):
      $\Psi(x) = A\sin{kx} + B\cos(kx)$\\

      On détermine ensuite $A$ et $B$:\\
      $\Psi$ étant continue, dérivable et de carré sommable, elle est nulle en ses limites $0$ et $L$.\\
      en $x=0$:
      $$
     \begin{aligned} 
        \Psi(0) = 0 &\Leftrightarrow A \sin(k \cdot 0) + B \cos(k \cdot 0) = 0\\
        &\Rightarrow B = 0
     \end{aligned}
      $$

      Donc $\Psi(x) = A\sin(kx)$\\
      en $x=L$:
      $$
\begin{aligned}
   \Psi(L) = 0 &\Leftrightarrow A\sin(kL) = 0\\
   &\Rightarrow \sin(kL) = 0 \text{ ou } A = 0\\
   \text{A ne pouvant être nul car la particule existe}\\
   &\Rightarrow kL = n\pi \text{ avec } n \in \ZZ^*\\
   &\Leftrightarrow k = \frac{n\pi}{L}\\
\end{aligned}
      $$

      On a donc $k^2 = \frac{n^2\pi^2}{L^2} = \frac{2mE}{\hbar^2}$\\

      D'où $E_n = \frac{\hbar^2 n^2 \pi^2}{L^2 2m}$ avec $n \in \ZZ^*$\\
      L'énergie est quantifiée car $n$ ne peut prendre que des valeurs entières donc discrètes.\\

      Donc $\Psi_n(x) = A\sin(k_n x)$\\

      Avec la condition de normalisation, on détermine $A$, \\
      $$\int_x \Psi_n^*(x) \Psi_n(x) dx = 1$$

      Avec $\Psi_n(x) = A \sin(k_n x)$ et ($\Psi$ étant réelle), $\Psi_n^*(x) = \Psi_n(x) = A \sin(k_n x)$\\
      On a :
      $$
\begin{aligned}
   \int_x \Psi_n^2(x)dx &= A^2 \int_x \sin^2(kx)dx\\
   &= A^2 \int_0^L sin(kx) sin(kx) dx\\
   &= A^2 \int_0^L \frac{1}{2} (cos(kx - kx) - cos(kx + kx)) dx\\
   &= A^2 \int_0^L \frac{1}{2} - cos(2kx) dx\\
   &= A^2 (\int_0^L \frac{1}{2}dx - \int_0^L cos(2kx) dx)\\
   &= A^2 (\frac{L}{2} - [\frac{sin(2k_nx)}{4k_n}]_0^L)\\
   &= A^2 (\frac{L}{2} - [\frac{sin(2\frac{n\pi}{L}x)}{4k_n}]_0^L)\\
   & 2\frac{n\pi}{L}x \text{ étant toujours nulle, son sinus le sera aussi}\\
   &= A^2(\frac{L}{2} - 0 - 0) = 1
\end{aligned}
      $$

      D'où $A^2 = \frac{2}{L}$ et $A = \pm \sqrt{\frac{2}{L}}$\\
      Le $\pm$ correspond au fait que les orbitales soient définis au signe près.\\

     \begin{rema} 
         $n$ ne peut être nul car la particule ne peut être à l'arrêt.
     \end{rema}

     \emph{Exercice :} Montrer que l'inégalité d'Heisenberg est vérifiée : $\Delta p \cdot \Delta q \geq \frac{\hbar}{2}$


     \subsubsection{Application 2 : particule dans une boîte 2D}
     %fig4 (page 12)

     On cherche l'ensemble des vecteurs et valeurs propres de la fonction d'onde $\hat{H} \Psi(x, y) = E \Psi(x, y)$.\\

     $$
\begin{aligned}
     & \hat{H} \Psi(x, y) = E \Psi(x, y)\\
     &\Rightarrow \frac{-\hbar^2}{2m} \Delta \Psi(x, y) = E \Psi(x, y)\\
     &\Rightarrow \frac{-\hbar^2}{2m} (\frac{\partial^2}{\partial x^2} + \frac{\partial^2}{\partial y^2}) \Psi(x, y) = E \Psi(x, y)\\
     & \Rightarrow -\frac{\hbar^2}{2m}\frac{\partial^2}{\partial x^2} \Psi(x, y) - \frac{\hbar^2}{2m}\frac{\partial^2}{\partial y^2}\Psi(x, y) = E \Psi(x, y)
\end{aligned}
     $$


     \emph{Méthode de séparation des variables :}\\
     Soit $\hat{A_{a,b}}$ (où $a$ et $b$ sont les variables) qui s'écrit comme la somme de deux opérateurs,
     chacun s'appliquant sur une variable telle que 
     $\hat{A_{a,b}} = \hat{h_a} + \hat{h_b}$ 
     alors le produit des vecteurs propres de $\hat{h_a}$ et $\hat{h_b}$ est vecteur propre de $\hat{A_{a,b}}$.\\

     La somme des valeurs propres de $\hat{h_a}$ et $\hat{h_b}$ est valeur propre de $\hat{A_{a,b}}$.\\

     On a alors:\\
     $$\hat{H} \Psi(x) + \hat{H} \Psi(y) = E \Psi(x, y)$$
     avec $\hat{A} = \hat{h_a} + \hat{h_b}$ et $\hat{h_a} \Psi(a) = \alpha \Psi(a)$, $\hat{h_b} \Phi(b) = \beta \Phi(b)$\\

     On a ainsi:
     $$
\begin{aligned}
      \hat{A} \Psi(a) \Phi(b) &= (\hat{h_a} + \hat{h_b}) \Psi(a) \Phi(b)\\
      &= \hat{h_a} \Psi(a) \Phi(b) + \hat{h_b} \Psi(a) \Phi(b)\\
      &= \Phi(b) \hat{h_a} \Psi(a) + \Psi(a) \hat{h_b} \Phi(b)\\
      &= \Phi(b) \alpha \Psi(a) + \Psi(a) \beta \Phi_(b)\\
      &= (\alpha + \beta) \Psi(a) \Phi(b)\\
      &= \hat{A} \Psi(a) \Phi(b)
\end{aligned}
     $$

     Les solutions $\Psi_{n_1, n_2}(x, y) = \pm \sqrt{\frac{2}{Lx}} \sqrt{\frac{2}{Ly}} \sin(k_{n_1}x) \sin(k_{n_2}y)$
     car on se ramène à deux problèmes en 1D et on les multiplies, ainsi :
     $$E_{n_1, n_2} = \frac{\pi^2 \hbar^2}{2m} (\frac{n_1^2}{L^2 x} + \frac{n_2^2}{L^2 y})$$


     De manière analogue, en 3D, on a :
     $$\Psi_{n_1, n_2, n_3}(x, y, z) = \pm \sqrt{\frac{2}{Lx}} \sqrt{\frac{2}{Ly}} \sqrt{\frac{2}{Lz}}\sin(k_{n_1}x) \sin(k_{n_2}y) \sin(k_{n_3} z)$$
   
   \subsubsection{Application 3 : particule sur une sphère}

   Soit une particule de masse $\mu$ évoluant à une distance $r$ constante de l'origine du repère.\\

   Le problème présente une symétrie sphérique, nous allons donc utiliser le système de coordonnées sphérique.\\

  \begin{rema} 
      \it{cf} cahier de TD, relations entre coordonnées cartésiennes et sphériques.\\

      Élément de volume et couverture de l'espace :
      $$dV = r^2dr \cdot \sin(\theta) d\theta \cdot d\rho$$

      Condition de normalisation :
      $$<\Psi | \Psi> = \int_0^\infty \int_0^\pi \int_0^{2\pi} \Psi^*(r, \theta, \rho) \Psi(r, \theta, \rho) r^2 \sin\theta dr d\theta d\rho$$
  \end{rema}

  \emph{Hamiltonien du système :}\\
  Seul l'opérateur énergie cinétique est pris en compte : $\hat{H} = \frac{-\hbar^2}{2\mu}\Delta = \frac{-\hbar^2}{2\mu} (\frac{\partial^2}{\partial x^2} + \frac{\partial^2}{\partial y^2} + \frac{\partial^2}{\partial z^2})$

  En coordonnées sphériques :
  $$\Delta = \frac{1}{r^2} (\frac{\partial}{\partial r} r^2 \frac{\partial}{\partial r} + \frac{1}{\sin \theta} \frac{\partial}{\partial \theta} \sin \theta \frac{\partial}{\partial \theta} + \frac{1}{\sin^2\theta}\frac{\partial^2}{\partial \rho^2})$$

  Où encore en utilisant l'opérateur de Legendre :
  $$\Delta = \frac{1}{r^2}(\frac{\partial}{\partial r} r^2 \frac{\partial}{\partial r} + \Lambda)$$

  Donc $\hat{H} = \frac{-\hbar^2}{2\mu r^2} \Lambda$ (étant donné $r$ constant).\\



  \emph{Résolution de l'équation aux valeurs propres :}\\

  Les fonctions propres de $\Lambda$ sont les harmoniques sphériques $Y_{l,m}(\theta, \rho)$ associées aux valeurs propres $-l(l+1)$.\\

  $$
\begin{aligned}
   \Lambda Y_{l,m}(\theta, \rho) &= -l(l+1) Y_{l,m}(\theta, \rho)\\
   \hat{H} Y_{l,m}(\theta, \rho) &= \frac{\hbar^2}{2\mu r^2}l(l+1) Y_{l,m}(\theta, \rho)
\end{aligned}
  $$

  \emph{L'expression générale des harmoniques sphériques :}\\

  $$Y_{l,m}(\theta, \rho) = (\frac{(2l+1)(l-|m|)!}{4\pi(l-|m|)!})^{1\over 2} e^{i\cdot m \cdot \rho}\cdot P_l^m(\cos \theta)$$
  où $P_l^m (\cos \theta)$ est le polynôme de Legendre de degré $l$ avec $l$ et $m$ entiers tels que $l \in \NN$ et $m \in \llbracket -l, l \rrbracket$

 \begin{rema} 
      \it{cf} tableau des harmoniques sphériques.
 \end{rema}

 \emph{Exercice :} vérifier la normalisation des harmoniques sphériques du tableau et l'orthogonalité des vecteurs propres.\\


 L'énergie du système est quantifiée par le nombre quantique azimutal ($l$).\\

   \`A chaque valeur de $l$ correspondent $2l+1$ fonctions propres différentes (dégénérescence d'ordre $2l+1$) car elles ont le même nombre quantique azimutal donc la même énergie.\\

   \emph{Énergie de rotation d'un corps de masse $\mu$ sur une sphère de rayon $r$ :}

   En classique:
   $$E_c = \frac{L^2}{2\mu r^2}$$
   où $L$ est le moment cinétique tel que $L^2 = l(l+1)\hbar^2$\\


   En quantique:
   $$E_c = \frac{\hbar^2}{2\mu r^2} l(l+1)$$

   \emph{Opérateur du moment cinétique au carré en coordonnées sphériques :}
   $$L^2 = -\hbar^2 \Lambda$$

   Donc $\hat{L_z} Y_{l,m} (\theta, \rho) = m \hbar Y_{l,m} (\theta, \rho)$\\

   La projection du moment cinétique selon $z$ est quantifiée par $m$.\\
   Pour une valeur de $l$ donnée, on a $2l+1$ valeurs de $m$.\\

   Exemple pour $l=1$:
   %fig5 (page 16 (spin))


\section{Les atomes hydrogénoïdes}

%fig6 (page 16 (atome))

\emph{Énergie du système :} $E = T_e + T_n + V_{e,n}$ (avec $T_e$, l'énergie cinétique de l'électron, $T_n$, celle du noyau, $V_{e,n}$, l'énergie des interactions entre les deux).\\

\emph{Expressions classique :}\\
$$T_e = \frac{p_e^2}{2m_e}$$
$$T_n = \frac{p_n^2}{2m_n}$$
$$V_{e,n} = -\frac{1}{4\pi\epsilon_0} \frac{Z_e^2}{r}$$

\emph{Opérateurs quantiques :}\\
$$\hat{T_e} = \frac{-\hbar^2}{2m_e} \Delta_e$$
$$\hat{T_n} = \frac{-\hbar^2}{2m_n} \Delta_n$$
$$\hat{V_{e,n}} = -\frac{1}{4\pi\epsilon_0} \frac{Z_e^2}{\hat{r}}$$


   L'équation dépend du mouvement des deux particules et est donc insolvable.\\
   On dispose alors de deux approches pour simplifier les équations:\\

   On décompose le mouvement en deux :
  \begin{itemize} 
         \item un mouvement de translation de l'atome
         \item un mouvement relatif de l'électron et du noyau.
  \end{itemize}

  Introduction d'une particule fictive de masse réduite $\mu$:
  $$\hat{H} = \hat{T_\mu} + \hat{V_{e,n}} = -\frac{\hbar^2}{2\mu} - \frac{1}{4\pi\epsilon_0} \frac{Z_e^2}{\hat{r}}$$
  avec $\mu = \frac{m_em_n}{m_e + m_n}$

  Approximation de Born-Oppenheimer : \\
  Du fait de leur large différence de masse, les électrons s'adaptent de façon instantanée et adiabatique a tout mouvement du noyau :\\
 \begin{itemize} 
         \item On considère donc le noyau fixe
         \item On ne considère plus qu'un hamiltonien électronique :
            $$\hat{H}= \frac{-\hbar^2}{2m_e} - \frac{1}{4\pi\epsilon_0}\frac{Z_e^2}{\hat{r}}$$
 \end{itemize}

 Les $\Psi(x,y,z)$ sont les fonctions propres (les états propres).\\
 On les appelles les orbitales atomiques.\\

 \emph{Exercice :} Montrer que $\hat{H}$ et $\Lambda$ commutent.\\


 Étant donné que les deux opérateurs commutent, ils admettent un même jeu de vecteurs propres.\\
 Les fonctions propres de $\Lambda$ sont les harmoniques sphériques $Y_{l,m}(\theta, \rho)$\\
 $$\Lambda \alpha Y_{l,m}(\theta, \rho) = -l(l+1) \alpha Y_{l,m}(\theta, \rho)$$
 où $\alpha$ est une constante réelle.\\

 D'où $\Psi(r, \theta, \rho) = R(r)Y_{l,m}(\theta, \rho)$ ($R(r)$ étant constante pour Legendre)\\

   On a $\hat{H}\Psi = E \Psi$ et $\Psi = R(r)Y(\theta, \rho)$\\

   Donc:
   $$
  \begin{aligned} 
     &\frac{-\hbar^2}{2m_er^2} (\frac{\partial}{\partial r} r^2 \frac{\partial}{\partial r} + \Lambda) R(r) Y_{l,m}(\theta, \rho) \frac{-Z_e^2}{4\pi\epsilon_0} \frac{1}{r} R(r) Y_{l,m}(\theta, \rho) = E R(r) Y_{l,m}(\theta, \rho)\\
     &\Rightarrow \frac{-\hbar^2}{2m_er^2} (Y_{l,m}(\theta, \rho) \frac{\partial}{\partial r} r^2 \frac{\partial}{\partial r} R(r) - R(r) l(l+1) Y_{l,m}(\theta, \rho) - \frac{S_e^2}{4\pi\epsilon_0} \frac{1}{r}R(r) Y_{l,m}(\theta, \rho)) =  E R(r) Y_{l,m}(\theta, \rho)\\
     &\Rightarrow \frac{-\hbar^2}{2m_er^2} (\frac{\partial}{\partial r} r^2 \frac{\partial}{\partial r} - l(l+1)) R(r) - \frac{Z_e^2}{4\pi\epsilon_0} \frac{1}{r} R(r) = E R(r)
  \end{aligned}$$

  On obtient alors une expression de la partie radiale de l'équation de Schrödinger.\\

  $R(r)$ a de fortes chances de dépendre de $l$ et $E$ esst liée à la partie radiale.\\


  \emph{Fonctions propres de l'équation radiale}.\\

  Les fonctions propres de la partie radiale sont notées $R_{n,l}(r)$ (\it{cf} tableau des expressions des $R_{n,l}(r)$).\\

  \emph{Exercice :} montrer que les parties radiales présentes dans le tableau sont normalisées.\\


  \emph{Valeurs propres de l'équation radiale}\\

  Pour les hydrogénoïdes, on a :
  $$E_n = -\frac{m_ee^4}{(4\pi\epsilon_0)^2\hbar^2} \frac{Z^2}{2n^2}$$

  L'énergie est donc quantifiée et ne dépend que de $n$.\\

  À l'état fondamental $n=1$, on a :
  $$E_1 = -\frac{Z^2}{2}\frac{m_ee^4}{(4\pi\epsilon_0)^2\hbar^2}$$

  Expression de l'énergie de première ionisation $E_{1|}$:
  $$E_{1|} = E_\infty - E_1 = -E_1$$

  On a donc les expression de l'énergie suivante :
  $$E_n = \frac{-E_{i_1}}{n^2} = \frac{E_1}{n^2} = \frac{E_1(H)Z^2}{n^2}$$


  On introduit la constante de Rydberg $R_H$
  $$R_H = \frac{m_ee^4}{8\epsilon_0^2\hbar^3c}$$
  $R_H$ est en $m^{-1}$\\


  Ce qui permet de donner une nouvelle expression de l'énergie à l'état $n$:
  $$E_n = \frac{-\hbar c R_H}{n^2}$$
  $E_n$ est en $J$\\
  $1eV = 1,60 \cdot 10^{-19} J$\\


  \emph{Formule de Rydberg :}
  $$\frac{1}{\lambda} = R_H Z^2 (\frac{1}{n_1^2} - \frac{1}{n_2^2})$$


  \emph{Pour les hydrogénoïdes et les états excités :}\\
  $$E_n = \frac{-13,6}{n^2}Z^2$$
  (donne une énergie en électron volt)\\


\begin{rap}[Système d'unité atomique (u.a.)]
   $e=1$\\
   $\hbar = 1$\\
   $4\pi\epsilon_0 = 1$\\
   $c = 1$\\
   $m_e = 1$\\
   $a_0 = 1$\\

   Les énergies s'écrivent donc sous la forme $E_1(H) = \frac{-1}{2} u.a.$\\

   D'où : $E_n(Z) = - \frac{Z^2}{2n^2} u.a.$

   L'unité atomique d'énergie est le Hartree.\\

   Notation : $\hat{H} = - \frac{1}{2} \Delta - \frac{Z}{r}$
\end{rap}

\subsection{Orbitale atomique : bilan}

   La forme générale des orbitales atomiques (fonctions propres) est $\Psi_{n,l,m}(r, \theta, \rho) = R_{n,l}(r)Y_{l,m}(\theta, \rho)$.\\

   Les parties $R(r)$ et $Y(\theta, \rho)$ sont normalisées.\\

   $l$ et $m$ caractérisent le moment cinétique de l'électron autour du noyau.\\

  \begin{itemize} 
         \item un électron dans une OA de nombre quantique $l$ a un moment cinétique de norme $L=\sqrt{l(l+1)}\hbar$.
         \item un électron dans une OA de nombre quantique $m$ a une composante selon $z$ du moment cinétique égale à $m\hbar$.\\
  \end{itemize}

 \begin{itemize} 
         \item $l = 0 \rightarrow $ sous-couche s
         \item $l = 1 \rightarrow $ sous-couche p
         \item $l = 2 \rightarrow $ sous-couche d
         \item $l = 3 \rightarrow $ sous-couche f
         \item $l = 4 \rightarrow $ sous-couche g
         \item $l = 5 \rightarrow $ sous-couche h
 \end{itemize}


 Les OA ayant les mêmes nombres quantiques $n$ et $l$ constituent des sous-couches.\\

 $m$ peut prendre $2l+1$ valeurs pour une sous-couche $nl$ donnée, il y a alors $2l + 1$ OA dans une sous-couche.

\begin{itemize}
         \item $l = 0 \rightarrow $ une OA
         \item $l = 1 \rightarrow $ trois OA
         \item $l = 2 \rightarrow $ cinq OA
\end{itemize}


\emph{Forme des OA : partie angulaire}\\
Pour tous le $l \neq 0$, les harmoniques sphériques sont complexes, \\
On veut en extraire des fonctions réelles (on utilise alors les formules d'Euler):
\begin{rap}[Formules d'Euler]
      $$\frac{e^{i\rho}+e^{-i\rho}}{2}=\cos{\rho}$$
      $$\frac{e^{i\rho}-e^{-i\rho}}{2i}=\sin{\rho}$$
\end{rap}

\emph{fonctions réelles :}
   $$S_{l, |m|}^+ = \frac{Y_{l,m} + Y_{l,-m}}{\sqrt{2}}$$
   $$S_{l, |m|}^- = \frac{Y_{l,m} - Y_{l,-m}}{i\sqrt{2}}$$

  \begin{ex} 
      $$
     \begin{aligned}
        \frac{Y_{1,1} + Y_{1,-1}}{2} &= \frac{\frac{1}{2}(\frac{3}{2\pi})^{\frac{1}{2}}\sin \theta e^{i\rho}+ e^{-i\rho}}{2}\\
        &= \frac{1}{2} (\frac{3}{2\pi})^{\frac{1}{2}}\sin \theta \cos \rho
     \end{aligned}
      $$
     \begin{rema} 
         Attention, cette fonction n'est pas normalisée, d'où le $\sqrt{2}$ dans $S^+$ et $S^-$.
     \end{rema}
  \end{ex}

  Toutes les OA de type $s$ sont à symétrie sphérique.\\

  Les trois harmoniques sphériques de type $l=1$ sont équivalentes par rotation de $90°$.
  C'est-à-dire que les OA correspondantes ont la même forme.\\
  Seule leur orientation est différente.\\
  On les note $p_x, p_y, p_z$.\\
  Chacune des fonctions présente une surface nodale.\\

  Chacune des harmoniques sphériques de type $l=2$ présente deux surfaces nodales.\\

  De manière plus générale, les OA présentent $l$ surfaces nodales.\\


\emph{Forme des OA : partie radiale}\\

  \begin{itemize} 
     \item $1s \rightarrow $ exponentielle décroissante %fig7 (p22, 1s)
     \item $2s \rightarrow $ s'annule puis croissance en $-e^x$ %fig8 (p22, 2s)
     \item $3s \rightarrow $ s'annule deux fois puis décroissance exponentielle %fig9 (p22, 3s)
     \item $2p \rightarrow $ s'annule en $0$, croit puis décroit exponentiellement %fig10 (p22, 2p)
     \item $3p \rightarrow $ s'annule en 0, est positive, s'annule une seconde fois, décroit puis croit en $-e^x$%fig11 (p22, 3p)
  \end{itemize}

 \begin{rema} 
      Chacune des fonctions présente $n-2$ points nodaux.
 \end{rema}


   Étant donné que pour représenter une fonction à trois variables, il est nécessaire d'utiliser quatre dimensions, un ne représente que les isosurfaces des OA, c'est-à-dire des surfaces sur lesquelles l'OA en question a la même valeur.\\


   \emph{Exercice :} représenter des isosurfaces des OA $3p$, $5p$, $4d_{x,y}$ et $5d_{z^2}$


   \emph{Condition de normalisation en coordonnées sphériques :}\\

   En rappelant $dV = r^2 \sin \theta dr d\theta d\rho$
   $$
\begin{aligned}
      \iiint_V R(r) Y^*(\theta, \rho) R(r) Y(\theta, \rho) dV&= \iiint_V R^2(r) |Y(\theta, \rho)|^2 dV\\
      &= \int_r R^2(r) r^2dr \cdot \int_\theta \int_\rho |Y(\theta, \rho)|^2 \sin \theta d\theta d\rho\\
      &= 1 \cdot 1\\
      &=1
\end{aligned}
   $$


   $dP(r) = R^2(r)r^2 dr$ : Probilité infinitésimale.\\

   $P(r) = \frac{dP(r)}{dr} = R^2(r) r^2$ : Densité de probabilité radiale.\\

  \begin{ex} 
      $R_{1,0}(r)$\\

      $R_{1,0}^2(r) r^2 dr = \frac{dP_{1,0}(r)}{dr}$
      %fig12(p23)
  \end{ex}


  \emph{Distance moyenne électron-noyau : }\\
  $<r> = \int_r R(r) r R(r) r^2 dr$\\
  $\Leftrightarrow <r> = \int_r R^2(r) r^3 dr$

 \begin{ex}[Atome d'hydrogène dans son état fondamental $\phantom{a}_1H : 1s^1$]
    $$R_{1,0}(r) = (\frac{Z}{a_0})^{\frac{3}{2}} 2e^{-\frac{Z_r}{a_0}}$$

    $$P(r) = R_{n,l}^2(r)r^2 \Rightarrow P(r) = (\frac{1}{a_0})^{3} 4 e^{-\frac{2}{a_0}} r^2$$
      
    $$\frac{d}{dr} R_{n,l}^2(r) r^2 = 8\frac{1}{a_0^3} r e ^{-\frac{2r}{a_0}} (1-\frac{1}{a_0}r)$$

    $$\frac{d}{dr} = 0 \Rightarrow r=0 \text{ ou } r=a_0$$

    $$
    \begin{aligned}
         <r> &= \int_r R_{n,l}^2(r)r^3 dr\\
         &= \int_r (\frac{1}{a_0})^3 4 e^{-\frac{2r}{a_0}}r^3dr\\
         &= \frac{4}{a_0^3} \int_r e^{-\frac{2r}{a_0}} r^3 dr \\
         &= \frac{4}{a_0^3} \frac{3!}{(\frac{2}{a_0})^4}\\
         &= \frac{4}{a_0^3} \frac{6 a_0^4}{16}\\
         &= \frac{24 a_0}{16}\\
         &= \frac{2}{3} a_0
    \end{aligned}
    $$

    $$
    \begin{aligned}
         <r^2> &= \int_r R_{n,l}^2(r) r^4 dr \\
         &= \frac{4}{a_0^3} \int_r e^{-\frac{2r}{a_0}} r^4 dr \\
         &= \frac{4}{a_0^3} \frac{4!}{(\frac{2}{a_0})^5} \\
         &= \frac{24}{32}a_0^2\\
         &= 3a_0^2
    \end{aligned}
    $$

    \emph{Incertitude sur la mesure :} $\Delta r = \sqrt{<r^2> - <r>^2}$\\
    $$
    \begin{aligned}
         \Delta r &= \sqrt{(3a_0^2) - (\frac{3}{2}a_0)^2}\\
         &= \sqrt{3a_0^2 - \frac{9}{4}a_0^2}\\
         &= a_0 \sqrt{\frac{3}{4}} \\
         &= \frac{\sqrt{3}}{2} a_0
    \end{aligned}
    $$
 \end{ex}

 \emph{Moment cinétique et magnétique :}\\

 Toute particule chargée avec un moment cinétique possède également un moment magnétique $\mu$ associé a ce dernier par :
 $$\vec{\mu} = \gamma \vec{L} = g \frac{q}{2m} \vec{L}$$
 où $\gamma$ est le rapport gyromagnétique,\\
 $g$ dépend de la particule et du type de moment cinétique.\\


 \emph{Expérience de Stern et Gerlach (1922)} (étude du spin)\\

 L'électron admet un moment magnétique intrinsèque et donc un moment cinétique intrinsèque : il s'agit du \emph{spin}.\\

 Le spin est une propriété intrinsèque des particules. 
 Le spin a des propriétés similaires à celes d'un moment cinétique.\\

 \emph{spin de l'électron :}\\
\begin{itemize} 
         \item deux nombres quantiques : $s=\frac{1}{2}$ et $m_s = \pm\frac{1}{2}$

         \item norme : 
            $$\hat{s}^2 | s, m_s> = s(s+1) \hbar^2 | s, m_s>$$
            $$s = \sqrt{s(s+1)}\hbar = \sqrt{\frac{3}{4} \hbar^2}$$

         \item projection selon $z$ : $s_z = m_s \hbar$\\
            %fig13(p25)
\end{itemize}

 \emph{moment cinétique orbital :}\\
\begin{itemize} 
         \item deux nombres quantiques : $l$ et $m_l$

         \item norme : 
            $$\hat{L}^2 | l,m_l> = l(l+1) \hbar^2 | l,m_l$$
            $$L = \sqrt{l(l+1)}$$

         \item projection selon $z$ : $L_z = m_l \hbar$
\end{itemize}

Le spin entre dans la description des électrons.\\
On définit alors des fonction spin-orbitales $\Psi_{n,l,m,m_s}$ :\\
$$\Psi_{n,l,m,m_s} = \Psi_{n,l,m} \sigma$$
où $\sigma$ est la fonction de spin\\

Il y a deux fonctions de spin (ou états de spin) possibles :\\
\begin{itemize}
         \item état $\alpha$ : $m_s = +\frac{1}{2}$
         \item état $\beta$ : $m_s = -\frac{1}{2}$
\end{itemize}


\end{document}
