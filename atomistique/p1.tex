\documentclass[../main.tex]{subfile}

%page 1

\begin{document}

\part{Introduction à la chimie quantique}
\section{Les insuffisances de la mécanique classique}
\begin{itemize}
	\item l'effet photoélectrique : \\Première mise en évidence de la dualité onde-corpuscule.
	\item le rayonnement du corps noir : \\Catastrophe UV.
	\item Le spectre d'émission des atomes: \\La discrétion du spectre obtenu ne peut être expliquée par la mécanique classique.
	\\(en mécanique quantique, on a la relation de De Broglie $\Delta E = E_i - E_f = h \nu = \frac{h c}{\lambda}$)
\end{itemize}

\section{Les bases de la mécanique quantique}
\begin{itemize}
	\item La dualité onde-corpuscule (De Broglie) : 
	\\ \`A toute particule qui a une masse on peut attribuer une onde : $\lambda = \frac{h}{m v}$
	\item L'inégalité de Heisenberg : 
	\\ $\Delta q \cdot \Delta p \geq \frac{\hbar}{2}$
	\\ où $\Delta q$ est l'incertitude sur la position,
	\\ et $\Delta p$, l'incertitude sur la quantité de mouvement.
\end{itemize}

\section{Domaine de la chimie quantique}
On se sert des outils de la chimie quantique sur les systèmes plus petits que les protéines. Pour des systèmes dont la taille est plus importante, les propriétés quantiques sont négligeables.

\section{Applications dans les recherches récentes}
\begin{itemize}
	\item dépliement et repliement des enzymes
	\item changements de conformation des molécules en solution
	\item changement de phase
	\item phénomènes d'excitation et de désexcitation dans les complèxes
	\item simulation de spectres RMN 
\end{itemize}

\section{Objectifs de ce cours}
\begin{itemize}
	\item être capable de comprendre et décrire la structure électronique des atomes et des molécules.
	\item comprendre, expliquer et prédire les propriétés structurelles et spéctrométriques ainsi que la réactivité des systèmes chimiques.
\end{itemize}

\end{document}
