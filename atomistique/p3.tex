\documentclass[../main.tex]{subfile}

%page x

\begin{document}

\part{Atomes hydrogénoides: équation de schrodinger}

\begin{defi}
	Les atomes hydrogénoïdes ne sont constitués que d'un noyau et d'un électron
\end{defi}

Energie classique du système:
$$E = T_e + T_N + V_{eN}$$

%### reprendre les expression des energies et des opérateurs correspondant

L'équation dépend du mouvement des deux particules, non solvable.

Deux approches pour simplifier cette équation:
\begin{itemize}
	\item Problème à deux corps, le mouvement est alors décomposé en deux:
\begin{itemize}	
	\item un mouvement de translation de l'atome
	\item un mouvement relatif de l'électron et du noyau
\end{itemize}
	\item Introduction d'une particule fictive de masse $p$ (proche de celle de l'électron)
	\item Approximation de Born-Oppenheimer : du fait de leur large différence de masse, les électrons s'adaptent de façon instantanée et adiabatique à tout mouvement des noyaux:
\begin{itemize}	
	\item On considère le noyau fixe
	\item On ne considère qu'un Hamiltonien électronique
	$\hat{H} = -\frac{\hbar^2}{2m_e} \Delta_e - \frac{1}{4\pi\varepsilon_0}\cdot \frac{Ze^2}{\hat{r}}$
\end{itemize}
\end{itemize}

$V(r)$ étant à symétrie sphérique, nous introduisons le système de coordonnées sphériques:
	$\hat{H} = -\frac{\hbar^2}{2m_er^2} (\frac{\partial}{\partial r} r^2 \frac{\partial}{\partial r}) - \frac{1}{4\pi\varepsilon_0}\cdot \frac{Ze^2}{\hat{r}}$

\end{document}
