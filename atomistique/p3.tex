\documentclass[../main.tex]{subfile}

%page x

\begin{document}

\part{Atomes hydrogénoïdes: équation de Schrödinger}

\begin{defi}
	Les atomes hydrogénoïdes ne sont constitués que d'un noyau et d'un électron
\end{defi}

Énergie classique du système:
$$E = T_e + T_N + V_{eN}$$

%### reprendre les expression des énergies et des opérateurs correspondant

L'équation dépend du mouvement des deux particules, non solvable.

Deux approches pour simplifier cette équation:
\begin{itemize}
	\item Problème à deux corps, le mouvement est alors décomposé en deux:
\begin{itemize}	
	\item un mouvement de translation de l'atome
	\item un mouvement relatif de l'électron et du noyau
\end{itemize}
	\item Introduction d'une particule fictive de masse $p$ (proche de celle de l'électron)
	\item Approximation de Born-Oppenheimer : du fait de leur large différence de masse, les électrons s'adaptent de façon instantanée et adiabatique à tout mouvement des noyaux:
\begin{itemize}	
	\item On considère le noyau fixe
	\item On ne considère qu'un Hamiltonien électronique
	$\hat{H} = -\frac{\hbar^2}{2m_e} \Delta_e - \frac{1}{4\pi\varepsilon_0}\cdot \frac{Ze^2}{\hat{r}}$
\end{itemize}
\end{itemize}

$V(r)$ étant à symétrie sphérique, nous introduisons le système de coordonnées sphériques:
	$\hat{H} = -\frac{\hbar^2}{2m_er^2} (\frac{\partial}{\partial r} r^2 \frac{\partial}{\partial r}) - \frac{1}{4\pi\varepsilon_0}\cdot \frac{Ze^2}{\hat{r}}$

%###redébut de cours (30/09)

	$$\hat{H}\Psi(r, \theta, \varphi) = -\frac{\hbar^2}{2m_er^2} (\frac{\partial}{\partial r} r^2 \frac{\partial}{\partial r} + \Lambda)\Psi(r, \theta, \varphi) - \frac{1}{4\pi\varepsilon_0}\cdot \frac{Ze^2}{r}\Psi(r, \theta, \varphi) = E\Psi(r, \theta, \varphi)$$
	$$\hat{H}\Psi(r, \theta, \varphi) = K\frac{1}{r^2} (\frac{\partial}{\partial r} r^2 \frac{\partial}{\partial r} + \Lambda)\Psi(r, \theta, \varphi) - \frac{1}{4\pi\varepsilon_0}\cdot \frac{Ze^2}{r}\Psi(r, \theta, \varphi) = E\Psi(r, \theta, \varphi)$$
	$$\hat{H}\Lambda = K\frac{1}{r^2} (\frac{\partial}{\partial r} r^2 \frac{\partial}{\partial r} + \Lambda) - \frac{Ze^2}{4\pi\varepsilon_0}\cdot \frac{1}{r}\Lambda$$
	$$\hat{H}\Lambda = K\frac{1}{r^2} (\frac{\partial}{\partial r} r^2 \frac{\partial}{\partial r}\Lambda + \Lambda^2) - \Lambda\frac{Ze^2}{4\pi\varepsilon_0}\cdot \frac{1}{r}$$
	$$\hat{H}\Lambda = K\frac{1}{r^2} (\Lambda\frac{\partial}{\partial r} r^2 \frac{\partial}{\partial r} + \Lambda^2) - \Lambda\frac{Ze^2}{4\pi\varepsilon_0}\cdot \frac{1}{r}$$
	$$\hat{H}\Lambda = \Lambda K\frac{1}{r^2} (\frac{\partial}{\partial r} r^2 \frac{\partial}{\partial r} + \Lambda) - \Lambda\frac{Ze^2}{4\pi\varepsilon_0}\cdot \frac{1}{r}$$
	$$\hat{H}\Lambda = \Lambda \hat{H}$$

	Par conséquent, les opérateurs $\hat{H}$ et $\Lambda$ commutent, 
	ils admettent donc les mêmes jeux de vecteurs propres.\\

	Les fonctions propres de $\lambda$ sont les $Y_{l,m}(\theta, \varphi)$, \\
	$$\Lambda a Y_{l,m}(\theta, \varphi) = -l(l+1) a Y_{l,m}(\theta, \varphi) $$
	$$a Y_{l,m}(\theta, \varphi)$$
	où $a$ est une constante pour l'opérateur de Legendre, c'est à dire une fonction de $r$.\\
	D'où:
	$$\Psi(r, \theta, \varphi) = R(r)Y_{l,m}(\theta, \varphi)$$

	On a donc:
	$$\Psi(r, \theta, \varphi) = R(r)Y_{l,m}(\theta, \varphi)$$
	$$-\frac{\hbar^2}{2mr^2}(\frac{\partial}{\partial r} r^2 \frac{\partial}{\partial r}+ \Lambda) R(r) Y_{l,m}(\theta, \varphi) - \frac{Ze^2}{4\pi\varepsilon_0}\frac{1}{r} R(r)Y_{lm}(\theta, \varphi) = ER(r)Y_{lm}(\theta, \varphi)$$
	$$-\frac{\hbar^2}{2mr^2}(Y_{lm}(\theta, \varphi)\frac{\partial}{\partial r} r^2 \frac{\partial}{\partial r}R(r) - R(r) l(l+1) Y_{lm}(\theta, \varphi) -\frac{Ze^2}{4\pi\varepsilon_0})\frac{1}{r} R(r)Y_{lm}(\theta, \varphi) = ER(r)Y_{lm}(\theta, \varphi)$$
	%###finir l'égalité et reprendre apres les propriétés des fonctions propres de l'équation radiale



	\subsection{Valeurs propres de l'équation de Schrödinger}
	Pour les hydrogénoïdes, on a, pour les solutions de l'équation radiale, 
	$$E_n = - (\frac{m_e e^4}{(4\pi\varepsilon_0)^2\hbar^2}) \frac{Z^2}{2n^2}$$
	L'énergie électronique ne dépend que de $n$ (ni de $l$, ni de $m$)

	en considérant l'état fondamental ($n=1$):
	$$E_1 = - (\frac{m_e e^4}{(4\pi\varepsilon_0)^2\hbar^2}) \frac{Z^2}{2} = - 13,6$$

	%liens entre E_1 et l'energie de première ionisation

	On introduit, pour l'atome d'hydrogène, la constate de Rydberg $\mathcal{R}_H$:
	$$\mathcal{R}_H = ...$$

	Formule de Rydberg:
	$$\frac{1}{\lambda} = \mathcal{R}H ...$$

	\subsection{système d'unité atomique}
	$e=1$, $\hbar = 1$, $4\pi\varepsilon_0 = 1$, $c=1$, $m_e = 1$, $a_0 = 1 = 0,529 \circ{A}$
	D'où:
	$$E_1(H) = -\frac{1}{2} u.a.$$
	$$E_n(Z) = -\frac{Z^2}{2n^2} u.a.$$

	L'unité atomique d'énergie est aussi appelée le Hartree.\\

	l'Hamiltonien pour l'atome d'hydrogène s'écrit donc:
	$$\hat{H} = -\frac{1}{2} \Delta - \frac{Z}{r}$$






	Les OA qui possèdent le même $l$ et le même $n$ constituent une sous couche:
\begin{itemize}	
	\item $l=0$ : sous couche s
	\item $l=1$ : sous couche p
	\item $l=2$ : sous couche d
	\item $l=3$ : sous couche f
	\item $l=4$ : sous couche g
\end{itemize}
	$m$ pouvant prendre $2l+1$ valeurs, pour une sous-couche $nl$ donnée, il y a $2l+1$ OA dans une sous-couche donnée

	Les couches et sous-couches sont des ensembles de fonctions.













	Pour tous les $l \neq 0$, les harmoniques sphériques sont complexes!\\
	Pour obtenir des fonctions réelles:\\
	$$S^+_{l|m|} = \frac{Y_{l,m} + Y_{l,-m}}{\sqrt{2}}$$
	$$S^-_{l|m|} = \frac{Y_{l,m} - Y_{l,-m}}{i\sqrt{2}}$$
	(il s'agit d'une adaptation des formules d'Euler)

	Toutes les OA de type $s$ sont à symétrie sphériques\\



	Les trois harmoniques sphériques de type $p$ sont équivalentes par des rotations de $90°$, \\
	les OA correspondantes auront donc la même forme\\
	Seule leur orientation est différente.\\
	Chaque fonction présente une surface nodale (un plan)\\




	Pour les harmoniques $d$, présente deux plans nodaux ou un cône





	Chaque fonction présente $n - l - 1$ points nodaux
	






	Pour représenter une fonction de trois variables, on a besoin de quatre dimension 
\end{document}
