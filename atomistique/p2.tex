\documentclass[../main.tex]{subfile}

%page 2 (postulats de la meca q)

\begin{document}
\part{Les postulats de la mécanique quantique}
La mécanique quantique se base sur six postulats qui ont toujours été vérifiés. Nous n'en verrons et utiliserons seulement quatre dans ce cours.

\section{Postulat 1}
L'état d'un système à un instant $t$ est complètement défini par la conaissance de sa fonction d'onde, notée $\Psi (\vec{r}, t)$.

\begin{itemize}
	\item Dans $\Psi (\vec{r}, t)$, $\vec{r}$ désigne la variable de l'espace tandis que $t$ désigne la variable du temps.
	\item La fonction d'onde $\Psi(\vec{r}, t)$ n'a aucun sens physique.
	\item $dP(\vec{r}, t) = \Psi^*(\vec{r}, t) \cdot \Psi(\vec{r}, t) dV = |\Psi(\vec{r}, t)|^2 dV$ est une probabilité infinitésimale.
	\\($\Psi^*(\vec{r}, t)$ est la fonction conjuguée de la fonction $\Psi(\vec{r}, t)$)
	\item $\frac{dP(\vec{r}, t)}{dV}$ est une densité de probabilité.
	\item La fonction d'onde est normée : $P(t) = \int\limits_{x}\int\limits_{y}\int\limits_{z} dP(\vec{r}, t) = 1$
	\item La fonction d'onde est une fonction continue, dérivable et de carré sommable. Sa dérivée première doit aussi être continue et dérivable.
	\item La notation de Dirac : \\
	\begin{itemize}
		\item à $\Psi(\vec{r}, t)$, on associe un "ket" $|\Psi(t)>$
		\item à $\Psi^*(\vec{r}, t)$, on associe un "bra" $<\Psi(t)|$
	\end{itemize}
	\item On définit ainsi le produit scalaire de deux fonctions d'ondes par :
	$$\int\limits_{x}\int\limits_{y}\int\limits_{z} \Psi^*(\vec{r}, t) \cdot \Theta(\vec{r}, t) dV = <\Psi(t) | \Theta(t)>$$
\end{itemize}


\section{Postulat 2}
\`A toute grandeur $A$ mesurable, on associe en mécanique quantique un opérateur linéaire et hermitique, noté $\hat{A}$.

\subsection{généralités sur les opérateurs}
\begin{itemize}
	\item Les opérateurs agissent sur les fonctions : $\hat{A}\Psi(\vec{r}, t) = \Theta(\vec{r}, t)$
	\item Cas particulier : l'opérateur est un scalaire : $\hat{A}\Psi(\vec{r}, t) = \lambda \Psi(\vec{r}, t)$
	On dit alors que :\\
	\begin{itemize}
		\item $\lambda$ est une valeur propre de $\hat{A}$
		\item $\Psi(\vec{r}, t)$ est un vecteur propre de $\hat{A}$ associé à la valeur propre $\lambda$
		\item on parle alors d'équation aux valeurs propres.\\
		L'équation de Schrödinger est une équation aux valeurs propres.
	\end{itemize}
	\item Notion de dégénéréscence : \\
	Si deux fonctions différentes donnent, pour un opérateur donné, la même valeur propre, elles sont dites dégénérées.
\end{itemize}
\subsection{Propriétés des opérateurs}
\begin{itemize}
	\item Somme : $\hat{S} = \hat{A} + \hat{B}$
	$$\hat{S}\Psi(\vec{r}, t) = \hat{A} \Psi(\vec{r}, t) + \hat{B} \Psi(\vec{r}, t)$$
	\item Produit (ou composition) : $\hat{P} = \hat{A} \cdot \hat{B}$
	$$\hat{P}\Psi(\vec{r}, t) = \hat{A}(\hat{B} \Psi(\vec{r}, ))$$
	\begin{itemize}
		\item Le produit n'est pas commutatif
		\item On introduit alors le commutateur (un autre opérateur) : 
		$$[\hat{A}, \hat{B}] = \hat{A} \cdot \hat{B} - \hat{B} \cdot \hat{A}$$
		\item Les opérateurs $\hat{A}$ et $\hat{B}$ commutent si et seulement si $[\hat{A}, \hat{B}]$ est nul.
		\item Deux opérateurs qui commutent admettent le même ensemble de fonctions propres.
	\end{itemize}
	\item Linéarité : $\hat{A}(\alpha \Psi(\vec{r}, t) + \beta \Theta(\vec{r}, t)) = \alpha \hat{A} \Psi (\vec{r}, t) + \beta \hat{A} \Theta(\vec{r}, t)$
	\begin{itemize}
		\item Démontrons que toute combinaison linéaire de fonctions propres dégénérées d'un opérateur est également fonction propre de cet opérateur avec la même valeur propre.
		\begin{proof}
			Pour $\hat{A}$, un opérateur,
			$$
			\begin{aligned}
				\hat{A} \cdot \sum\limits_{i} \alpha_i \Psi_i &= \sum\limits_i \alpha_i \hat{A} \Psi_i\\
				&= \sum\limits_i \alpha_i \lambda \Psi_i \ \ \ \text{ les fonctions $\Psi_i$ ayant toutes la même valeur propre $\lambda$}\\
				&= \lambda \sum\limits_i \alpha_i \Psi_i
			\end{aligned}
			$$
		\end{proof}
	\end{itemize}
	\item Hermiticité : ces valeurs valeurs propres sont réelles et ses vecteurs propres sont orthogonaux.
\end{itemize}
\subsection{Principe de correspondance}

	En mécanique quantique, tout opérateur peut être construit à partir des opérateurs position et quantité de mouvement:
\begin{itemize}
	\item opérateur position : associé à la coordonnée $q_i$ d'une particule, il consiste à multiplier par la variable $q_i$, $\hat{q_i} = q_i \cdot$:
	$$
	\left\{
\begin{array}{l}
	\hat{x} = x \cdot \\
	\hat{y} = y \cdot \\
	\hat{z} = z \cdot 
\end{array}
	\right .
	$$
	\item opérateur quantité de mouvement : associé à la coordonnée $p_i$ d'une particule, il consiste à dériver par rapport à cette coordonnée et à multiplier par $-i\hbar$ : $\hat{p_i} = -i\hbar \cdot \frac{\partial}{\partial q_i}$
	$$
	\left\{
\begin{array}{l}
	\hat{p_x} = -i\hbar \cdot \frac{\partial}{\partial x}\\
	\hat{p_y} = -i\hbar \cdot \frac{\partial}{\partial y}\\
	\hat{p_z} = -i\hbar \cdot \frac{\partial}{\partial z}
\end{array}
	\right.
	$$
\end{itemize}


\section{Postulat 3}
L'évolution d'un système est régie par l'équation de Schrödinger dépendante du temps.


\section{Postulat 4}
Les valeurs de $A$ mesurées expérimentalement ne peuvent être que des valeurs propres de son opérateur $\hat{A}$.



\end{document}
