\documentclass[../main.tex]{subfile}

%page 1

\begin{document}

\part{Séries numériques}
On considère une suite (réelle ou complexe) $(u_n)_{n\in \NN}$ et on étudie la somme infinie $\sum\limits_{n = 0}^{+\infty} u_n$.\\
\begin{rema}
	Question:\\
	Est-ce que cette somme est bien définie, c'est-à-dire appartient à $\RR$ ou $\CC$.
\end{rema}

\begin{ex}
\begin{itemize}
	\item $u_n = 1$, $\sum\limits_{n = 0}^\infty u_n = +\infty$
	\item $u_n = n$, $\sum\limits_{n = 0}^\infty u_n = +\infty$
	%### trouver un moyen d'écrire une factorielle \item $u_n = \frac{1}{\fact(n)}$, $\sum\limits_{n = 0}^\infty u_n = e$
	\item $u_n = \frac{1}{n}$, $\sum\limits_{n = 0}^\infty u_n = +\infty$
	\item $u_n = \frac{1}{n^2}$, $\sum\limits_{n = 0}^\infty u_n = \frac{\pi^2}{6}$
\end{itemize}
\end{ex}

\section{Quelques rappels et compléments sur les suites}
Une suite $(u_n)_{n\in \NN}$ converge s'il existe $l \in \RR$ tel que $\forall \varepsilon > 0, \ \exists N \in \NN, \ \forall n \geq N, \ |u_n - l| < \varepsilon$.\\
On dit que la suite $(u_n)_{n\in \NN}$ est une suite de Cauchy si $\forall \epsilon > 0, \ \exists N \in \NN, \ \forall n \geq N, \ \forall m \geq N, \ |u_n - u_m| < \epsilon$.\\

\begin{prop}
	Toute suite convergente est de Cauchy.
\end{prop}

\begin{prop}
	Dans $\RR$ ou dans $\CC$, toute suite de Cauchy est convergente.
\end{prop}

On dit que $\RR$ (ou $\CC$) est un espace métrique complet.\\

	\'Equivalents : \\
	Si $u_n = n^2$ et $v_n = n^3$, on a $\lim\limits_{n \to \infty} u_n = + \infty$ et $\lim\limits_{n \to + \infty} v_n = + \infty$.\\
	Cependant, $(v_n)_{n\in \NN}$ est plus rapide que $(u_n)_{n\in \NN}$. Les équivalents servent à comparer des vitesses de convergence.\\

\begin{defi}
	Deux suites $(u_n)_{n\in \NN}$ et $(v_n)_{n \in \NN}$ sont équivalentes si $\lim\limits_{n \to \infty} \frac{u_n}{v_n} = 1$.\\
	On note $u_n \sim v_n$
\end{defi}

Dans ce cas, les deux suites ont la même vitesse de convergence.\\
Cela n'a aucun intérêt que si $(u_n)$ tend vers $0$, $+\infty$ ou $-\infty$\\

\begin{ex}
	Si $\lim\limits_{n \to \infty} a_n = 0$, alors, \\
\begin{itemize}
	\item $e^{a_n} -1 \sim a_n$
	\item $e^{a_n} -1 \sim a_n$
	%### finir exemples et opérations
\end{itemize}
\end{ex}

\part{Séries numériques}
On prend une suite réelle (ou complexe) $(u_n)_{n \in \NN}$.\\
La somme infinie $\sum\limits_{n \geq 0} u_n$ est la série de terme général $u_n$.\\

On lui associe la suite des sommes partielles $(S_n)_{n \in \NN}$, définie par $S_n = \sum\limits_{k = 0}^n u_k$.\\
On dit que la série $\sum\limits_{n \geq 0} u_n$ converge si la suite $(S_n)_{n \in \NN}$ converge dans $\RR$ (elle a une limite et cette limite est réelle). 
Sinon, la série diverge.\\

Si la série converge, son reste est la suite $(R_n)_{n \in \NN}$ définie par $R_n = \sum\limits_{k = n+1}^{+\infty} u_k$ et $\sum\limits_{k = 0}^{+\infty}$ est la somme de la série.\\
On a $S_n + R_n = \sum\limits_{k=0}^{+\infty} u_k$.\\
Notons que $\lim\limits_{n \to +\infty} R_n = 0$.

\begin{exerc}
	(\'A savoir), la série géométrique $u_n = a^n$ avec $a \in \RR$, \\
	$$S_n = \sum\limits_{k = 0}^n u_k = \sum\limits_{k = 0}^n a^k = \frac{1 - a^{n + 1}}{1 - a}$$
\begin{itemize}	
	\item Si $a = 1$, on a $u_n = 1$ pour tout entier naturel $n$, 
	donc $\sum\limits_{n \geq 0} u_n$ converge.

	\item Si $a=-1$, $S_n = \frac{1-(-1)^{n+1}}{2}$ n'a pas de limite et donc $\sum\limits_{n \geq 0} u_n$ diverge.

	\item Si $-1 < a < 1$ (ou $|a| < 1$), $S_n = \frac{1 - a^{n+1}}{1-a}$ %### finir l'exemple
\end{itemize}
\end{exerc}

En résumé, $\sum\limits_{n \geq 0} a^n$ converge si et seulement si $|a| < 1$ et si $|a| < 1$, alors $\sum\limits_{n=0}^{+\infty} a^n = \frac{1}{1 - a}$

\begin{prop}
	Critère de Cauchy : \\
	$\sum\limits_{n \geq 0} u_n$ converge si et seulement si la suite $(S_n)_{n \in \NN}$ converge. c'est-à-dire $\forall \varepsilon > 0, \ \exists N \in \NN, \ \forall n \geq N, \forall, \ \forall m \geq N, \ |S_n - S_m| < \varepsilon$ ou encore %###compléter
\end{prop}

Application:\\
La série harmonique $\sum\limits_{n \geq 1} \frac{1}{n}$.\\
On montre que cette série est divergente.\\
On note $S_n = \sum\limits_{k = 1}^n \frac{1}{k}$\\

Pour tout entier $n \geq 1$, on a : \\
$$S_{2n} - S_n = \frac{1}{n + 1} + ... + \frac{1}{2n} \geq \frac{1}{2n} .. $$




%### à compléter




%cours de mercredi


%correction d'exo

\section{Séries à termes positifs}

On suppose que $u_k \geq 0$ pour tout $n \in \NN$ (ou du moins à partir d'un certain rang).
Dans ce cas, la suite des sommes partiels $(S_n)_{n \in \NN}$ est positive et croissante.
Par conséquent, $(S_n)_{n \in \NN}$ converge si et seulement si elle est majorée.
Dans le cas contraire, on a $\lim\limits_{n \to \infty} S_n = + \infty$.

\begin{theo}
	$\sum\limits_{k \geq 0} u_n$ converge si et seulement si $(S_n)_{n \in \NN}$ est majoré.
\end{theo}

\subsection{Comparaison entre séries et intégrales généralisées}

\begin{theo}
	Soit $f: \RR \to \RR$, une fonction décroissante et positive.
	Alors, la suite $(U_n)_{n \in \NN}$ définie par $U_n = \sum\limits_{k = 0}^n f(k) - \int_0^n f(t)dt$ converge.\\
	Par conséquent, la série $\sum\limits_{n \geq 0} f(n)$ et l'intégrale $\int_0^{+\infty} f(t) dt$ ont la même nature.
\end{theo}

\begin{proof}
	Pour tout $k \in \NN$, on a :
	%### insérer fig 31
	$$f(k+1) \cdot 1 \leq \int_k^{k+1} f(t) dt \leq f(k) \cdot 1$$
	On montre alors que $(U_n)_{n \in \NN}$ est minorée par $0$.\\
	On a :
	$$
\begin{aligned}
	U_n &= f(n) + \sum\limits_{k = 0}^{n-1} f(k) - \sum\limits_{k=0}^{n-1} \int_{k}^{k+1} f(t) dt\\
	&= f(n) + \sum\limits_{k = 0}^{n-1} (f(k) - \int_{k}^{k+1} f(t) dt)\\
\end{aligned}
	$$
	avec $f(n) \geq 0$ et $f(k) - \int_k^{k+1} f(t)dt \geq 0$.\\
	Donc, $U_n \geq 0$ pour tout $n \in \NN$.\\
	On montre maintenant que $(U_n)_{n \in \NN}$ est décroissante.\\
	$$
\begin{aligned}
	U_{n+1} - U_n &= \sum\limits_{k =0}^{n+1} f(k) - \int_0^{n+1} f(t)dt - \sum\limits_{k=0}^n f(k)\\
	&= f(n+1) - \int_n^{n+1} f(t)dt\\
	&\leq 0
\end{aligned}
	$$
	Donc $U_{n+1} \leq U_n$.\\

	En résumé, on a :
	$$\int_0^n f(t)dt + U_n = \sum\limits_{k=0}^n f(k)$$
	avec $\lim\limits_{n \to \infty} U_n = l \in \RR$\\
	Si $\int_0^{+\infty} f(t)dt$ converge, alors, $\lim\limits_{n\to \infty} \int_0^n f(t)dt \in \RR$ et donc $\lim\limits_{n \to \infty} \sum\limits_{k=0}^n f(k) \in \RR$
	Si $\int_0^{+\infty} f(t)dt$ diverge, alors, $\lim\limits_{n\to \infty} \int_0^n f(t)dt = +\infty$ et donc $\lim\limits_{n \to \infty} \sum\limits_{k=0}^n f(k) = +\infty$

	Réciproquement, si la somme converge, alors l'intégrale converge et si la somme diverge, alors l'intégrale diverge.\\
\end{proof}

Application:\\
Pour les séries de Riemann, si $\alpha \in \RR$, \\
$\sum\limits_{n \geq 1} \frac{1}{n^\alpha}$ converge si et seulement si $\alpha > 1$.\\



\subsection{Critères de comparaison}
On considère deux suites positives $(u_n)_{n\in \NN}$ et $(v_n)_{n \in \NN}$.\\

\begin{theo}
	On suppose que $0 \leq u_n \leq v_n$ pour tout $n \in \NN$, \\
	alors:
\begin{itemize}	
	\item Si $\sum_{n \leq 0}v_n$ converge alors $\sum\limits_{n \leq 0} u_n$ converge
	\item Si $\sum_{n \leq 0}v_n$ diverge alors $\sum\limits_{n \leq 0} u_n$ diverge
\end{itemize}
\end{theo}

\begin{proof}
	$$S_n = \sum\limits_{k=0}^n u_k$$
	$$S'_n = \sum\limits_{k=0}^n v_k$$
	On a $0 \leq S_n \leq S'_n$\\
	$$\sum$$
	%###finir la démo
\end{proof}

\begin{corrol}
	S'il existe $a$ et $b$ strictements positifs tels que $a \leq \frac{u_n}{v_n} \leq b$ pour tout $n \in \NN$, 
	alors $\sum\limits_{n \geq 0} u_n$ et $\sum\limits_{n \geq 0} v_n$ ont la même nature.
\end{corrol}

\begin{proof}
	$$av_n \leq u_n \leq bv_n$$
\end{proof}

\begin{corrol}
	Si on a $u_n \sim v_n$, alors $\sum\limits_{n \geq 0} u_n$ et $\sum\limits_{n \geq 0} v_n$ ont la même nature.

	De plus, si $\sum\limits_{n \geq 0} u_n$ et $\sum\limits_{n \geq 0} v_n$ convergent, alors les restes sont équivalents.\\
	$R_n = \sum\limits_{k = n + 1}^\infty u_k \sim R'_n = \sum\limits_{k = n+1}^\infty v_k$\\
	Dans le cas où elles divergent, les sommes partielles sont équivalentes:
	$$S_n = \sum\limits_{k=0}^n u_k \sim S'_n = \sum\limits_{k=0}^n v_k$$
\end{corrol}

\begin{proof}
	Si $u_n \sim v_n$, alors, $\lim\limits_{n \to \infty} \frac{u_n}{v_n} = 1$\\
	Donc, $\exists N \in \NN, \ \forall n \leq N, \ 0,5 \leq \frac{u_n}{v_n} \leq 1,5$\\
	D'où $0,5 v_n \leq u_n \leq 1,5 v_n$
\end{proof}











\begin{ex}
\begin{itemize}	
	\item 
\end{itemize}
\end{ex}

\subsection{Comparaison avec une suite géométrique}

\begin{rap}
	Si $a \in \RR$ (ou $\CC$), alors $\sum\limits_{n \geq 0} a^n$ converge si et seulement si $|a| < 1$.\\
	Et dans ce cas, on a;
	$$\sum\limits_{n=0}^{+\infty} a^n = \frac{1}{1 - a}$$
\end{rap}

On suppose que $u_n \geq 0$ pour tout $n \in \NN$.\\

\begin{theo}{Règle de Cauchy}
	On suppose que $\lim\limits_{n \to \infty} (u_n)^{\frac{1}{n}} = l \geq 0$.\\
	Alors :
\begin{itemize}	
	\item Si $l < 1$, alors $\sum\limits_{n \geq 0} u_n$ converge
	\item Si $l > 1$, alors $\sum\limits_{n \geq 0} u_n$ diverge grossièrement
\end{itemize}
\end{theo}

\begin{rema}
	Si $l=1$, tout peut arriver.\\
	$u_n = \frac{1}{n^\alpha}$

$$
\begin{aligned}
	(u_n)^{\frac{1}{4}} &= (\frac{1}{n^\alpha})^{\frac{1}{4}}\\
	&= \frac{1}{n^{\frac{\alpha}{n}}}\\
	&= \frac{1}{e^{\frac{\alpha}{n}}\ln n}\\
	&= e^{-\frac{\alpha}{n}\ln n=}\\
\end{aligned}
$$
Comme $\lim\limits_{n \to +\infty} \frac{\ln n}{n} = 0$\\
On a $\lim\limits_{n \to +\infty} e^{-\frac{\alpha}{n}\ln n} = e^0 = 1$.
On a $\lim\limits_{n \to +\infty} (u_n)^{\frac{1}{n}} = 1$ pour tout $\alpha$.\\
mais si $\alpha > 1$, $\sum u_n$ converge\\
et si $\alpha \leq 1$, $\sum u_n$ diverge\\

\begin{proof}
	On a $\lim\limits_{n \to \infty} (u_n)^{\frac{1}{n}} = l$, donc, \\
	%### compléter la démo
\begin{itemize}	
	\item 
	\item Si $l > 1$, \\
	On prend $\varepsilon > 0$, tel que $1 < l - \varepsilon$.\\
	Il existe $N \in \NN$ tel que $\forall n \geq N, \ 1 < l - \varepsilon <  (u_n)^{\frac{1}{n}}$\\
	Donc $u_n > (l - \varepsilon)^n$ pour tout $n \geq N$.\\
	Comme $l - \varepsilon > 1$, on a $\lim\limits_{n \to \infty} (l - \varepsilon)^n = +\infty$\\
	et donc $\lim\limits_{n \to \infty} u_n = + \infty$
\end{itemize}
\end{proof}
\end{rema}

\begin{theo}{Règle de d'Alembert}
	On suppose que $u_n > 0$ à partir d'un certain rang.\\
	On suppose de plus que $\lim\limits_{n \to \infty} \frac{u_{n+1}}{u_n} = l$\\
	Alors : 
\begin{itemize}	
	\item Si $l < 1$, alors $\sum u_n$ converge
	\item Si $l > 1$, alors $\sum u_n$ diverge grossièrement
\end{itemize}
\end{theo}

\begin{rema}
	Si $l = 1$, on ne peut rien dire:
	$u_n = \frac{1}{n^\alpha}$\\
	$$\lim\limits_{n \to \infty} \frac{u_{n+1}}{u_n}$$
\end{rema}

\begin{proof}
	$\lim\limits_{n \to \infty} \frac{u_{n+1}}{u_n} = l$\\
	$$\forall \varepsilon > 0, \ \exists N \in \NN, \ \forall n \geq N, \ l-\varepsilon < \frac{u_{n+1}}{u_n} < l + \varepsilon$$
\begin{itemize}	
	\item Si $l < 1$, on prend $\varepsilon > 0$ tel que $l+ \varepsilon < 1$. 
	Il existe $N \in \NN$ tel que $\forall n \geq N, \ 0 \leq \frac{u_{n+1}}{u_n} \leq l + \varepsilon$.
	Si $n > N$, on a :
	$$\prod\limits_{k = N}^{n-1} \frac{u_{k+1}}{u_k} = \frac{u_n}{u_N} < (l + \varepsilon)^{n - N}$$
	On obtient alors 
	$$0 \leq u_n \leq u_N$$
	$$(l + \varepsilon)^{n - N}$$%###à finir
	%### finir le point 1
	\item Idem 
\end{itemize}
\end{proof}


\subsection{Séries à termes de signe quelconque}
\subsubsection{Séries absolument convergentes}
\begin{defi}
	On dit que la série $\sum\limits_{n\geq 0} u_n$ est absolument convergente si $\sum\limits_{n\geq 0} |u_n|$ converge.
\end{defi}

\begin{theo}
	Si la série $\sum\limits_{n \geq 0} u_n$ converge absolument alors elle converge.
\end{theo}

\begin{proof}
	$\sum\limits_{n \geq 0} |u_n|$ convergeant, la série vérifie le critère de Cauchy :\\
	Si on prend $\varepsilon > 0$, il existe $N \in \NN$ tel que :
	$$\forall n \geq N, \ \forall p \in \NN, \ \left| \sum\limits_{k = n}^{n+p} |u_k| \right| < \varepsilon$$
	On a donc, pour tout $n \geq N$, et pour tout $p \in \NN$, 
	$$\left| \sum\limits_{k=n}^{n+p} u_k \right| \leq \sum\limits_{k=n}^{n+p} |u_k| < \varepsilon$$
	Donc la série $\sum\limits_{n \geq 0}u_n$ vérifie le critère de Cauchy,
	donc elle converge.
\end{proof}

\begin{rema}
	Attention, la réciproque est fausse :\\
\begin{ex}	
	$u_n = \frac{(-1)^n}{n}$\\
	$\sum\limits_{n\geq 1} \frac{(-1)^n}{n}$ converge\\
	mais $\sum\limits_{n\geq 1} \left| \frac{(-1)^n}{n}\right|$ converge\\
\end{ex}
\end{rema}

\subsubsection{Séries alternées}
	Ce sont des séries de la forme $\sum\limits_{n \geq 0} (-1)^na_n$ où $a_n \geq 0$ pour tout $n \in \NN$.

\begin{theo}	
	Si la suite $(a_n)_{n\in \NN}$ est décroissante et converge vers $0$, alors $\sum\limits_{n \geq 0} (-1)^n a_n$ converge. \\
	De plus, si on note $R_n = \sum\limits_{k = n + 1}^{\infty} (-1)^ka_k$, 
	alors on a $|R_n| \leq a_{n+1}$
\end{theo}

\begin{proof}
	La preuve sera faite en exercice
\end{proof}

\begin{ex}
	$\sum\limits_{n\geq 1} \frac{(-1)^n}{n}$ converge.\\
\end{ex}

\subsubsection{Règle d'Abel}
\begin{theo}
	Soient deux suites $(\alpha_n)_{n \in \NN}$ et $(u_n)_{n \in \NN}$ qui vérifient :
\begin{enumerate}	
	\item $(\alpha_n)_{n \in \NN}$ est positive, décroissante et converge vers $0$.
	\item Il existe $M > 0$ tel que 
	$$\forall n \in \NN, \ \left| \sum\limits_{k = 0}^n u_k \right| \leq M$$
\end{enumerate}
	Alors $\sum\limits_{n \geq 0} \alpha_n u_n$ converge.
\end{theo}

\begin{proof}
	La preuve sera faite en exercice
\end{proof}

\begin{ex}
	$\sum\limits_{n \geq 1} \frac{\cos n}{n}$\\
	On a $\frac{\cos n}{n} = \frac{1}{n} \cos n$.\\
\begin{itemize}	
	\item $(\frac{1}{n})_{n \geq 1}$ est positive, décroissante et converge vers $0$.
	\item On montre qu'il existe $M > 0$ tel que pour tout $n \in \NN^*$, on a :
	$$\left| \sum\limits_{k =1}^n \cos(k) \right| \leq M$$

	On écrit $\cos(k) = \Re (e^{ik})$\\
	Donc, 
	$$
\begin{aligned}
	\left| \sum\limits_{k=1}^n \cos(k) \right| &= \left| \sum\limits_{k=1}^n \Re(e^{ik}) \right|\\
	&= \left| \Re(\sum\limits_{k=1}^n e^{ik}) \right|\\
	&= \left| \Re(\sum\limits_{k=0}^n e^{ik}) -1 \right|\\
\end{aligned}
	$$
	On sait aussi que 
	$$\sum\limits_{k=0}^n e^{ik} = \sum\limits_{k=0}^n (e^i)^k = frac{1 - e^{i(n+1)}}{1 - e^i}$$
\begin{rap}	
	Si $z \in \CC$, alors, $|\Re(z)\| \leq |z|$ et $|\Im(z)| \leq |z|$
\end{rap}
	On en déduit :
	$$\left| \sum\limits_{k=1}^n \cos(k) \right| 
	\leq \left| \Re(\sum\limits_{k=0}^n e^{ik}) \right| + 1
	\leq \left| \sum\limits_{k=0}^n e^{ik} \right| + 1 $$

	Donc:
	$$\left| \sum\limits_{k=1}^n \cos(k) \right| 
	\leq \left| frac{1 - e^{i(n+1)}}{1 - e^i} \right| + 1
	\leq frac{|1 - e^{i(n+1)}|}{|1 - e^i|} + 1
	\leq frac{|1| + |e^{i(n+1)}|}{|1 - e^i|} + 1
	\leq frac{2}{|1 - e^i|} + 1
	$$
\end{itemize}
\end{ex}

\begin{exerc}	
	Utilisation du Théorème d'Abel pour montrer le théorème des séries alternées:
	$$\sum\limits_{n \geq 0} (-1)^n a_n$$
	avec $(a_n)_{n \in \NN}$ décroissante et converge vers $0$.\\

	Il suffit de vérifier qu'il existe $M > 0$ tel que 
	$$\forall n \in \NN, \ \left| \sum\limits_{k=0}^n (-1)^k \right| \leq M$$

	Selon la parité de $n$,	$\left| \sum\limits_{k=0}^n (-1)^k \right|$ vaut $0$ ou $1$.\\
	Donc $\left| \sum\limits_{k=0}^n (-1)^k \right| \leq 1$ pour tout $n \in \NN$

\end{exerc}



\end{document}





